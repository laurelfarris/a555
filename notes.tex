\documentclass[12pt]{article}
\usepackage[margin=1in]{geometry}
\setlength{\parindent}{0em}
\setlength{\parskip}{0.5em}
\usepackage{enumerate}
\usepackage{hyperref}
\usepackage{color}
\usepackage{mdwlist}
\usepackage{amsmath}
\usepackage{amssymb}
\usepackage{marvosym}

%\usepackage{titlesec}
%\titleformat*{\section}{}
\renewcommand{\thesection}{\Roman{section}}

\begin{document}

\textcolor{red}{\textbf{Important things to know}:
\begin{itemize*}
    \item Derive the Virial Theorem
    \item Schecter function
    \item Sersic relation
    \item deVouceleurs
    \item Fundamental Plane
    \item Tully Fisher relation
    \item Kormendy relation
    \item isophotes: boxy vs.\ disky
    \item inner SB profiles: cuspy vs.\ core
\end{itemize*}
}

\section{History}
\begin{enumerate}
  \item 1700's: Messier objects, 39 of
  which are actually galaxies (total of 110 in the final catalogue).
  \item 1864: GC, 1888: NGC (William Herschel and son John)
  \item 1920's: `Great Debate' between Curtis and Shapley on whether
  or not galaxies were located within the MW\@. Resolved by Hubbles
  discovery of Cepheids in M31.
  \item 1980's: Importance of environment recognized:
  \textbf{morphology/density relation}; `Nature vs. Nurture'.
  \item 1990's: Techniques for finding and confirming high redshift
  galaxies (z>2): \textbf{Lyman break galaxies}:
  \begin{itemize}
    \item Lyman limit
    \item Rydberg formula: $\frac{1}{\lambda}=
    R\Big(\frac{1}{n_u^2}-\frac{1}{n_l^2}\Big) $
  \end{itemize}
  First large scale surveys, both nearby and at medium redshift. HST
  imaging of distant galaxies. N-body (dark matter) simulations.
  \item 2000's:Precision Cosmology and LCDM, outside optical
  wavelengths: IR (Spitzer) and sub-mm (JCMT).
  \item 2010's: Extended gas halos in galaxies. Possibly detection of
  DM and dark energy.
\end{enumerate}

\section{Approaches}
    Multifaceted approach to studying galaxy formation and evolution:
    \subsection*{Galaxy ``archaeology''}
    Study nearby galaxies in detail,
    attempt to understand processes that led to their current
    appearence.
            \begin{itemize}
                \item Advantages: can resolve structure, individual stars
                    in nearest galaxies, high S/N observations
                \item Disadvantages: some information may be erased by
                    physical processes (e.g.\ merging), degeneracies in
                    integrated light
            \end{itemize}
   \subsection*{Distant galaxies}
   look at galaxy samples at different lookback
   times, study distribution of properties (galaxy population) as
   a function of time.
            \begin{itemize}
                \item Advantages: direct probe of different stages.
                    Relationship between lookback time and redshift.
                \item Disadvantages: brightness/selection effects, lack
                    of detail, difficulty in associating objects at one redshift
                    to those at another.
            \end{itemize}
   \subsection*{Physics of galaxy formation}
            \begin{itemize}
                \item Advantages: some physics (e.g.\ gravity) is well
                    understood
                \item Disadvantages: some physics (e.g.\ star formation)
                    is not. Dyanamic range of the problem is huge.
                    \begin{itemize}
                        \item Dynamic range in distances, from stellar scales
                            to largest scale structure.
                        \item Dynamic range in mass, from stellar scales
                            (1 $M_{\odot}$ to
                            largest scale structure ($10^6-10^{15} M_{\odot}$)
                    \end{itemize}
            \end{itemize}

\section{Overview of galaxies and galaxy formation}
Components:
  \begin{itemize*}
    \item Dark Matter: usually non-baryonic, but some baryonic matter
    can be hard to see, such as brown dwarfs. Dominates mass of
    galaxies.
    \item Stars: observed properties depend primarily on mass, age,
    and composition. Variety leads to multiple luminosities and colors
    in galaxies.
    \item ISM: molecular, atomic, and ionized gas phases; dust; mass
    of ISM varies widely between galaxies.
    \item Central black holes
  \end{itemize*}
Processes
  \begin{itemize*}
    \item Gravitational collapse (of dark matter, and later, baryons)
    in cosmological framework.
    \begin{itemize}
      \item How big are initial lumps at different size scales?
      \item How much angular momentum?
      \item How fast do lumps grow?
    \end{itemize}
    \item Condensation of gas and cooling
    \begin{itemize}
      \item ``hot'' vs. ``cold'' accretion
    \end{itemize}
    \item Star Formation (not well understood)
    \begin{itemize}
      \item Under what conditions do stars form?
      \item What types (masses) of stars form?
      \item Drives chemical evolution, which may impact
      cooling and future star formation
    \end{itemize}
    \item Black hole formation
    \begin{itemize}
      \item Primordial formation vs.\ formation from early stars
      \item How common?
    \end{itemize}
    \item Feedback/mass loss
    \begin{itemize}
      \item How much energy? Does mass escape or just delay accretion?
      \item What objects generate it? Winds, supernovae, galactic
      nucleii? How?
    \end{itemize}
    \item Continued accretion from IGM
    \begin{itemize}
      \item How much?
      \item What mode?
      \item What composition?
    \end{itemize}
    \item Merging: Gas-rich vs.\ gas-poor
    \item Cluster (group?) environment: ram pressure, tides
    \item Dynamical evolution
    \begin{itemize}
      \item Dynamical instabilities
      \item Migration
      \item Internal vs.\ external triggers
    \end{itemize}
  \end{itemize*}
These processes have characteristic timescales, and the relation between
them may influence how galaxies form, evolve, and appear.

\subsection*{Example}
Overly simple, but illustrates the consequences of
this scenrio (what scenerio???)). Galaxies range size from
10$^{11}$ - 10$^{12} M_{\odot}$.
Self-gravitating cloud has two timescales:
\begin{enumerate}[1.]
        \item dynamical, or free-fall:
            $$ t_{dyn} \sim (G\rho)^{-1/2} $$
        \item cooling time:
            $$ t_{cool} \sim nkT_g/n^2\Lambda(T) $$
            where $\Lambda$ is the cooling function
\end{enumerate}
If $t_{cool} > t_{dyn}$, then a cloud can be in quasi-static
equilibrium, i.e.\ cooling is unimportant. If $t_{cool} < t_{dyn}$
the cloud cools, kinetic energy is converted to radiation, and
the cloud collapses. Given a cooling curve for primordial
composition, one can calculate the relevant timescales, and find
that collapse is unlikely to occur for $M > 10^{12}M_{\odot}$.
This implies that \emph{dissipation} is important, at least for
objects we observe as galaxies (e.g.\ luminous objects).

This argument is really only a suggestion, for a number of reasons:
halos are not uniform density, so there's no such thing as a single
cooling time for the entire halo.

\textcolor{red}{More Questions: Add these to the ones above, and keep them in mind
while studying. Read with a question in mind, connect old
information to new information. Add main bullets first, then go
through and add details, depending on how much time you have.}

\subsection*{Questions}
\begin{itemize*}
    \item When do each of these steps happen and what are their
        relative importances?
    \item What sets the masses of galaxies? Sizes?
        ($10^6-10^{12}M_\odot$) Luminosities?
    \item What sets the distributions of numbers of galaxies as a
        function of mass/luminosity?
    \item Does the ratio of baryonic mass/total mass change for
        different galaxies?
    \item What triggers star formation in galaxies?
    \item What is responsible for the rango of galaxy morphology?
    \item How much of present structure is determined by initial
        conditions, e.g.\ initial overdensity, angular momentum (and what
        are those initial conditions)?
    \item How much does present appearance depend on basic physics
        within galaxies, e.g.\ dynamics and chemical evolution?
    \item How much depends on environment, e.g.\ mergers and
        interactions, background radiation?
    \item Does the relative importance of these effects (initial
        conditions, internal evolution, environment) vary for different
        galaxies?
\end{itemize*}
\newpage
% ---------------------------------page 6-----------------------------------%
\begin{centering}
    \begin{raggedright}
    {\LARGE\textbf{Broad overview/review of nearby galaxy population}}
    \end{raggedright}
\end{centering}
\section*{Morphological classification}
Historically, galaxies were considered in terms of their morphology,
i.e.\ the Hubble sequence.
However, it is not totally clear to what extent morphological classification
traces underlying physics, and descriptive morphology may be biased by things
that may not be fundamental.
Still, it is widely used, so important to understand a bit.

\subsection*{Morphological systems}
Good reference: Sandage in Galaxies and the Universe, 1975.
Some pictures:
\begin{itemize*}
    \item \href{http://astronomy.nmsu.edu/holtz/a555/html/diagrams/a616/ellips.htm}
        {\textcolor{blue}{ellipticals}}
    \item \href{http://astronomy.nmsu.edu/holtz/a555/html/diagrams/a616/s0.htm}
        {\textcolor{blue}{S0s}}, also known as lenticulars
    \item \href{http://astronomy.nmsu.edu/holtz/a555/html/diagrams/a616/spirals.htm}
        {\textcolor{blue}{spirals}}
    \item \href{http://astronomy.as.virginia.edu}
        {\textcolor{blue}{Tuning fork diagram}}
\end{itemize*}

\subsection*{Hubble classification}
\begin{itemize*}
           \item ellipticals: given as En, where $n = 10(\frac{1-b}{a})$.
              E.g., E0 (sphere) $\rightarrow$ E7 (skinny).
              No distinction between dwarf ellipticals and dwarf spheroidals
           \item spirals: barred or unbarred (SBa,SBb,SBc; or Sa,Sb,Sc;
              respectively).
              where a,b,c denote size of bulge (bulge-to-disk ratio: B/D)
              in decreasing order. Also classified according to tightness of
              arms and the degree to which arms are resolved into HII regions.
              Later, have SA (normal spirals), SB (barred), and SAB (transition)
            \item SOs (aka lenticulars, or disks): intermediate, no spiral
                structure, split into $S0_1$, $S0_2$, $S0_3$, depending on the
                amount of dust.
            \item Irregulars: Irr I (Magellenic irregulars with lots of
                 distinct HII regions) and Irr II (lack the resolution into
                distinct HII regions).
            \item Some pictorial examples
\end{itemize*}

Based on global characteristics, the Hubble morphological classification of
ellipticals is probably not fundamental.
It may be more meaningful to classify by \emph{isophotal shape}
(\href{http://astronomy.as.virginia.edu}
{\textcolor{blue}{(Kormendy and Bender classification)}})
or \emph{kinematically} by $v/\sigma$.

\subsection*{deVaucouleurs/RC3 classification}
\href{http://astronomy.nmsu.edu/holtz/a555/html/diagrams/a616/rc3class.htm}
{\textcolor{blue}{diagram.}}
See also
\href{http://astronomy.as.virginia.edu}
{\textcolor{blue}{here}}.
\begin{itemize*}
    \item Extends Hubble to later spiral types Sd, Sm, and finally Im.
    \item Extra classes around $S0$ ($S0^{-}, S0^{+}$)
    \item Allows for intermediate between barred and unbarred:
        \begin{itemize*}
            \item SA: normal spirals
            \item SB: barred spirals
            \item SAB: transition
        \end{itemize*}
    \item Adds extra distinction for \emph{ring} vs.\ \emph{s} shaped.
    \item Location along spiral dequence may not be the same as in
        ``Hubble classification'' (more based on B/D).
    \item Some \href{http://astronomy.as.virginia.edu}
        {\textcolor{blue}{pictorial examples}}
    \item \href{http://astronomy.nmsu.edu/holtz/a555/html/diagrams/a616/numtype.htm}
        {\textcolor{blue}{Numerical galaxy types (T)}}
\end{itemize*}
Note that
morphological classification often depends on multiple characteristics,
and therefore it can become somewhat subjective. Quantitative classification
schemes have been worked on, but are not widely used.
% ---------------------------------page 7-----------------------------------%

\subsection*{Wavelength dependence}
Morphology depends on wavelength
(e.g.\ \href{http://ned.ipac.caltech.edu/level5/Kuchinski/frames.html}
{\textcolor{blue}{O'Connell 9609101}}).
Be careful about morphological classification at higher redshift
(or at least a direct comparison with lower redshift galaxies)
because you may be looking at morphology at a different wavelength
(morphological \textbf{K Correction:} correction to magnitude (or flux) due
to redshift).

\subsection*{Classification correlations for spirals}
For the spiral sequence, there are correlations of
luminosity, surface brightness, rotational velocity and gas fraction
with Hubble type, but each category has a broad range of global observables,
and the catroies overlap significantly. Some representative data from
\href{http://adsabs.harvard.edu/cgi-bin/nph-bib_query?bibcode=1994ARA\%26A..32..115}
{\textcolor{blue}{Roberts and Haynes review (ARAA 1994)}}:
\begin{itemize*}
    \item Considering mean or median values, there is little trend in
        radius, $L_B$(?), and total mass for S0-Sc, but later types tend
        to be smaller, less massive, and less luminous
        (\href{http://astronomy.nmsu.edu/holtz/a555/html/diagrams/a616/rh2.htm}
        {\textcolor{blue}{RH figure 2}}).
        However, note that LSB(???) galaxies tend to be later types, and,
        since these are not included, there may be a bias here.
    \item Typical SB may be lower for latest types, even without inclusion
        of LSB galaxies; CSB(?????) much higher for large ellipticals than
        spirals. Mass surface density appears to decrease monotonically with
        morphological class in spirals
        (\href{http://astronomy.nmsu.edu/holtz/a555/html/diagrams/a616/rh3.htm}
        {\textcolor{blue}{RH figure 3}}).
    \item Cold gas absent in ellipticals, and neutral hydrogen (HI) content
        appears to increase monotonically with type for spirals. It is possible
        that molecular gas content decreases with type, but not certain.
        (\href{http://astronomy.nmsu.edu/holtz/a555/html/diagrams/a616/rh4.htm}
        {\textcolor{blue}{RH figure 4}}).
\end{itemize*}

\subsection*{Quantitative classification schemes}
All of the aforementioned classification schemes are subjective at some level.
Some more quantitative schemes have been proposed, both parametric and non-parametric.
\begin{itemize*}
    \item Bulge-to-disk ratio, using B/D decomposition (e.g.\
        \href{http://adsabs.harvard.edu/cgi-bin/nph-bib_query?bibcode=2003ApJ...582..689}
        {\textcolor{blue}{MacArthur, Courteau, \& Holtzman, ApJ 582, 689 (2003)}}
        ).
        Issues: covariance between parameters, 1D vs.\ 2D, and validity of models.
    \item Global profile fit, e.g.\ Sersic index
    \item Concentration, e.g.\ SDSS $r_{90}/r_{50}$, $r_{80}/r_{20}$,
        using circular apertures. Elliptical apertures can also be considered,
        e.g.\ based on \emph{second order moments}(???).
        Related to B/D. Sensitivity to seeing/distance.
\end{itemize*}

\subsection*{Non-symmetric galaxies}
Many of the schemes used for nearby galaxies fail for non-symmetric galaxies,
for which other indicators have been suggested:
\begin{itemize*}
    \item \textbf{Asymmetry:} e.g.\ \textcolor{blue}{Abraham et al, ApJS 107, 1, 1996}:
        rotate about center, self-subtract
        $$A = 0.5 \vert subtracted \vert /total$$
        Possible indicator of mergers.
    \item \textbf{Clumpiness:} e.g.\
        S(\textcolor{blue}{Conselice ApJS 147, 1 (2003)}):
        subtract smoothed version of image from original image,
        ratio of flux in subtracted image to flux in original images gives S.
        Possible indicator of star formation.
    \item \textbf{Gini coefficient:} e.g.\
        \textcolor{blue}{Abraham et al, ApJ 588, 218 (2003),
        Lotz et al., AJ 128, 163 (2004)}:
        applied to sorted list of pixel values.
        May be more stable at low resolution, S/N.
    \item \textbf{M20:} second order moment of brightest 20\% of
        galaxy light (\textcolor{blue}{Lotz et al., AJ 128, 163 (2004)}).
        May be more stable at low resolution, S/N.
\end{itemize*}

% ---------------------------------page 8-----------------------------------%
\section*{Basic observational properties of galaxies}
\subsection*{Surface Brightness}
Surface Brightness (SB) is the basic measured property for imaging of
a \emph{resolved} object. Units:
\begin{itemize*}
    \item erg cm$^{-2}$ s$^{-1}$ sterradian$^{-1}$
    \item mag arcsec$^{-2}$
\end{itemize*}
Magnitudes are often used, but if \emph{summing} multiple SBs, flux units
must be used. If you're adding and subtracting magnitudes, you're probably
doing something wrong. \Smiley

SB is independent of distance until geometry of the universe becomes
important:
$$SB \sim (1+z)^{-4}$$
In general, flux is a function of wavelength, so SB is measured in
different bandpasses, e.g.\ UBVRI or SDSS (ugriz).

The \emph{ratio} of the flux of the same object in different bandpasses
gives an estimate of the \emph{spectral slope} between bandpasses;
often expressed as a difference in magnitudes, or
\emph{color index}, such as U-R.

Relevance of SB distribution: when investigating how stars are distributed in
galaxies, be aware that \emph{mass} in stars doesn't necessarily track
\emph{light} in stars (nor does it trace mass in other components).

In principle, SB is measured directly from a 2D detector as an
arbitrary function of location.
Galaxies are relatively faint: the SB of the central region of a typical
galaxy is:
\begin{itemize*}
    \item Spiral: $V \sim 20 - 21$ mag arcsec$^{-2}$
        (not much brighter than typical dark sky)
    \item Elliptical: $V \sim 16 - 17$ mag arcsec$^{-2}$
\end{itemize*}
More than half the light from galaxies comes from regions with
$SB < SB_{\textrm{sky}}$ so $S/N$ is difficult to obtain.

Problems of seeing and \href{http://astronomy.nmsu.edu/holtz/a555/html/diagrams/a616/sky.htm}
{\textcolor{blue}{sky determination}} for SB distribution in center and
outer parts of galaxies. Sky problems are worse with small detectors and/or
large galaxies.

Since SB is the key observable, there can be strong \emph{selection effects}
against low SB objects. If all galaxies had the same SB profile (they don't)
then low SB galaxies are \emph{strongly} biased against in either
magnitude-limited or size-limited catalogs (apparent ``size'' of a galaxy will
depend on its SB).

Most galaxies are symmetric at a significant level;
isophotes are often well represented by ellipses.
Elliptical contours fit spirals and ellipticals for different reasons:
\begin{itemize*}
    \item Spiral: viewing angle/disk thickness
    \item Elliptical: intrinsic ellipticity
\end{itemize*}
(\href{http://astronomy.nmsu.edu/holtz/a555/resources/n1226.h.jpg}
{\textcolor{blue}{NGC1226}},
\href{http://astronomy.nmsu.edu/holtz/a555/resources/n1316.h.jpg}
{\textcolor{blue}{NGC1316}}).

Some basics of techniques for ellipse fitting: (insert links!)

In reality, galaxies have more complex features, e.g.\ asymmetries,
departures from elliptical isophtoes, bars, spiral arms, jets, etc.
(See below).

For a purely axisymmetric object, SB distribution reduces to a 1D SB profile.
SB profiles are often parameterized, with distributions of the form
of a \textbf{Sersic Law}:
$$ \Sigma(r) = \Sigma_e\exp\left(-b_n\left[\left(
\frac{r}{r_e}\right)^{1/n}-1\right]\right) $$
where $r_e$ is the half-light radius (the radius that encloses half the
total light if the model is extrapolated to infinity),
$\Sigma_e$ is the SB at the \emph{effective} radius
($\Sigma_0 \sim 2000\Sigma_e$),
$b_n \approx 2n - 0.324$ and is determined from the definition of $r_e$,
and $n$ depends on the type of galaxy
($n=4$ for ellipticals and $n=1$ for disks).
\begin{itemize*}
    \item Disks: well represented \emph{on average} by an exponential:
        $$ \Sigma(r) = \Sigma_s\exp\left(-\frac{r}{r_s}\right) $$
        $$ m(r) = m_s + kr $$
    \item Spherioids: historically characterized by
        \textbf{deVaucouleurs profile}, otherwise known as the
        `$r^{1/4}$' law:
        $$ \Sigma(r) = \Sigma_e\exp\left(-7.67\left[\left(
            \frac{r}{r_e}\right)^{1/4}-1\right] \right)$$
        $$ m = m_0 + kr^{1/4}$$
\end{itemize*}
There are other forms that can be used.

Sizes of galaxies:
$\ldots$

\subsection*{Integrated Brightness}

\subsection*{Spectral Energy Distribution}
\begin{itemize*}
    \item luminus components
    \item velocities via Doppler shift: $\nabla \lambda / \lambda = v/c $
\end{itemize*}


\subsection*{Distances}
need this to get properties like luminosities and linear sizes.
$$    m_M = 5\log{d}-5 $$
Various techniques:
\begin{itemize*}
    \item Variable stars
    \item Geometric techniques
    \item SB fluctuations: requires objects with a `standard' population
    \item Planetary nebulae luminosity function
    \item \textbf{Scaling relations between velocity and luminosity}
    \begin{itemize*}
      \item Spirals: \textbf{Tully-Fisher relation} between maximum
        rotational velocity and luminosity
      \item Ellipticals: fundamental plane relation between velocity
        dispersion, surface brightness, and physical size ($D_n -
        \sigma$) relation
    \end{itemize*}
    \item Redshift
    \begin{itemize*}
      \item At low $z, z \approx v/c$
      \item At $z \geq 0.1$
    \end{itemize*}
\end{itemize*}

\section*{Statistical properties}
\subsection*{Luminosity Function}
Number density as a function of luminosity [lum or mag]
\begin{itemize}
    \item $\Phi(L)$ - number density of galaxies per unit volumne with
      luminosity between $L$ and $L$ + $dL$
    \item $\Phi(M)$ - number of galaxies per unit volume with absolute
      mag between $M$ and $M$ + $dM$
    \item Integrate given number density of galaxies
\end{itemize}
First step toward understanding a fundamental question of galaxy formation: What
    sets the \emph{range} of galaxy luminosities and the relative
    \emph{numbers} of objects at different luminosities (though mass
    may be a more fundamental characteristic). Also an important
    cosmological probe for evolution of the galaxy population.

\begin{itemize}
      \item LFs usually well characterized by a \textbf{Schecter Function}
          \begin{equation}
              \phi(L) =
              \frac{\phi_*}{L_*}(\frac{L}{L_*})^{\alpha}exp(-\frac{L}{L_*})
          \end{equation}
          \begin{equation}
            \phi(M) \propto 10^{-0.4(\alpha+1)(M-M^*)}exp(-10^{0.4(M^*-M)}
          \end{equation}
      where $\phi(L)$ is the number of galxies with luminosity between
      $L$ and $L + dL$, $\alpha$ is the faint-end slope, and $\phi_*$
      and $L_*$ are parameters. Typical `local' parameters (e.g. SDSS
      at $z \sim 0.1$ are $\phi^*\sim0.015h^3Mpc^{-3}$,
      $M_B^*\sim-19.5$, $M_R^*\sim-20.5$, and $\alpha=-1$ to $-1.5$.
      \begin{itemize}
        \item Integrate Schecter function to get total luminosity
            density:
            \begin{equation}
              j = \phi_*L_*\Gamma(\alpha + 2)
            \end{equation}
        \item Significant discussian over detailed shape and
            normalization of LF\@. Observational issues (e.g.\ selection
            functions) and astrophysical ones (e.g.\ large scale structure,
            ``cosmic variance'').
      \end{itemize}
\end{itemize}
LF evolution $\rightarrow$ galaxy evolution
\begin{itemize}
      \item Possibilities:
      \item LF shows evolution (Faber et al. 2007), all galaxy samples
          show evolution as well, but red galaxies show number density
          evolution.
\end{itemize}

\subsection*{Colors of galaxies}
  \begin{itemize}
    \item Consider distribution of spectral energy distributions of
    galaxies, to first order represented by their color
    \item Bimodality (Strateva et al., AJ 122, 1861 (2001) )
    \begin{itemize}
      \item Red and Blue sequences (with `green valley' in between).
      \item Red sequence is tighter than blue sequence, so latter is
      sometimes called the `blue cloud'.
      \item Red sequence extends to higher luminosities, blue to lower
      luminosities, though significant overlap.
      \item Cause of bimodality: ellipticals and early-type spirals
      don't have much SF and appear red, and later-type spirals with
      current SF appear blue, although dust, bulges, and metallicity
      all play a role. The correspondance with morphology is supported
      by correlation with structural parameters, e.g.\ multidimensional
      correlations (Blanton and Moustakis ARAA 2009).
    \end{itemize}
    \item Easily observed and quantified
    \item Given likely differences in stellar populations,
    luminosity-stellar mass relation is probably different for the two
    different sequences. At a rough level, the color allows one to
    estimate the stellar mass from the luminosity. Some colors have
    more information about this than others, and some bandpasses for
    luminosity are more affected by differences in the stellar
    populations.
    \item Relation shifts when expressed in therms of stellar mass
    (Baldry et al. 2006), and there appears to be a transistion mass
    around 10$^{10}$ M$_{\odot}$ between the two sequences.
  \end{itemize}

\section*{Elliptical/Spheroidal galaxies}
\begin{itemize}
    \item Elliptical galaxies are not simple collections of
    \item SB profiles
        \begin{itemize}
            \item Ellipticals better fit by Sersic profiles than by
                deVaucouleurs (Caon et al. 1993).
                \begin{itemize}
                    \item correlation of Sersic indices with other parameters
                      suggests that there is something physical going on, but there are degeneracies.
                    \item Spiral bulges, low luminosity Ellipticals may be better
                      represented by exponentials.
                    \item Slope of SB profile (Sersic?) appears to be corelated with
                      Luminosity
                    \item SB vs L turns over at intermediate luminosity: two
                      families of shperoidal systems? $\rightarrow$ \textbf{Cuspy and
                      Core}
                    \item Ellipticals exist over wide range of
                      sizes/luminosities/SBs
                \end{itemize}
            \item other profiles
                \begin{itemize}
                    \item King model (truncated Gaussian for velocity distribution)
                    \item Hubble profile
                        $$ \Sigma(r) = \frac{\Sigma_s}{(1+\frac{r}{r_s})^2} $$
                        where $\Sigma_s$ = 0.25$\Sigma_o$
                \end{itemize}
            None are perfect matches to data over all scales (Burkert 1993 -
            comparison with deVaucouleurs law).
        \end{itemize}
    \item Inner regions of spheroidals deviate from Sersic profiles
        fit to outer regions. Inner regions are of particular interest
        because they may reflect dissipational collapse of low angular
        momentum material$\ldots$ initially seemed like galaxies had cores,
        though this is partly an effect of seeing. Higher spatial
        resolution (HST) shows that galaxies have both flat and steep
        (cuspy) inner profiles. Inner profiles seem to be roughly bimodal.
        \begin{itemize}
            \item Parametric fits to account for these become more complex,
                e.g.\ the `nuker law':
                $$ I(r) = I_b2^{(\beta-\gamma)/\alpha)}
                \Big(\frac{r_b}{r}\Big)^{\gamma}
                \Big[1+\Big(\frac{r_b}{r}\Big)^{\alpha}\Big]^{(\gamma-\beta)/\alpha} $$
                where $\gamma$ is the slope of the inner power law, $\beta$ is
                the slope of the outer power law, and $\alpha$ is the sharpness
                of the break between them.
            \item Profile type is correlated with luminosity. Luminous
                ellipticals ten to have `cuspy' cores with a break radius and
                shallower central density profile. Lower luminosity ellipticals
                have power laws all the way in.
            \item Galaixes with cusps$\ldots$
        \end{itemize}
    \item outer regions
    \item Intrinsic (3D) shapes of ellipticals: oblate (2 long axes) vs.\
      prolate (1 long axis) vs.\ triaxial (all axes different length).
      Determine true shape by looking at distribution of ellipticals.
      \begin{itemize}
          \item distribution function is different for fainter and
              brighter ellipticals
          \item For bright giant E's, distribution is inconsistent with
              either prolate or oblate itrinsic shapes: not enough circular
              galaxies (Tremblay and Merritt Fig 3).
          \item For fainter E's, distribution is \textit{consistent} with
              oblate, prolate, or triaxial.
          \item Triaxiality is also inferred for some giant E's from
              observation of isophotal twisting, which you cant get from
              oblate or prolate shape (de Zeeuw, Fig 1).
      \end{itemize}
    \item Non-axisymmetric features in galaxies
    \begin{itemize}
        \item Often described by amplitudes of Fourier moments of
          intensity distribution as a function of radius, e.g.
          $\alpha_1$, $\alpha_2$, $\alpha_4$
          $$ I(r,\theta)=\Sigma c_m\cos(m\theta)+\Sigma s_m\sin(m\theta) $$
          $$ \alpha_m = \frac{\sqrt{c^2_m + s_m^2}}{I_o} $$
          ``First even term above ellipses is $\alpha_4$ term:'' (WTF???)
          \begin{itemize}
             \item ``boxy" isophotes: $\alpha_4 < 0$;
                 bright, slow, central cores, strong radio and x-ray.
             \item ``disky" isophotes: $\alpha_4 > 0$;
                 faint, significant rotation-flattening, little
                 radio or x-ray emission, steep cusp.
          \end{itemize}
      \item Deviations
      \item Possibly
      \item also possible
      \item some ellipticals
    \end{itemize}

  \item Kinematics
  \begin{itemize}
      \item key kinematic quantity is \textbf{velocity dispersion};
          galaxies can be characterized by central velocity dispersion, but
          $\sigma$ does vary with radius.
      \item Some ellipticals have some rotation, mostly in lower
          luminosity systems.
       \item Relative importance of organized over random motion can be
        characterized by $v_{rot}/\sigma$.
        \item Shapes are expected to be influenced by rotation: for an
        oblate model with isotropic velocity distribution which is
        flattened by rotation: $v_{rot}/\sigma =
        \sqrt{\epsilon/(1-\epsilon)}$
        \item Giant ellipticals have less rotation than this, implying
        anisotropic velocity dispersions to account for their shape
        (required for triaxial systems).
        \item Low/medium luminosity (high SB) ellipticals may be isotropic
        with flattening caused by rotation.
        \item Low SB ellipticals appear to have anisotropic velocity
        dispersion, i.e.\ more eccentric than expected from rotation;
        ``measured in LG ellipticals 185 (factor of three low in
        $v_{rot}/\sigma$, 147, factor of 10 low)." seriously, wtf.
        \item Significant fraction of ellipticals may have dynamical
        subcomponents, e.g.\ velocity and dispersion for some interesting
        cases.
  \end{itemize}

  \item Relations between different parameters and different families
      of ellipticals
      \begin{itemize}
        \item Photometric parameters: SB with size, SB with luminosity (``Kormendy
        relations"), SB \textit{shape} (e.g. Sersic index) with luminosity.
        \item kinematics vs.\ luminosity
        \begin{itemize}
          \item More luminous galaxies have higher velocity dispersions
          (\textbf{Faber-Jackson relation}: roughly $L \sim \sigma^4$, but
          lots of scatter).
          \item More luminous galaxies have less rotation.
        \end{itemize}
        \item Faber-Jackson relations and the Kormendy relation (between
        SB and luminosity/size) are manifestations of the
        \textbf{Fundamental Plane of elliptical galaxies}:
        \begin{itemize}
          \item Correlation between residuals in the Faber-Jackson
          relation and SB.
          \item Galaxies do not populate the entire 3D space of $I$, $r$,
          and $\sigma$, but instead populate only a \textit{plane} in this
          space.
          \item relation between the \textbf{three fundamental global
          observables}: SB (or luminosity), size, and velocity dispersion
          (Dressler et al. 1987; Djorgovski and Davis 1987;
          Bender, Burstein and Faber 199x)
          \begin{itemize}
            \item observed relation given by Virgo ellipticals:
            \begin{equation}
               r_c \propto (\sigma_o^2)^{0.7}I_c^{-0.85}
            \end{equation}
            \item Origin of relation? Three assumptions would have to be
            true: Ellipticals are in virial equilibrium, M/L varies
            systematically with luminosity, and ellipticals form a
            ``homologous" family, all, e.g.\ with deVaucouleurs profiles.
            In this case, one expects:
            \begin{align*}
                L &= c_1I_cr_c^2\\
                M &= c_2\frac{\sigma^2_or_e}{G}
            \end{align*}
            where the first is a definition and the second is related to
            the \textit{virial theorem}, which applies to the mean
        potention energy and mean kinetic energy per unit mass,
        averaged over the entire system. The constants are related to
        the shapes and other details (not necessarily constant among
        the variety of ellipticals).
      \item Combining these, we get
      \begin{equation}
          r_e =
          \frac{c_2}{c_1}\Big(\frac{M}{L}\Big)^{-1}\sigma_o^2I_c^{-1}
      \end{equation}
      If one has $M/L \propto L^{~0.2}$, one then recovers the
      observed fundamental plane; could arise from stellar populations
      or from variations in baryon to total mass. Alterm=natively, one
      could have constant $M/L$ with a structure which varies relative
      to one or more of the fundamental variables (which we know it
      does, e.g., systematic variations of Sersic n with luminosity,
      but not clear if this is entire ``explanation").
      \item Need to understand origin of assumptions; why should
      parameters, e.g.\ mass-to-light (mass includes dark matter)
      vary smoothly with luminosity? If ellipticals have dark matter
      halos, they don't require luminous inner parts to be in virial
      equilibrium.
      \item There is relatively little scatter around the fundamental
      plane, implying that the assumptions are reasonably valid over a
      large range of elliptical properties, which implies some
      significant regularities in the galaxy formation process.
      \end{itemize}
    \item Galaxies do not fully populate the entire plane defined by
    our relation.
    \begin{itemize}
      \item Consequently,when the plane is projected onto the other
      two axes, one can see a correlation.
      \item In the luminosity(size)-$\sigma$ plane, one finds that $L
      \propto \sigma^4$, which is known as the \textbf{Faber-Jackson
      relation}. However, since the locus of ellipticals isn't
      perfectly linear and the plane defied by the ellipticals isn't
      perpendicular to this dimension, the scatter around F-J is
      larger than the scatter around the fundamental plane. One can
      define a new radius which incorporates SB, such that the new
      radius vs $\sigma$ views the fundamental plane edge-on; such a
      size measurement is called $D_n$, the isophotal
      \textit{diameter} of the B=20.75 isophote. This $D_N - \sigma$
      relation provides a very useful distance estimator - if the
      fundamental plane really is fundamental.
      \item In the SB-size plane, one sees a relation in which smaller
      galaxies have higher SB for normal ellipticals; the diffuse
      ellipticals have different behavior where smaller galaxies have
      low SB\@. These are sometimes known as the \textbf{Kormendy
      relations} and are one of the main bases for separating these
      two types of objects.
      \item SB-$\sigma$ plane - presumably related to underlying
      physical parameters: density and virial temperature. In this
      plane, one can only form galaxies where cooling is effective,
      i.e.\ at larger densities and hotter temperatures. This restricts
      the area in the space in which we can find galaxies.
      \item Additional features of the galaxy formation process may
      introduce additional restrictions into allowed locations of
      galaxies on the fundamental plane. Most luminous ellipticals are
      located along one line (with some scatter) in the fundamental
      plane, and most diffuse ellipticals are located along another.
    \end{itemize}
    \item Isophotal deviations vs.\ luminosity and kinematics
    \item Several types of ellipticals??
    \end{itemize}
  \end{itemize}
  \item Spectral energy distributions
  \begin{itemize}
    \item \textbf{Typical Features}: stellar absorption lines (4000
    $A$ break, Mg, Fe, etc.)
    \item Generally red in color.
    \item Color-luminosity relation:
    \item Mg line strenth-luminosity relation:
    \item Some ellipticals have signatures of a younger population:
    \item Stellar populations within ellipticals:
  \end{itemize}
  \item Interstellar matter in ellipticals
  \begin{itemize}
    \item Ellipticals have significant interstellar gas, seen in X-ray
    emission (hot!), from stellar evolution or the environment.
    \item Also evidence of colder gas and dust; $~50\%$ of ellipticals
    show evidence for dust their cores (Lauer et al. 2005)
    typically small components by mass.
  \end{itemize}
\end{itemize}



\section{Spiral/Disk galaxies}
    \begin{itemize}
      \item SB profiles
      \item Size-luminosity/SB-luminosity relations
      \item Vertical distributions
      \item Non-axisymmetric features in disk galaxies: bars and
      spiral arms
      \item Kinematics: spirals are kinematically cold, meaning the
      random motion of stars is small compared to the organized motion
      (rotation), though there is some dispersion of velocities.
      \item \textbf{Tully-Fisher relation}: kinematics(maximum
      rotational velocity)-luminosity relation.
      \item Spectra
      \item Gas and Dust
    \end{itemize}

%\item To sum up: Galaxy parameter (scaling) relations
\section{To sum up: Galaxy parameter (scaling) relations}
    \begin{itemize}
      \item Structural: luminosity-SB
      \begin{itemize}
        \item ellipticals: isophotal shape-luminosity relation (boxy
        vs. disky).
      \end{itemize}
      \item Kinematic: Faber-Jackson, Tully-Fisher
      \begin{itemize}
        \item ellipticals: luminosity-rotation relation
      \end{itemize}
      \item structural/kinematic: Fundamental Plane
      \item Stellar populations
      \item Gas
      \item Black holes
    \end{itemize}

%\item Environments of galaxies: clusters and cluster galaxies
\section{Environments of galaxies: clusters and cluster galaxies}
    \begin{itemize}
      \item Galaxies are not homogeneously distributed in space...
      Correlation function
      \item Groups and Clusters
      \item Abell cluster
      \item Galaxies in galaxy clusters (Boselli and Gaazzi, PASP 118,
      517, 2006)
      \item Distributed hot gas in clusters, with mass comparable to
      that found in galaxies (one individual galaxy?)
      \begin{itemize}
        \item X-ray observations
        \item temperatures are 10$^7$ - 10$^8$ K
        \item intracluster gas is enriched in heavy elements
        \item X-ray observations useful for probing cluster masses,
        under assumption of hydrostatic equilibrium.
        \begin{itemize}
          \item Need measurement of density and temperature profiles
          of X-ray gas
          \item Estimates can be made from X-ray luminosity and/or
          temperature (better): the $M - kT$ relation
          \item typical masses are 10$^{14}$ - 10$^{15} M_{\odot}$; a
          typical cluster mass function (Wen et al. 2010).
        \end{itemize}
      \end{itemize}
      \item Shapes of clusters, inhomogeneities: not all clusters are
      in equilibrium
    \end{itemize}

\section{Some galaxies to be familiar with}
    \begin{itemize}
      \item stuff
    \end{itemize}

%-------------------------------------------------------------------%
\section{The building blocks of galaxies}

\subsection*{Stars and Stellar populations}

\subsubsection*{Relating observables to intrinsic population characteristics}
\begin{itemize*} % AA
    \item \textbf{Population characteristics:} distribution of masses, compositions
        and ages (star formation history).
    \item \textbf{Observe spectral energy distributions (or colors) of stars:}
        individual stars in very nearby galaxies, interated starlight
        for most galaxies.
\end{itemize*}% end AA

\subsubsection*{Stellar evolution}
\begin{itemize*} % BB
    \item Internal structure of stars: determined by mass, chemical
        composition, and age (Russell-Vogt theorem). Exclues
        non-spherical symmetric effects, e.g.\ rotation, magnetic
        fields, binarity, etc.
    \item Luminosity (radius) and effective temperature derived from
        equations of stellar structure: mass conservation, hydrostatic
        equilibrium, energy equation, energy transport, along
        with auxiliary relations: equation of state, opacity, nuclear
        reaction rates.
    \item Main stages of stellar evolution:
        \begin{itemize*} % CC
            \item Hydrogen core burning: main sequence (MS)
            \item Hydrogen shell burning: giant branch (for lower mass
                stars)
            \item Helium core burning: horizontal branch, red clump,
                blue core helium burners. Note key transition around 2
                solar masses (depends on metallicity): shift to helium
                flash at lower masses.
            \item Helium shell burning: Asymtotic Giant Branch (AGB)
            \item Other nuclear burning for high mass stars
            \item White dwarf or supernova
        \end{itemize*} % end CC
    \item Model tracks (evolution as a function of time for a given
      mass); for spherical symmetry, calculations are 1D.
    \item \textbf{Isochrones}: cross-section of poroperties at a
      fixed time across a range of masses. Some well-known groups that
      calculate evolutionary tracks/isochrones:
      \begin{itemize*}
          \item Padova
          \item BASTI (Teramo)
          \item Dartmouth
          \item Yale-Yonsei
          \item Victoria-Regina
          \item Geneva
      \end{itemize*}
    \item Uncertainties that lead to some differences between different
        calculated isochrones: convective overshoot, diffusion, convection,
        helium abundance, mass loss, etc. Generally, Uncertainties are larger
        for later stages of evolution. Additionally, there may be missing
        physics, e.g.\ rotation and magnetic fields, that would require a full
        3D treatment.
    \item Given effective temperature,
        surface gravity (from mass and radius), and composition,
        stellar atmospheres give observables: spectral energy
        distribution/colors. Some model atmospheres (need links!):
        \begin{itemize*}
            \item \textcolor{blue}{Kurucz}
            \item \textcolor{blue}{MARCS}
        \end{itemize*}
    \item Theoretical Color-Magnitude Diagrams (CMDs) $\rightarrow$
      observed; need distance and reddening/extinction
    \item Age effects (model isochrones) from Yi et al 2001.
    \item Metallicity effects: internal (opacity) and atmosphere
      (line blanketing) effect combine in the same direction to make
      more metal-rich populations redder: \textcolor{blue}{CMDs}
      \begin{itemize*}
          \item Metallicity terminology: often given as mass fractions of
              hydrogen (X), helium (Y), and heavier elements (Z).
          \item solar abundance: X=0.7, Y=0.28, Z=0.019 (roughly)
          \item also given by
              $$ \left[\frac{\textrm{Fe}}{\textrm{H}}\right] =
              \log\left[
              \frac{\left( {\textrm{Fe}}/{\textrm{H}} \right)}
              {\left({\textrm{Fe}}/{\textrm{H}}\right)_{\odot}}
              \right]$$
          \item Be aware that this is an oversimplification, as Z contains
              lots of different elements. More later.
      \end{itemize*}
    \item Main sequence (MS)
        \begin{itemize*}
            \item Very rough scaling relation between luminosity and mass:
                $ L \propto M^{3.5} $
            \item Main sequence shifts with metallicity: redder for
                higher metallicity
            \item Location also depends on helium abundance
        \end{itemize*}
    \item Red Giant Branch (RGB)
        \begin{itemize*}
            \item note that because of more rapid evolution after MS,
                RGB stars all have roughly the same mass.
            \item temperature of RGB depends on age/mass: younger and more
                massive stars are hotter. Temperature also depends on
                metallicity: latter is dominant effect in older populations
                (greater than 5 Gyr).
            \item At lower masses/larger ages, tip of RGB is close to
                constant bolometric luminosity regradless of age or
                metallicity. In observed plane, leads to roughly fixed
                tip luminosity \emph{if} observing at long wavelengths
                (e.g.\ I band): basis for Tip of Red Giant Brance (TRGB)
                distance indicator.
            \item RGB bump (\textcolor{blue}{RGBB} from
                \textcolor{blue}{Bono et al 2001}),
                location of which depends on mass/age and metallicity;
                arises when H burning shell crosses chemical discontinuity.
        \end{itemize*}
    \item Horizontal Branch (HB), aka.\ Red Clump (RC) or more generally
        He core burning sequence
        \begin{itemize*}
            \item High mass stars form blue He core-burning branch
                (also note red plume)
            \item Intermediate mass stars form red clump.
            \item Low mass stars form horizontal branch. Variable mass loss
                on RGB and at He flash gives a range of masses.
            \item Horizontal branch morphology depends on metallicity: more
                metal-poor populations have bluer HB.
            \item However, there is something else that also affects HB
                morphology, leading to the so-called second-parameter problem
                (e.g.\ \textcolor{blue}{M3/M13}, from \textcolor{blue}{Rey et al
                2001}). Possibilities: Age? He abundance? Heavy element
                abundances? Density? Rotation?
            \item RR Lyrae stars: in instability strip (caused by doubly
                ionized He, also includes Cepheids, delta Scuti stars, etc.)
                RR Lyrae stars are indicators of old metal-poor population.
                Periods of 0.5 days, but multiple groups (Oosterhoff classes)
                depending on stellar parameters$\ldots$
        \end{itemize*}
    \item Asymtotic Giant Branch (AGB)
        \begin{itemize*}
            \item For intermediate masses/ages, AGB is significantly more
                luminous than the TRGB, hence of potential critical importance
                to studies of integrated light.
            \item For lower masses (older populations), AGB tip comparable
                to TRGB and AGB asymtotically approaches RGB (hence its name).
        \end{itemize*}
    \item Potential importance of binaries/interactions
        \begin{itemize*}
            \item unresolved (but otherwise non-interacting) binaries:
                broaden sequences, depending on mass ratios: equal masses
                give the appearence of a star 0.75 mag brighter.
            \item interacting binary stars
            \item blue stragglers: possible stellar merger/interaction products?
                \textcolor{blue}{M3 example} from Sandage 1953.
            \item supernovae type SNIa: arise from binaries, produce different
                heavy element abundances than core collapse SNe. (e?)
        \end{itemize*}
    \item End stages of stellar evolution:
        WDs, neutron stars, black holes, and supernovae
        \begin{itemize*}
            \item \textcolor{blue}{projenitor-final mass relation}
                (from Binney and Merrifield)
            \item \textcolor{blue}{White dwarf cooling sequence}
                (from \textcolor{blue}{Hansen et al 2007}: potential for
                age dating).
            \item Supernovae: generate significant fraction of heavy elements
                (but not all). Significant energy input, thermal and mechanical.
        \end{itemize*}
    \item Galactic globular clusters: cornerstone of understanding
      stellar evolution historically, as apparent examples of a
      ``simple stellar population (SSP)'', with all stars of the same
      age and abundance.
      \begin{itemize*}
          \item However, it's not recognized that not all GCs are so simple.
          \item Some CMDs show clear evidence of multiple components, e.g.\
              \textcolor{blue}{NGC2808}.
          \item Long history of evidence of abundance variations between
              different stars, typified by the Na-O anticorrelation;
              demonstrated through observations of main sequence stars that
              this is not a mixing effect.
          \item Two phenomena have recently been coupled, but not yet a clear
              understanding of how the multiple populations arise.
      \end{itemize*}
  \end{itemize*}

  \subsubsection*{More information in a CMD than is represented by isochrones}
  one can also consider relative numbers of stars at
      different locations: CMDs that incorporate number of stars are
      reffered to as Hess diagrams (e.g.\ \textcolor{blue}{Fornax}
      from Battaglia et al., 2006).
      \begin{itemize*}
          \item For a simple stellar population (SSP, a population with
              a single age and metllicity), the relative number of stars
              at each stage is determined by the initial mass function (IMF)
              and the age.
          \item IMF determinations and parameterizations:
              \begin{itemize*}
                  \item ``classical'' Salpeter IMF (power law with
                      $ dN/dM \propto M^{-2.35}$) and \textcolor{blue}{others}
                      (from \textcolor{blue}{Pagel}). Note power law form:
                      $ dN/dM \propto M^{\alpha}  $,
                      or alternatively,
                      $ dN/d\log{M} \propto M^{\Gamma} \propto M^{\alpha + 1} $
                  \item Widely used determination of local IMF is by
                      \textcolor{blue}{Droupa Tout, and Gilmore 1993}, who find
                      $$ dN/dM \propto M^{-2.7}\ \textrm{for}\ M > 1M_{\odot} $$
                      $$ dN/dM \propto M^{-2.2}\ \textrm{for}\ 0.5 < M < 1M_{\odot} $$
                      $$ dN/dM \propto M^{-1.3}\ \textrm{for}\ M < 0.5M_{\odot} $$
                  \item Chabrier IMF: log-normal form,
                      $ dN/dM \propto \exp(\log{M}-\log{M_o})^2  $
                  \item Empirically, no strong evidence for variations of the
                    IMF as functions of, e.g., metallicity.
                  \item No well-established theory for predicting the IMF
              \end{itemize*}
      \end{itemize*}

\subsubsection*{Star Formation Histories (SFHs) from resolved stellar populations}
      \begin{itemize*}
          \item In general, galaxies are not SSPs, i.e.\ their CMDs don't look
              like those of a cluster. Can fit Hess diagrams of resolved stellar
              populations with combinations of SSPs to derive constraints on SFH,
              e.g.\ \textcolor{blue}{simulated galaxies} (from \textcolor{blue}
              {Tolstoy et al}).
          \item Generally speaking, we want the star formation history:
              $SFH(t,Z,M)$ gives the number of stars (or stellar mass) at all
              combinations of age, metallicity, and mass; because of lack of
              observed IMF variation, usually IMF is separated out:
              $$ SFH(t,Z,M) = \xi(M)\psi(t,Z) $$
              where $\xi$ is the IMF.
          \item MW neighbors resolved down to oldest MS turnoff ($M \sim 5$),
              because typical distances give distance modulii $\lesssim$ 20.
          \item M31 and neighbors somewhat shallower ($m-M\sim$ 24.5)
              without large investment of telescope time.
          \item Results for MW:
              \begin{itemize*}
                  \item \textcolor{blue}{solar neighborhood Hipparcos sample}:
                      roughly constant \textcolor{blue}{SFH}, but be aware
                      of dynamical effects and limited volume bias nearby
                      sample to younger populations. When corrected for, star
                      formation history in solar neighborhood likely to have
                      significantly declined (see, e.g.\ \textcolor{blue}
                      {Aumer \& Binney 2009}).
                  \item Age of oldest stars can be studied by WD sequence,
                      gives $t_{\textrm{oldest}} > 8$ Gyr. Also note presence
                      of disk RR Lyrae stars ($t > 10$ Gyr??)
                  \item Might consider studying age distribution of clusters,
                      as they are simpler populations, but note problems with
                      disruption of clusters that would lead to a bias to
                      younger clusters.
                  \item Bulge can also be studied, but note foreground population
                      and extinction issues. Predominantly old population,
                      e.g.\ \textcolor{blue}{bulge CMD}, and \textcolor{blue}
                      {luminosity function} (from \textcolor{blue}{Ortolani
                      et al 1995}).
                  \item Halo (as defined kinematically/spatially) also
                      prdominantly old, both from field population and from
                      globular cluster population.
                  \item Historically, distinction between disk stars/open
                      clusters (population I) and halo stars/globular clusters
                      (population II), with pop I being younger and more metal
                      rich. Note, however, the pop II association with low
                      metallicity is now recognized not to be fundamental;
                      inner halo/bulge significantly more metal rich.
                  \item For metallicities, note that there is a significant
                      puzzle in that there is \emph{not} a strong age-metallicity
                      relation in the solar neighborhood (e.g.\ Orion has
                      roughly solar metallicity, but is nearly 5 billion years
                      younger): radial migration, inhomogeneous ISM, inflow
                      $\ldots$?
              \end{itemize*}
          \item Results for Local Group (LG) dwarf galaxies
              \begin{itemize*}
                  \item Carina dSph: striking evidence of episodic star
                      formation, but this is NOT characteristic.
                  \item Others (from Tolstoy et al); note range of SFHs.
                  \item LG dIrrs compilation (from Dolphin et al 2005)
                  \item LG dSphs (from Dolphin et al 2005); possible similarity
                      to dIrrs apart from lack of recent SF?
                  \item Many dwarfs show population gradients, e.g.\ from
                      RGB/HB morphology (e.g., Harbeck et al 2001), likely
                      metallicity gradients: some show evidence of age gradients,
                      with younger population in center.
              \end{itemize*}
          \item Results for M33 (from Holtzman et al 2011); note significant age
              gradient, and significant component of younger stars.
          \item M31 halo (from Brown et al 2006); significantly different from
              MW bulge, with more of an intermediate-old population?
          \item Can do more distant galaxies if information from advanced stages
              (RC, HB) is reliable; (e.g.\ local volume SFH (from Williams et al,
              in prep), dwarf SFH (from Weisz et al 2011)).
          \item Even farther if info from AGB is reliable.
          \item General conclusions:
              \begin{itemize*}
                  \item galaxies are not SSPs.
                  \item galaxies have gradients in their stellar populations
                  \item significant population of old stars in nearby galaxies?
                  \item more massive galaxies formed stars earlier??
              \end{itemize*}
      \end{itemize*}

\subsubsection*{Integrated light from \emph{un}resolved stellar populations}
      \begin{itemize*}
        \item What stars contribute the most light?
            \begin{itemize*}
                \item Along the MS, use combination of M-L relation and IMF:
                    $$ L \propto (M^{-2.35}M^{3.5}) \propto M^{1.15} $$
                    Massive stars dominate light.
                \item Compare MSTO with evolved population: RGB has almost
                    same mass as MSTO, but significantly more luminous: relative
                    contributions depend on relative number of starsalong the
                    giant branch compared with MSTO stars, but one finds that
                    the luminous evolved populations dominate (Renzini fig 1.5).
                \item Since post-MS evolution is fast compared to MS evolution
                    for all masses, it is true that at any given time when the
                    most luminous stars are the most evolved stars, the luminosity
                    is given predominantly by stars of (nearly) a single mass
                    (true for all except youngest ages, see Renzini 1.1) and
                    luminosity vs.\ lifetime plots.
            \end{itemize*}
        \item Integrated brightness: do SSPs get brighter or fainter as they age?
            \begin{itemize*}
                \item At younger ages, have MS ``peel-off'' effect, but also
                    supergiants! (WTF$\ldots$)
                % ------------------page 6-----------------------------------%
                \item At older ages, have competing effects of IMF and rate of
                    evolution.
                    \begin{itemize*}
                        \item To determine the luminosity of a stellar
                            population, consider the number of stars evolving
                            off the MS, or equivalently, dying, which is given
                            by the \emph{evolutionary flux}:
                            $$ b(t) = \psi(M_{\textrm{TO}})|\dot{M}|
                            \textrm{\ [stars\ yr}^{-1}]$$
                            where TO stands for turnoff, $\psi$ is the IMF,
                            and $|\dot{M}|$ is the time derivative of the
                            turnoff mass, e.g.\ Renzini 1.1. The key point is
                            that lower mass stars evolve more slowly.
                        \item The IMF is important, along with age, in
                            determining the luminosity evolution of a galaxy.
                        \item Once one reaches an age where most massive stars
                            have died, all stars of lower mass reach comparable
                            luminosity (tip of RGB), hence luminosity of
                            population depends on number of red giants.
                        \item For a typical IMF number of stars increases
                            slower toward lower mass than rate of generation
                            of turnoff stars decreases.
                        \item Luminosity of the turnoff also has some smaller
                            effect.
                    \end{itemize*}
                \item Do integrated colors depend on the IMF?
                    \begin{itemize*}
                        \item Need to know the relative contributions of each
                            stage of evolution, i.e., from stars of different
                            masses.
                        \item Number of stars in each post-MS stage is
                            determined predominatnly by the time spend in each
                            stage:
                            $$ N_j = b(t)t_j $$
                            because all post-MS stars have nearly the same mass.
                            So although the total flux is sensitive to the IMF,
                            the relative contributions of each stage are not,
                            at least for older populations (Renzini fig 1.5).
                        \item For all except the youngest population, the later
                            stages of evolution provide a majority of the light.
                            Evolved stages are nearly all the same mass.
                        \item Consequently, the integrated spectrum and the
                            relative contributions of various evolutionary
                            stages are nearly independent of the IMF\@. They
                            \emph{do} depend on the age and metallicity.
                    \end{itemize*}
                \item Dependence of color and luminosity on time (from
                    Bruzual and Charlot 199?) for a SSP (single burst)
                    normalized to one solar mass.
                    \begin{itemize*}
                        \item luminosity variation is a combination of IMF plus
                            rate of evolution for lower masses.
                        \item color evolution comes from changing mix of stellar
                            population, but this is just for a single (solar)
                            metallicity.
                        \item note RSG phase, then subsequent dimming and
                            reddening.
                        \item note this is bandpass dependent, with less dimming
                            (at older ages where the RGB is dominant) at longer
                            wavelengths.
                    \end{itemize*}
                \item Implication: if range of ages is present, integrated light
                    is (significantly) weighted toward younger populations.
                % ------------------page 7-----------------------------------%
            \end{itemize*}
        \item What stars contribute the most \emph{mass}?
            \begin{itemize*}
                \item For all measured
                \item relative stellar M/L ratios
            \end{itemize*}
        \item Can you estimate stellar mass from integrated light?
            \begin{itemize*}
                \item Consider the
                \item For individual stars
                \item Absolute value of
                \item stellar M/L ratio depends on
                \item So, it is possible
                    \begin{itemize*}
                        \item Variety of different
                        \item Some uncertainties
                        \item Larger
                        \item if IMF were to be variable
                    \end{itemize*}
            \end{itemize*}
        \item Star formation histories from integrated colors
            \begin{itemize*}
                \item Issue: intgrated colors are affected by combination of age
                    distribution,
% ----------------------------------page 8-----------------------------------%
                    metallicty distribution, and reddening distribution.
                \item color variations with age:
                \item situation somewhat improved
                \item Even if age-metallicity
            \end{itemize*}
        \item Given age-metallicty degeneracy from colors,
        \item Spectral evolution (BC fig 4) for a variety of SFRs
        \item Dependence of contributor on wavelength
        \item Potential problems with synthetic integrated spectra:
        \item What about spectral features?
            \begin{itemize*}
                \item For populations
                \item For older populations
% ----------------------------------page 9-----------------------------------%
                \item typical optical spectra
            \end{itemize*}
        \item Even with
        \item Main area of application
% ---------------------------------page 10-----------------------------------%
      \end{itemize*}

\subsubsection*{Chemical evolution}
\begin{itemize*} % A
    \item Understanding
    \item Start with some initial conditions,
        then add star formation history: (IMF + SFR). This results
        in chemical enrichment for stars as a function of mass and
        metallicity. Also consider input and output of gas from any
        particular region, with either primordial or modified
        composition.
        \begin{itemize*} % AA
            \item ICs: No heavy elements and pure gas (no stars).
            \item SFH: birthrate function $\psi(m,t)$ gives
                number of stars formed with mass $m$ per unit
                volume, usually split into separable function:
                $$ B(m,t) = \xi(M)\psi(t) $$
                where $\xi(M)$ is the IMF and $\psi(t)$ is the SFR\@.
                This implicityly assumes an IMF which is constant
                in time, which may or may not be true, but so far,
                there is no strong observational evidence against
                it (although note pop III issues).
            \item For any particular star, use stellar evolution
                to compute the amount and composition of mass
                returned to the ISM vs.\ amount locked up in stellar
                remnants (Pagel fig 2, from Maeder 1992). Note that
                the \emph{rate} of return depends on stellar mass,
                especially for the case of SnIa, which also depends
                on the binary fraction.
% ---------------------------------page 11-----------------------------------%
            \item nucleosynthesis: main element groups and their sources:
            \begin{itemize*} % AAA
                \item light elements: Big Bang Nucleosynthesis (BBN)
                    plus subsequent destruction (?)
                \item $\alpha$ elements (even Z from O up$\ldots$???):
                        massive stars (core collapse SN)
                \item Fe-peak elements: type Ia SN and core collapse SN
                \item $s$ and $r$ process (neutron capture, plux beta
                        decay) elements: core collapse SN, AGB stars$\ldots$
                        (need neutron capture diagram here)
            \end{itemize*} % end AAA
            \item Note that infall might arise from primordial clouds, or from
                processed material, e.g.\ mass-loss from halo stars. Outflow
                might come from SN winds, and in this case, it's possible that
                the composition of outflow material might be more enriched than
                the typical composition at any given time. So things can get
                complicated.
        \end{itemize*} % end AA
    \item Basic equations of chemical evolution
        \begin{itemize*}
            \item Allowing for
            \item Consider a total (galaxy) mass M, split into a gas mass
                $g$ and stellar mass $s$. For a simple model, consider the
                gas at any time to be well mixed. At any given time, we wish
                to know the fraction abundance ($Z_i$) of element $i$.
                Consider an inflow rate into the system, $F$ and outflow
                (ejection) rate, $E$. Then we have:
                $$ M = g + s $$
                $$ \frac{dM}{dt} = F - E $$
                $$ \frac{dg}{dt} = F - E + e - \psi't $$
                where $\psi'$ is the SFR in units of mass per time,
                $$ \psi' = \psi\int\xi\textrm{d}m $$
% ---------------------------------page 12-----------------------------------%
                (so $\xi$ is normalized to a total of one), and $e$ is the
                ejection rate of mass from stars,
                $$ e = \int_{m_t}^{m_U} \left(m-m_{\textrm{rem}}\right)
                \psi\left(t-\tau(m)\right)
                \xi\left(m\right)\textrm{d}m $$
                which is a sum over all stellar masses of the product of the
                SFR at the time of formation of each mass with the mass returned
                to the ISM, weighted by the IMF\@. The lower limit of integration
                is at the stellar mass which is dying at time $t$; lower mass
                stars don't contribute because they haven't ejected any mass yet.
                We also have
                $$ \frac{ds}{dt} = \psi' - e$$
                wince the mass in stars increases by te number of stars formed,
                but decreases by the amount of mass lost back to the ISM from
                the previous generation of stars. For the elements, we have
                $$ \frac{d(gZ_i)}{dt} = e_Z(i) - Z_i\psi + Z_FF - Z_EE  $$
                ($Z_i$ is the mass fraction of element $i$; hereafter we'll drop
                the subscript for simplicity). The mass in each element
                \emph{increases} by:
                \begin{itemize*}
                    \item amount of mass released by previous generations
                    \item amount of mass added by inflow
                \end{itemize*}
                but \emph{decreases} by:
                \begin{itemize*}
                    \item amount of mass locked up in new stars
                    \item amount of mass lost to outflow
                \end{itemize*}
                The term $e_Z$ is given by:
                $$
                    \psi(t-\tau(m))\xi(m)\textrm{d}m
                $$
                where $q_Z$ represents the fractional mass of element $Z$
                synthesized and ejected during stellar evolution (so left term
                in brackets gives material which returns unprocessed and right
                term gives newly synthesized contribution).
                In a simple model, the synthesized masses are independent of
                the metallicity of the population (although we know this is not
                true for some elements, more on this later).
            \item To simplify, consider an approximation to these formulae called
                the \emph{instantaneous recycling approximation} which assumes
                that all elements are returned instantaneously - good for products
                of massive stars, but less good for products from lower mass stars,
                e.g.\ iron. Then we have
% ---------------------------------page 13-----------------------------------%
                $$ e = \psi\int(m-m_{\textrm{rem}})\xi(m)\textrm{d}m
                \equiv (1-\alpha)\psi $$
                where $\alpha$ is the \emph{lock-up fraction}: the fraction
                of mass ``locked up'' in stars and remnants.
                $$ \ldots $$
% ---------------------------------page 14-----------------------------------%
                $$ \ldots $$
                and finally,
                $$ g\frac{dZ}{ds} = \frac{d(gZ)}{ds} - Z\frac{dg}{ds} =
                p + (Z_F - Z)\frac{F}{\alpha\psi} $$
                \textcolor{red}{This is a basic equation of chemical synthesis}.
            \item The \emph{simplest} model for chemical evolution assumes no
                inflow or outflow, a homogeneous system without any spatial
                differentiation of metallicity, zero initial metallicity, and
                yields which are independent of composition. This is known as the
                Simple, one-zone model.

                In the simple model, we have
                $$ \frac{dg}{dt} = -\frac{ds}{dt} $$
                $$ g\frac{dZ}{ds} = -g\frac{dZ}{dg} = p $$
% ---------------------------------page 15-----------------------------------%
                Solving for $Z(g)$, we get
                $$ Z
                $$
        \end{itemize*}
\end{itemize*}
% ---------------------------------page 16-----------------------------------%
\subsubsection*{Abundances as function of time}
% ---------------------------------page 17-----------------------------------%
\subsubsection*{Relative abundances of different elements}
% ---------------------------------page 18-----------------------------------%



% ---------------------------------page 19-----------------------------------%
\subsection*{Gas}
\subsection*{Dust}
\subsection*{Central black holes}
\subsection*{Galaxy spectral energy distributions}
\subsection*{Dark matter and galaxy masses}

\end{document}
























