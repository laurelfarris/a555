\documentclass{article}
%\usepackage[left=1in, right=1in, top=1in, bottom=1in]{geometry}
\renewcommand\familydefault{\sfdefault}
\usepackage[margin=1.5in]{geometry}
\setlength{\parindent}{0em}

\usepackage{hyperref}
\usepackage{amsmath}
\usepackage{amssymb}
\usepackage{marvosym}
\usepackage{graphicx}
\usepackage{tcolorbox}
\usepackage{lipsum}
\usepackage{enumitem}
\usepackage{xcolor}
\usepackage{ragged2e}
%\usepackage{mathptmx}
\usepackage{mathpazo}

\usepackage{fancyhdr}
\pagestyle{headings}  % Put section name in top left, page number in top right
%\pagestyle{fancy}
%\fancyhf{}  % Clear all headers and footers (including default page number).
%\setlength{\headheight}{15pt}
%\renewcommand{\headrulewidth}{0pt} % remove the header rule
%\lhead{text} % Top left
%\rhead{text} % Top right
%\cfoot{text} % Top center
\rfoot{\thepage} % Page numbers on bottom right corner

\usepackage[symbol]{footmisc}
\usepackage{perpage}
\MakePerPage{footnote}
%\renewcommand{\footnoterule}{\kern-3pt\hrule width\textwidth height 0.4pt\kern 2pt}

\definecolor{bred}{rgb}{0.8, 0.0, 0.0}
\definecolor{cadmiumorange}{rgb}{0.93, 0.53, 0.18}
\definecolor{darkcyan}{rgb}{0.0, 0.55, 0.55}
\definecolor{cadmiumgreen}{rgb}{0.0, 0.42, 0.24}
\definecolor{mypur}{rgb}{0.4, 0.22, 0.33}
\definecolor{myblue}{rgb}{0.0, 0.2, 0.6}
\definecolor{mygreen}{rgb}{0.67, 0.88, 0.69}
\definecolor{mygray}{rgb}{0.90, 0.90, 0.90}

\newcommand{\mynotes}[1]{\textcolor{cadmiumorange}{#1}}
\usepackage{tikz}\usetikzlibrary{backgrounds, shapes.misc}
\newcommand*\circled[1]{\tikz[baseline=(char.base)]
            \node[draw=black,shape=rounded rectangle,draw,inner sep=10pt] (char) {#1};}

%\setcounter{secnumdepth}{1}
\usepackage{titlesec}
%\usepackage[compact]{titlesec}
%   [<shape>]{<format>}{<label>}{<sep>}{<before-code>}{<after-code>}
\titleformat{\section}%
  {\fontsize{16}{18}\selectfont\bfseries\color{myblue}}
  {\fontsize{46}{50}\selectfont\color{mypur}\arabic{section}\color{black}$\vert$}
  {0em}{}
\titleformat{\subsection}%
  {\fontsize{14}{16}\selectfont\bfseries\color{mypur}}
  {\color{myblue}\circled{\arabic{section}.\arabic{subsection}}}
  {0.5em}{}
  [\vspace{-2.5pt}{\color{mygray}\titlerule[5pt]}]
  %[\vspace{-20pt}\colorbox{mygray}{% \begin{minipage}{\textwidth}% %\vspace*{2pt}%Space before \hfill %\vspace*{2pt}%Space after \end{minipage}}]
\titleformat{\subsubsection}%
  {\fontsize{13}{14}\selectfont\bfseries\color{mypur}}
  {\color{myblue}\arabic{section}.\arabic{subsection}.\arabic{subsubsection}}
  {1em}{}
  [\vspace{-2.5pt}{\color{mygray}\titlerule[3pt]}]
\titleformat{\paragraph}%
  {\fontsize{12}{13}\selectfont\bfseries\color{myblue}}
  {}
  {0.5em}{}

%\titlespacing*{⟨command⟩}{⟨left⟩}{⟨before-sep⟩}{⟨after-sep⟩}[⟨right-sep⟩]
\titlespacing*{\section}{-0.75in}{0ex}{0ex}
\titlespacing*{\subsection}{-0.25in}{0ex}{0ex}[-0.25in]
\titlespacing*{\subsubsection}{0pt}{2ex}{-1ex}
\titlespacing*{\paragraph}{0pt}{1ex}{-2ex}

% Lists
\setlist[itemize]{noitemsep, topsep=0ex,
    label=$\vcenter{\hbox{\tiny$\bullet$}}$}
%\renewcommand{\labelitemi}{$\vcenter{\hbox{\tiny$\bullet$}}$}
%\renewcommand{\labelitemii}{$\vcenter{\hbox{\tiny$\circ$}}$}
\setlist[enumerate]{noitemsep, topsep=0ex}
\setlist[description]{noitemsep, topsep=0ex,}
\definecolor{cadet}{rgb}{0.33, 0.41, 0.47}

\renewcommand{\descriptionlabel}[1]{%
    \bfseries\textcolor{cadet}{#1}
}
\definecolor{mypurple}{rgb}{0.41, 0.16, 0.38}
\definecolor{pinegreen}{rgb}{0.0, 0.47, 0.44}
\usepackage{hyperref}
\hypersetup{colorlinks=true,
    urlcolor=myblue,
    linkcolor=black,}
\urlstyle{same}
\newcommand{\myref}[1]{\textcolor{pinegreen}{\S{} \ref{#1}}}

\begin{document}
\setlength{\parskip}{0pt}
\tableofcontents
\newpage
\setlength{\parskip}{10pt}
\hypersetup{colorlinks=true,
    urlcolor=myblue,
    linkcolor=pinegreen,}

\section{History}
\begin{enumerate}
    \item 1700's: Messier objects, 39 of
        which are actually galaxies (total of 110 in the final catalogue).
    \item 1864: GC, 1888: NGC (William Herschel and son John)
    \item 1920's: `Great Debate' between Curtis and Shapley on whether
        or not galaxies were located within the MW\@. Resolved by Hubbles
        discovery of Cepheids in M31.
    \item 1980's: Importance of environment recognized:
        \textbf{morphology/density relation}; `Nature vs. Nurture'.
    \item 1990's: Techniques for finding and confirming high redshift
        galaxies (z>2): \textbf{Lyman break galaxies}:
        \begin{itemize}
            \item Lyman limit
            \item Rydberg formula: $\frac{1}{\lambda}=
                R\Big(\frac{1}{n_u^2}-\frac{1}{n_l^2}\Big) $
        \end{itemize}
        First large scale surveys, both nearby and at medium redshift. HST
        imaging of distant galaxies. N-body (dark matter) simulations.
    \item 2000's:Precision Cosmology and LCDM, outside optical
        wavelengths: IR (Spitzer) and sub-mm (JCMT).
    \item 2010's: Extended gas halos in galaxies. Possibly detection of
        DM and dark energy.
\end{enumerate}

\section{Approaches}
Multifaceted approach to studying galaxy formation and evolution:
\subsection{Galaxy ``archaeology''}
Study nearby galaxies in detail,
attempt to understand processes that led to their current
appearence.
\begin{itemize}
    \item Advantages: can resolve structure, individual stars
        in nearest galaxies, high S/N observations
    \item Disadvantages: some information may be erased by
        physical processes (e.g.\ merging), degeneracies in
                    integrated light
\end{itemize}
\subsection{Distant galaxies}
look at galaxy samples at different lookback
times, study distribution of properties (galaxy population) as
a function of time.
\begin{itemize}
    \item Advantages: direct probe of different stages.
        Relationship between lookback time and redshift.
    \item Disadvantages: brightness/selection effects, lack
        of detail, difficulty in associating objects at one redshift
                    to those at another.
\end{itemize}
\subsection{Physics of galaxy formation}
\begin{itemize}
    \item Advantages: some physics (e.g.\ gravity) is well
        understood
    \item Disadvantages: some physics (e.g.\ star formation)
        is not. Dyanamic range of the problem is huge.
        \begin{itemize}
            \item Dynamic range in distances, from stellar scales
                to largest scale structure.
            \item Dynamic range in mass, from stellar scales
                (1 $M_{\odot}$) to
                largest scale structure ($10^6-10^{15} M_{\odot}$)
        \end{itemize}
\end{itemize}

\section{Overview of galaxies and galaxy formation}
Components:
\begin{description}
    \item [Dark Matter] (DM); usually non-baryonic, though even some baryonic
        matter can be hard to see, such as brown dwarfs. DM dominates mass of
        galaxies.
    \item [Stars] Observed properties depend primarily on mass, age,
        and composition. Variety leads to multiple luminosities and colors
        in galaxies.
    \item [ISM] molecular, atomic, and ionized gas phases; dust; mass
        of ISM varies widely between galaxies.
    \item [Central (supermassive) black holes]
\end{description}
Why do galaxies appear as they do? Important processes (not necessarily
in chronological order):
\begin{itemize}
    \item Gravitational collapse (of dark matter, and later, baryons)
        in cosmological framework.
        \begin{itemize}
            \item How big are initial lumps at different size scales?
            \item How much angular momentum?
            \item How fast do lumps grow?
        \end{itemize}
    \item Condensation of gas and cooling
        \begin{itemize}
            \item ``hot'' vs. ``cold'' accretion
        \end{itemize}
    \item Star Formation (not well understood)
        \begin{itemize}
            \item Under what conditions do stars form?
            \item What types (masses) of stars form?
            \item Drives chemical evolution, which may impact
                cooling and future star formation
        \end{itemize}
    \item Black hole formation
        \begin{itemize}
            \item Primordial formation vs.\ formation from early stars
            \item How common?
        \end{itemize}
    \item Feedback/mass loss
        \begin{itemize}
            \item How much energy? Does mass escape or just delay accretion?
            \item What objects generate it? Winds, supernovae, galactic
                nucleii? How?
        \end{itemize}
    \item Continued accretion from IGM
        \begin{itemize}
            \item How much?
            \item What mode?
            \item What composition?
        \end{itemize}
    \item Merging: Gas-rich vs.\ gas-poor
    \item Cluster (group?) environment: ram pressure, tides
    \item Dynamical evolution
        \begin{itemize}
            \item Dynamical instabilities
            \item Migration
            \item Internal vs.\ external triggers
        \end{itemize}
\end{itemize}
These processes have \emph{characteristic timescales}, and the relation between
them may influence how galaxies form, evolve, and appear.\footnote{
see Mo et al.}

\subsection{Example}
Overly simple, but illustrates the consequences of this scenerio\footnote{
Rees 1995}.
\begin{itemize}
    \item Galaxies appear to have a maximum size of order
        10$^{11}$ - 10$^{12} M_{\odot}$, even though there are larger
        agglomerations of matter, e.g.\ clusters.
    \item Can be crudely explained by understanding of dissipation
    \item Self-gravitating cloud has two timescales:
        \begin{enumerate}
            \item dynamical, or free-fall:
                \[
                    t_{dyn} \sim (G\rho)^{-1/2}
                \]
            \item cooling time:
                \[
                    t_{cool} \sim \frac{nkT_{g}}{n^{2}\Lambda(T)}
                \]
                where $\Lambda$ is the cooling function
        \end{enumerate}
        If $t_{cool} > t_{dyn}$, then a cloud can be in quasi-static
        equilibrium, i.e.\ cooling is unimportant. If $t_{cool} < t_{dyn}$
        the cloud cools, kinetic energy is converted to radiation, and
        the cloud collapses. Given a cooling curve for primordial
        composition, one can calculate the relevant timescales, and find
        that collapse is unlikely to occur for $M>10^{12}M_{\odot}$.
        This implies that dissipation is important, at least for
        objects we observe as galaxies (e.g.\ luminous objects).
\end{itemize}

This argument is really only a suggestion, for a number of reasons:
halos are not uniform density, so there's no such thing as a single
cooling time for the entire halo.

\textcolor{red}{More Questions: Add these to the ones above, and keep them in mind
while studying. Read with a question in mind, connect old
information to new information. Add main bullets first, then go
through and add details, depending on how much time you have.}

\subsection{Questions}
\begin{itemize}
    \item When do each of these steps happen and what are their
        relative importances?
    \item What sets the masses of galaxies? Sizes?
        ($10^6-10^{12}M_\odot$) Luminosities?
    \item What sets the distributions of numbers of galaxies as a
        function of mass/luminosity?
    \item Does the ratio of baryonic mass/total mass change for
        different galaxies?
    \item What triggers star formation in galaxies?
    \item What is responsible for the rango of galaxy morphology?
    \item How much of present structure is determined by initial
        conditions, e.g.\ initial overdensity, angular momentum (and what
        are those initial conditions)?
    \item How much does present appearance depend on basic physics
        within galaxies, e.g.\ dynamics and chemical evolution?
    \item How much depends on environment, e.g.\ mergers and
        interactions, background radiation?
    \item Does the relative importance of these effects (initial
        conditions, internal evolution, environment) vary for different
        galaxies?
\end{itemize}
\newpage


\newpage
\section{Observing galaxies}
\subsection{Imaging}
\textbf{Surface Brightness (SB)} is the basic measured property for imaging of
a \emph{resolved} object.
\textcolor{bred}{SB is independent of distance until geometry of the
universe becomes important:}
\[
    SB \propto (1+z)^{-4}
\]

Units:
\begin{itemize}
    \item erg cm$^{-2}$ s$^{-1}$ sterradian$^{-1}$
    \item mag arcsec$^{-2}$
\end{itemize}
For adding multiple SBs, flux units must be used.
If you're adding and subtracting magnitude units, you're probably
doing something wrong. \Smiley
Differences in magnitudes correspond to \emph{ratios} in fluxes:
\[
    m_{1} - m_{2} = -2.5\log\frac{F_{1}}{F_{2}}
\]
In general, flux is a function of wavelength, so SB is measured in
different bandpasses, e.g.\ UBVRI or SDSS (ugriz).
The ratio of the flux of the same object in different bandpasses
gives an estimate of the \textbf{spectral slope} between bandpasses;
often expressed as a \emph{color index}, the difference in magnitude
between the two bandpasses, such as U-R.

Relevance of SB distribution:
\textcolor{bred}{How are stars distributed in galaxies?}
Be aware that \emph{mass} in stars doesn't necessarily track
\emph{light} in stars (nor does it trace mass in other components).

In principle, SB is measured directly from a 2D detector as an
arbitrary function of location. However,
galaxies are relatively faint. Typical values in galactic centers are:
\begin{description}[align=right,labelwidth=5em]
    \item [Spiral] $V \sim 20 - 21$ mag arcsec$^{-2}$
        (not much brighter than typical dark sky)
    \item [Elliptical] $V \sim 16 - 17$ mag arcsec$^{-2}$
\end{description}
More than half the light from galaxies comes from regions with
$SB < SB_{\mathrm{sky}}$ so $S/N$ is difficult to obtain.

Problems of seeing and \href{http://astronomy.nmsu.edu/holtz/a555/html/diagrams/a616/sky.htm}
{sky determination} for determination of SB distribution in center and
outer parts of galaxies. Sky problems are worse with small detectors/large
galaxies.

Since SB is the key observable, there can be strong \emph{selection effects}
against low SB objects. If all galaxies had the same SB profile (they don't)
then low SB galaxies are \emph{strongly} biased against in either
magnitude-limited or size-limited catalogs.
The apparent ``size'' of a galaxy will depend on its SB.

Most galaxies are symmetric at a significant level;
\textbf{isophotes} (contours of constant SB) are often well represented by ellipses.
Elliptical contours fit spirals and ellipticals for different reasons:
\footnote{Some basics of techniques for ellipse fitting:
Kent, ApJ 266, 562 (1983); Lauer, ApJ 311, 34}
\begin{description}[align=right, labelwidth=5em]
    \item [Spiral] viewing angle/disk thickness
    \item [Elliptical] intrinsic ellipticity
\end{description}

In reality, galaxies have more complex features, e.g.\ asymmetries,
departures from elliptical isophotes, bars, spiral arms, jets, etc.\
(see below).

For a purely axisymmetric object, SB distribution reduces to a 1D SB profile.
SB profiles are often parameterized, with distributions in the form
of a
\href{http://astronomy.nmsu.edu/holtz/a555/html/diagrams/a616/sersic.htm}
{\textbf{Sersic Law}}\footnote{plot from
\href{http://adsabs.harvard.edu/abs/2003ApJ...582..689M}
{MacArthur et al. (2003)}}
(though there are other forms that can be used as well):
\[
    \Sigma(r) = \Sigma_e\exp\left(-b_n\left[\left(
    \frac{r}{r_e}\right)^{1/n}-1\right]\right)
\]
\begin{itemize}
    \item $\Sigma_{r}$ = SB at radius $r$
    \item $r_{e}$ = \textit{effective} radius (aka.\ half-light radius),
        the radius enclosing half the
        total light if the model is extrapolated to infinity.
    \item $\Sigma_{e}$ = SB at $r_{e}$
        ($\Sigma_{0} \sim 2000\Sigma_{e}$) \mynotes{at galaxy center?}
    \item $b_{n} \approx 2n - 0.324$ and is determined from the definition of
        $r_{e}$ \mynotes{(I assume this is talking about where that particular
        expression came from, including the constants.)}
    \item $n$ depends on the type of galaxy ($n=4$ for ellipticals and $n=1$
        for disks).
\end{itemize}
\subsubsection{SB profile for disk galaxies}
Disks are usually well represented \emph{on average} by an exponential:
\[
    \Sigma(r) = \Sigma_{s}\exp\left(-\frac{r}{r_{s}}\right)
    \quad\mathrm{[flux\;units]}
    \]
\[
    m(r) = m_{s} + kr
    \quad\mathrm{[magnitudes]}
    \]

\subsubsection{SB profile for spheroids}
Spheroids are historically characterized by
\textbf{deVaucouleurs profile}, otherwise known as the ``$r^{1/4}$'' law:
\[
    \Sigma(r) = \Sigma_e\exp\left(-7.67\left[\left(
    \frac{r}{r_e}\right)^{1/4}-1\right] \right)
    \quad\mathrm{[flux\;units]}
    \]
\[
    m = m_{0} + kr^{1/4}
    \quad\mathrm{[magnitudes]}
    \]

\subsubsection{Sizes of galaxies}
Generally, galaxies do not appear to have sharp edges (although SB can
drop sharply in some disk galaxies).
\begin{itemize}
    \item Isophotal ($D_{25}$ \mynotes{???} in RC \mynotes{(rotation curve?)}
        is B=25.0): Beware for low SB galaxies \mynotes{???}
    \item Half-light radii (e.g.\ $R_{e}$)
    \item Petrosian radius: radius at which the \emph{local}
        SB at that radius drops to some fraction of the \emph{mean} SB within
        that radius.  (in an annulus, SDSS used 0.8$r$ - 1.25$r$, with a
        fraction of 0.2)
\end{itemize}
Typical angular sizes: arcminutes for nearby galaxies, arcseconds for
distant galaxies. Can get linear sizes with distance estimate
(which depends on cosmology). Typical physical size (of the luminous
component) is $\sim$ 1-30 kpc.

\subsubsection{Integrated brightness}\label{intb}
Issue: galaxies don't have edges. Types of mags:
\begin{description}[align=right, labelwidth=0.5in,
        labelindent=0.25in,
        leftmargin=1in]
    \item [Metric:] brightness within an aperture of fixed
        angular or physical size. Be careful if galaxies have a range
        of sizes.
    \item [Isophotal:] brightness within a specified SB
        contour, e.g.\ Holmberg mags, which is the brightness within
        $\mu_{pg} = 26.5$. \mynotes{wtf is $\mu_{pg}$??}
        Be careful if galaxies have a range of SB profiles.
    \item [Model:] require assumption about outer SB profile
    \item [Petrosian:]
        mag within some fixed number of Petrosian
        radii (SDSS uses 2).
        Advantages with regard to SB (distance) and seeing effects. Note
        \href{http://astronomy.nmsu.edu/holtz/a555/images/petrogal.htm}
        {relation of Petrosian mag to model mag} depends on SB profile.
        \footnote{\url{http://classic.sdss.org/dr5/algorithms/photometry.html\#mag_petro}}
\end{description}
Typical integrated brightnesses of galaxies, and star-galaxy separation
and crossover in number counts \mynotes{(\ldots yes?)}

Other issues in getting luminosities of galaxies:
\begin{itemize}
    \item Foreground extinction, which depends strongly on Galactic
        latitude
        \footnote{\href{http://adsabs.harvard.edu/cgi-bin/nph-bib_query?bibcode=1984ApJS...54...33B}
            {Burstein \& Heiles, ApJS 54,33 (1984)};
            \href{http://adsabs.harvard.edu/cgi-bin/nph-bib_query?bibcode=1998ApJ...500..525S}
        {Schlegel, Finkbeiner, \& Davis, ApJ 500,525 (1998)}}
    \item Internal extinction/inclination, which can be significant
        (up to 1.5 mag in RC3 prescription \mynotes{(???)})
    \item If looking at a population, be sure to compare luminosities in
        the same bandpass. This is an issue if looking over a range of
        redshifts; a fixed observed wavelength range corresponds to a
        different rest frame wavelength range at different redshifts.
        \textbf{K-corrections} are used to account for this at some
        level\footnote{\href{http://arxiv.org/abs/astro-ph/0210394}
            {Hogg et al. astro-ph/0210394};
            \href{http://adsabs.harvard.edu/cgi-bin/nph-bib_query?bibcode=1996ApJ...467...38K}
            {Kinney et al. ApJ 467.38 (1996)};
            \href{http://adsabs.harvard.edu/cgi-bin/nph-bib_query?bibcode=1980ApJS...43..393C}
            {Coleman, Wu, Weedman, ApJS 43,393 (1980)}}.
        These require knowledge/assumptions about the spectral energy
        distribution of galaxies. Alternatively, measure at different
        observed wavelengths.
\end{itemize}

\subsection{Spectroscopy}
The SED of a galaxy is the other fundamental observable; obviously,
the most fundamental is spectral energy SB, but spectra are often
obtained over some aperture, or long slit. Note
increasing use of integral field spectroscopy (e.g.\ SDSS MaNGA).

The overall shape and features in \href{http://astronomy.nmsu.edu/holtz/a555/resources/galaxyspectra.gif}
{SEDs} (see also \href{http://astronomy.nmsu.edu/nicole/teaching/ASTR505/lectures/quickview.html}
{here}) provide important information
about the nature of the luminous components, i.e.\ physical properties
of the stellar population and the gas component
As we'll discuss later, these can be challenging to sort out.
The composite spectrum of a galaxy is the sum of different types of
stars, weighted by relative numbers and luminosities, plus emission
from gas and dust.

Spectra also allow the study of velocities via the Doppler shift:
\[
    \frac{\nabla\lambda}{\lambda} = \frac{v}{c}
    \]
which measures both the bulk velocity of the galaxy and the
distribution of stellar velocities within the galaxy.
\paragraph{Bulk velocities}
\begin{itemize}
    \item Mean velocities of galaxies indicate that they are receding, with
        velocity proportional to distance $\rightarrow$ expansion of the
        universe.
    \item There is a dispersion of velocities around the mean relation; the
        deviation is called the \textbf{peculiar velocity}.  In the field,
        typical peculiar velocities are a few hundred km s$^{-1}$.  In
        clusters, they can be a thousand km s$^{-1}$.
\end{itemize}
\paragraph{Internal velocities}
\href{http://astronomy.nmsu.edu/holtz/a555/resources/intvel.gif}
{Internal velocities} can be organized (rotation in disks), or they can
be random, leading to broader lines that represent the
velocity distribution of stars within a galaxy.
\mynotes{This broadening would be the same at all points in the galaxy.}
\[
    F(\lambda) \propto \int{F(v_{los})S\left(\lambda-\frac{v_{los}\lambda}{c}
    \right)\mathrm{d}v_{los} }
    \]

\begin{itemize}
    \item $S$ = spectrum of an individual star
    \item $F$ = fractional distribution of stars at
        different line-of-sight velocities.
\end{itemize}
Lines are usually fairly well represented
by a Gaussian velocity profile, characterized by the mean velocity:
\[
    \int{v_{los}F(v_{los})\mathrm{d}v_{los}}
    \]
and the velocity dispersion:
\[
    \int{(v_{los}-\overline{v}_{los})F(v_{los})\mathrm{d}v_{los}}
    \]
e.g., central velocity dispersion or, more generally, the velocity
dispersion profile.

Deviations from a Gaussian can be measured, and are usually characterized
by higher order moments: \textit{skew} ($h_{3}$) and \textit{kurtosis}
($h_{4}$):
\[
    F(v_{los}) \propto \exp\frac{-\omega^{2}}{2}\left[
        1 + \sum{h_{k}H_{k}(\omega)}\right]
\]
where $\omega \equiv (v_{los}-\overline{v})/\sigma$ and
$H_{k}$ are Gauss-Hermite functions.
Some
\href{http://astronomy.nmsu.edu/holtz/a555/resources/examplesh3h4.gif}
{examples of velocity data}.

\underline{\textbf{Relevance}}:
Internal velocities can be used to probe mass distribution within
galaxies (assuming that gravity drives the motions).
For a rotationally supported system:
\[
    v(r) = \sqrt{\frac{GM(r)}{r}}
    \]
For a collisionless system of \mynotes{(independent)} particles,
the distribution of velocities at
any given location characterized by the velocity ellipsoid, and
Jeans equation leads to:
\[
    \frac{\mathrm{d}\rho\sigma^{2}}{\rho{\mathrm{d}r}} +
    \frac{2\beta\sigma^{2}}{r} = -g(r)
    \]
where $\beta$ is the \textbf{velocity anisotropy}:
\[
    \beta \equiv \left(1-
    \frac{\sigma_{\theta}^{2}+\sigma_{\phi}^{2}}{2\sigma_{r}^{2}}\right)
    \]
though $\beta$ is not generally known.

\subsection{Distances to galaxies}
Some properties of galaxies, such as luminosity and linear size,
require knowledge of distances to galaxies:
\[
    m-M = 5\log{d_{L}}-5
\]
\begin{itemize}
    \item $d_{L}$ = distance [pc] \mynotes{$L$ = ?}
    \item $M$ = absolute magnitude (or SB)
    \item $m$ = apparent magnitude
\end{itemize}

Various techniques for getting distances\footnote{see
\href{http://adsabs.harvard.edu/abs/2010ARA\%26A..48..673F}
{Freedman \& Madore, ARAA review}}:
\begin{description}
    \item [Variable stars] Cepheids/RR Lyrae stars through period-luminosity
        (period-luminosity-color/metallicity) relation.
    \item [Geometric techniques]
        \begin{itemize}
            \item masers (proper motion compared with radial velocity in
                central disks)
            \item exclipsing binaries
        \end{itemize}
    \item [SB fluctuations] from imprint of Poisson statistics:
        requires objects with a ``standard'' population (i.e.\ stellar
        luminosity function).
    \item [Planetary nebulae luminosity function]
    \item [More luminous standard candles:] type-Ia supernova (``standardizable'')
    \item [Scaling relations between velocity and luminosity]
        \mynotes{for relatively nearby galaxies}
        \begin{itemize}
            \item Spirals: \textbf{Tully-Fisher relation} between maximum
                rotational velocity and luminosity
            \item Ellipticals: fundamental plane relation between velocity
                dispersion, SB, and physical size
                ($D_{n}-\sigma$) relation.
        \end{itemize}
    \item [Redshift] (see next section)
\end{description}
\subsubsection{Distances from redshift}
Galaxies expand with everything else in the cosmic flow.
\[
    \frac{\Delta\lambda}{\lambda} = z;\quad
    1+z = \frac{lambda_{obs}}{\lambda_{emit}}
    \]
At \emph{low} $z$, $z \approx v/c, v=Hd$.
At \emph{high} $z$, need knowledge of cosmological
parameters to get distance from redshift (i.e.\ redshift-distance
relation.\footnote{see, e.g.\
\href{http://ned.ipac.caltech.edu/level5/Hogg/Hogg_contents.html}
{Hogg, Distance Measures in Cosmology}})

Galaxies also move around relative to the uniform expansion because of local
gravitational attraction from concentrations of mass.
As mentioned above, the deviation of the
velocity from the smooth flow is called the \textit{peculiar velocity}.
Typical peculiar velocities are on the order of several hundred km s$^{-1}$,
but depend on environment; c.f., \mynotes{??}
Milky Way and galaxy clusters; if these are ignored, introduces spurious
features in distance maps
(\href{http://astronomy.nmsu.edu/holtz/a555/resources/cfa.html}
{``fingers of God'', redshift space distortions}).

At $z \gtrsim 0.1$, peculiar velocities lead to relatively small distance
errors. At lower redshifts, might estimate distances using model of mass
distribution and velocity field, but this is likely not well determined.

\paragraph{Spectroscopic redshift}
Determining redshift spectroscopically for distant (faint) galaxies is
observationally time-consuming. Accurate positions of individual lines
require reasonable S/N; emission line galaxies are significantly easier.
It is possible to get good redshifts from low
S/N observations by taking advantage of multiple spectral features,
e.g., by cross-correlation of spectrum against a template
in $\log\lambda$ space.

\paragraph{Photometric redshift}
Lower accuracy distances are possible from so-called \textit{photometric
redshifts} \mynotes{(though can't disperse light.)}
Use low resolution spectral
information, i.e.\ multiband photometry to constrain redshift.\footnote{
    e.g. \href{http://astronomy.nmsu.edu/holtz/a555/resources/photoz.png}
    {plot} (from Niemack et al 2009, ApJ)}
The different intrinsic SEDs of galaxies, internal reddening, and
redshift need to be separated.
Accuracy of results obviously depends on quality of photometry
and number of bandpasses. Typical accuracies (checked against spectroscopic
redshifts) give
\[
    \frac{\Delta{z}}{\left(1+z\right)} \sim 0.1
    \]
(sometimes better); note typical $\sim$ 10\% outlier fraction
(varies for different types of galaxies).

\subsection{Morphological classification}
Historically, galaxies were considered in terms of their morphology, i.e.\ the
Hubble sequence.  However, it is not totally clear to what extent morphological
classification traces underlying physics, and descriptive morphology may be
biased by things that may not be fundamental.  Still, it is widely used, so
important to understand a bit.

\subsubsection{Morphological systems}
Good reference: Sandage in Galaxies and the Universe, 1975.
Some pictures:
\begin{itemize}
    \item \href{http://astronomy.nmsu.edu/holtz/a555/html/diagrams/a616/ellips.htm}
        {{ellipticals}}
    \item \href{http://astronomy.nmsu.edu/holtz/a555/html/diagrams/a616/s0.htm}
        {{S0s}}, also known as lenticulars
    \item \href{http://astronomy.nmsu.edu/holtz/a555/html/diagrams/a616/spirals.htm}
        {{spirals}}
    \item \href{http://astronomy.as.virginia.edu}
        {{Tuning fork diagram}}
\end{itemize}

\subsubsection{Hubble classification}
\begin{itemize}
    \item \textbf{ellipticals:} given as En, where
        \[
            n = 10(\frac{1-b}{a})
        \]
        E.g., E0 (sphere) $\rightarrow$ E7 (skinny).
        No distinction between dwarf ellipticals and dwarf spheroidals
    \item \textbf{spirals:} barred or unbarred (SBa,SBb,SBc; or Sa,Sb,Sc;
        respectively)
        where a,b,c denote size of bulge (bulge-to-disk ratio: B/D)
        in decreasing order. Also classified according to tightness of
        arms and the degree to which arms are resolved into HII regions
        (note that the latter two are essentially impossible to judge
        for edge-ons).
        Later, have SA (normal spirals), SB (barred), and SAB (transition)
    \item \textbf{SOs (aka lenticulars, or disks):} intermediate, no spiral
        structure but have disk system, split into $S0_1$, $S0_2$, $S0_3$,
        depending on the amount of dust.
    \item \textbf{Irregulars:}
        \begin{itemize}
            \item Irr I (Magellenic irregulars with lots of distinct HII regions)
            \item Irr II (lack the resolution into distinct HII regions).
        \end{itemize}
    \item Some pictorial examples
\end{itemize}

\subsubsection{deVaucouleurs/RC3 classification}
\href{http://astronomy.nmsu.edu/holtz/a555/html/diagrams/a616/rc3class.htm}
{diagram}.
See also \href{http://astronomy.as.virginia.edu}{here}.
\begin{itemize}
    \item Extends Hubble to later spiral types Sd, Sm, and finally Im.
    \item Extra classes around $S0$ ($S0^{-}, S0^{+}$)
    \item Allows for intermediate between barred and unbarred:
        \begin{itemize}
            \item SA: normal spirals
            \item SB: barred spirals
            \item SAB: transition
        \end{itemize}
    \item Adds extra distinction for \emph{ring} vs.\ \emph{s} shaped.
    \item Note that location along spiral dequence may not be the same as in
        ``Hubble classification'' (more based on B/D).
    \item Some \href{http://astronomy.as.virginia.edu}
        {{pictorial examples}}
    \item \href{http://astronomy.nmsu.edu/holtz/a555/html/diagrams/a616/numtype.htm}
        {{Numerical galaxy types (T)}}
\end{itemize}
Note that morphological classification often depends on multiple
characteristics, and therefore it can be somewhat subjective.
Quantitative classification schemes have been worked on
%(see \S{} \ref{sssec:quantitative}),
(see \myref{sssec:quantitative}),
but are not widely used
for nearby galaxies.

\subsubsection{Wavelength dependence}
Morphology depends on wavelength
(e.g.\ \href{http://ned.ipac.caltech.edu/level5/Kuchinski/frames.html}
{O'Connell 9609101}).
Be careful about morphological classification at higher redshift
(or at least a direct comparison with lower redshift galaxies)
because you may be looking at morphology at a different wavelength
(morphological \textbf{K Correction:} correction to magnitude (or flux) due
to redshift).

Based on global characteristics, the Hubble morphological classification of
ellipticals is probably not fundamental. It may be more meaningful to classify
by \emph{isophotal shape}
(\href{http://astronomy.as.virginia.edu}{(Kormendy and Bender classification)})
or kinematically by $v/\sigma$.

For the spiral sequence, there are correlations of luminosity, surface
brightness, rotational velocity and gas fraction with Hubble type. However,
each category has a broad range of global observables, and the categories overlap
significantly.\footnote{Some representative data from
\href{http://adsabs.harvard.edu/cgi-bin/nph-bib_query?bibcode=1994ARA\%26A..32..115}
{Roberts and Haynes review (ARAA 1994)}}:
\begin{itemize}
    \item Considering mean or median values, there is little trend in
        radius, $L_{B}$ \mynotes{(Bolometric luminosity?)},
        and total mass for S0-Sc, but later types tend
        to be smaller, less massive, and less luminous
        (\href{http://astronomy.nmsu.edu/holtz/a555/html/diagrams/a616/rh2.htm}
        {{RH figure 2}}).
        However, note that LSB \mynotes{(Low Surface Brightness?)}
        galaxies tend to be later types, and
        since these are not included, there may be a bias here.
    \item Typical SB may be lower for latest types, even without inclusion
        of LSB galaxies; CSB(?????) much higher for large ellipticals than
        spirals. Mass surface density appears to decrease monotonically with
        morphological class in spirals
        (\href{http://astronomy.nmsu.edu/holtz/a555/html/diagrams/a616/rh3.htm}
        {{RH figure 3}}).
    \item Cold gas is absent in ellipticals, and neutral hydrogen (HI) content
        appears to increase monotonically with type for spirals. It is possible
        that molecular gas content decreases with type, but not certain.
        (\href{http://astronomy.nmsu.edu/holtz/a555/html/diagrams/a616/rh4.htm}
        {{RH figure 4}}).
\end{itemize}

\subsubsection{Quantitative classification schemes}\label{sssec:quantitative}
As stated above, all of the aforementioned classification schemes are
subjective at some level.  Some of the more quantitative schemes that have been
proposed, both parametric and non-parametric, are:
\begin{itemize}
    \item \textbf{Bulge-to-disk ratio}, using B/D decomposition\footnote{
        \href{http://adsabs.harvard.edu/cgi-bin/nph-bib_query?bibcode=2003ApJ...582..689}
        {MacArthur, Courteau, \& Holtzman, ApJ 582, 689 (2003)}}.
        Issues: covariance between parameters, 1D vs.\ 2D, and validity of models.
    \item \textbf{Global profile fit}, e.g.\ Sersic index
    \item \textbf{Concentration}, e.g.\ SDSS $r_{90}/r_{50}$, $r_{80}/r_{20}$,
        using circular apertures. Elliptical apertures can also be considered,
        e.g.\ based on \emph{second order moments}.
        Related to B/D. Sensitivity to seeing/distance.
\end{itemize}

\subsubsection{Non-symmetric galaxies}
Many of the schemes used for nearby galaxies fail for non-symmetric galaxies,
for which other indicators have been suggested:
\begin{itemize}
    \item \textbf{Asymmetry:}\footnote{\href{http://adsabs.harvard.edu/cgi-bin/nph-bib_query?bibcode=1996ApJS..107....1A}
        {Abraham et al, ApJS 107, 1, 1996}}
        Rotate about center, self-subtract,
        \[
            A = \frac{0.5\vert\mathrm{subtracted}\vert}{\mathrm{total}}
            \]
        Possible indicator of mergers.
    \item \textbf{Clumpiness:}\footnote{\href{http://adsabs.harvard.edu/cgi-bin/bib_query?2003ApJS..147....1}
        {Conselice ApJS 147, 1 (2003)}} e.g.\ S. Subtract smoothed version of
        image from original image, ratio of flux in subtracted image to flux in
        original images gives S. Possible indicator of star formation.
    \item \textbf{\href{http://astronomy.nmsu.edu/holtz/a555/images/gini.htm}
        {Gini coefficient:}}\footnote{\href{http://adsabs.harvard.edu/cgi-bin/nph-bib_query?bibcode=2003ApJ...588..218A}
        {Abraham et al, ApJ 588, 218 (2003)}}
        Applied to sorted list of pixel values. May be more stable at low
        resolution, S/N.
    \item \textbf{M20:} second order moment of brightest 20\% of
        galaxy light
        \footnote{\href{http://adsabs.harvard.edu/cgi-bin/nph-bib_query?bibcode=2004AJ....128..163L}
        {Lotz et al., AJ 128, 163 (2004)}}
        May be more stable at low resolution, S/N.
\end{itemize}

\newpage
\section{Statistical properties of galaxies}
Consider relative number of galaxies with different parameters,
e.g.\ color or luminosity.
\subsection{Luminosity Function}
The luminosity function (LF) gives the number density of galaxies as a
function of luminosity, which can be expressed either in luminosity or
absolute magnitude units.
\begin{description}[labelwidth=3em, labelindent=0.25in]
    \item [$\Phi(L)$] = number density of galaxies with
        luminosity between $L$ and $L$ + $\mathrm{d}L$
    \item [$\Phi(M)$] = number density of galaxies with absolute
        mag between $M$ and $M$ + $\mathrm{d}M$
\end{description}
The integral over the LF gives the total number density of galaxies (at all
luminosities). The LF is the first step toward understanding a fundamental
question of galaxy formation: What sets the \emph{range} of galaxy luminosities
and the relative \emph{numbers} of objects at different luminosities (though
mass may be a more fundamental characteristic). It is also an important
cosmological probe for evolution of the galaxy population (more later).

\subsubsection{Measuring the LF}
Obviously, distance measurements are required to get luminosity.
Measurement critically depends on knowledge of the \textbf{selection
function}. For magnitude limited samples, more luminous objects will
be overrepresented. This is called the Malmquist bias\footnote{Be
aware that this name is sometimes used to refer to different things.}

The simplest idea for correction is to weigh counts by $V_{max}$, the
largest volume to which the galaxy could have been observed given the
selection function. However, the issue here is large-scale structure.
$\rightarrow \cfrac{V}{V_{max}}$ test for uniformity and understanding
of selection function\footnote{Significantly more sophisticated
methods have been developed; see, e.g.\ Binggelli et al (ARAA 26, 26,
1988) for more details}.

Also may need to be careful about systematic biases arising from
uncertainties in measured magnitudes and distances, if population is
not uniform in underlying distribution of absoluate mags and distances
sampled (which it isn't). The LF rises toward fainter magnitudes, so
more galaxies are scattered bright than are scattered faint
\mynotes{(WTF??)}. In distance, more galaxies are located farther
than nearer because of volume element \mynotes{(seriously, WTF)}, so
more galaxies scattered into sample (with corresponding error in
absolute magnitude) than scattered out of it.

Note that many biases are possible aside from a simple magnitude cut, e.g.\
color bias, SB bias, etc. In general, need to be \emph{very} careful about
understanding selection effects; model (and verify) them.

\subsubsection{Observed LF}
\begin{itemize}
    \item Absolute magnitudes ($M_{V}$) range from -22 to -8.
    \item Nomenclature: ``dwarf'' galaxies vs.\ ``normal'' galaxies.
    \item What objects dominate by number? What objects dominate the
        total luminosity?
    \item Different types have clearly different LFs
    \item Since proportion of a given type depends on environment, so
        must the general LF.
\end{itemize}

\subsubsection{Functional fits to the observed LF and parameters of the LF}
LFs are usually well characterized by a \textbf{Schecter Function}
\[
    \phi(L) = \frac{\phi_{*}}{L_{*}}\left(\frac{L}{L_{*}}\right)^{\alpha}
    \exp\left(-\frac{L}{L_{*}}\right)
\]
\[
    \phi(M) \propto 10^{-0.4\left(\alpha+1\right)\left(M-M^*\right)}
    \exp\left(-10^{0.4(M^{*}-M)}\right)
\]
where $\phi(L)$ is the number of galxies with luminosity between
$L$ and $L + dL$.  $\phi_{*}$, $L_{*}$, and $\alpha$ (the faint-end slope)
are the three parameters. The Schecter function is only an approximation, and
is purely empirical.

A possibly useful feature of the Schecter function is that it can be
integrated to get total luminosity density:
\[
    j = \phi_{*}L_{*}\Gamma\left(\alpha + 2\right)
\]
Typical ``local'' parameters (e.g. SDSS at $z\sim0.1$) are:
\begin{itemize}
    \item $\phi^{*}\sim0.015h^{3}$ Mpc$^{-3}$
    \item $M_{B}^{*}\sim-19.5$
    \item $M_{R}^{*}\sim-20.5$
    \item $\alpha=-1$ to $-1.5$
\end{itemize}
As noted above, there is a dependence on type. So what is the implied
luminosity density? And (getting ahead of ourselves) the implied stellar
mass density, c.f. cosmological.

There has been significant discussian over detailed shape and normalization of
LF\@.  There are probably both observational issues (e.g.\ selection functions)
and astrophysical ones (e.g.\ large scale structure, ``cosmic variance'').

\subsubsection{Evolution of the LF}
LF evolution contains a lot of information about galaxy evolution.
Some possibilities:
\begin{itemize}
    \item No evolution: distribution of galaxiy luminosities is unchanging.
    \item Passive evolution: number and internal (stellar) makeup of
        galaxies is unchanging \mynotes{(no star formation)},
        but luminosity evolves as stars evolve.
    \item Luminosity evolution: stellar makeup of galaxies changes with
        time (star \emph{formation}, leading to luminosity changes).
    \item Number (density) evolution: number of galaxies changes with time.
        Galaxies may be either created (formed) or destroyed (e.g.\
        mergers) as a function of luminosity.
\end{itemize}
There is a large amount of work on this, but certainly the LF shows
evolution\footnote{Faber et al. (2007)}. All samples show luminosity
evolution, but red galaxies show number density evolution.
Schecter parameters (link?)

\subsection{Colors of galaxies}
Consider the distribution of SEDs of galaxies, to first order represented by
their color. There is a striking bimodality in color\footnote{quantified from SDSS
by (Strateva et al., AJ 122, 1861 (2001))}:
\begin{itemize}
    \item Widley used terminology of red and blue sequences
        (with ``green valley'' in between).
    \item Red sequence is tighter than blue sequence, so latter is
        sometimes called the ``blue cloud''.
    \item Red sequence extends to higher luminosities, blue to lower
        luminosities, though there is significant overlap.
\end{itemize}
At some level, this bimodality is nothing more than the bimodality between
ellipticals/early-type spirals without much current star formation (red)
and later-type spirals with current star formation (blue), although
dust, bulges, and metallicity all play a role.
The correspondance with morphology is supported
by correlation with structural parameters, e.g.\ multidimensional
correlations\footnote{Blanton and Moustakis ARAA 2009}.

The key point is that color is easily observed and quantified.  Given likely
differences in stellar populations, the relation between luminosity and stellar
mass is probably different for the two different sequences. At a rough level,
the color allows one to estimate the stellar mass from the luminosity, though
some colors certainly have more information about this than others, and some
bandpasses for luminosity are more affected by differences in the stellar
populations (more later).

Note that the relation shifts when expressed in terms of stellar
mass\footnote{Baldry et al. 2006}, and that there appears to be a transistion
mass around $10^{10}$ M$_{\odot}$ between the two sequences.

\section{Elliptical/Spheroidal galaxies}
Elliptical galaxies are not simple collections of stars that are all
similar to one another. They exist over a wide range in luminosity and
other quantities.

\subsection{SB profiles}
Ellipticals better fit by Sersic profiles than by
deVaucouleurs\footnote{Caon et al. 1993}.
\begin{itemize}
    \item Correlation of Sersic indices with other parameters
        suggests that there is something physical going on, but be aware that
        there may be fit degeneracies.
    \item Spiral bulges, low luminosity Es may be better
        represented by exponentials.
    \item Slope of SB profile (Sersic $n$) appears to be correlated with
        luminosity
    \item SB vs L ``turns over'' \mynotes{(?)} at intermediate luminosity:
        two families of shperoidal systems? \mynotes{
            $\rightarrow$ \textbf{Cuspy and Core}}
    \item Ellipticals exist over wide range of sizes, luminosities, and SBs.
\end{itemize}
Other profiles that have been used:
\begin{itemize}
    \item King model (truncated Gaussian for velocity distribution)
    \item Hubble profile:
        \[
            \Sigma(r) = \frac{\Sigma_{s}}{\left(1+\frac{r}{r_{s}}\right)^{2}}
        \]
        where $\Sigma_{s}$ = 0.25$\Sigma_{o}$
\end{itemize}
None are perfect matches to data over all scales\footnote{see Burkert 1993 for
comparison with deVaucouleurs law}.

Inner regions of spheroidals deviate from Sersic profiles fit to outer regions,
(though outer regions can also deviate from Sersic profiles,
especially notable in central cluster (cD) galaxies).
Inner regions are of particular interest because they may reflect dissipational
collapse of low angular momentum material\footnote{See, e.g.\ NGC 4472}.
Initially, it seemed like
galaxies had cores, but this is partly an effect of seeing. Higher spatial
resolution (HST) shows that galaxies have both flat and steep (cuspy) inner
profiles\footnote{e.g.\ NGC720 vs.\ NGC4621}.
Inner profiles seem to be roughly bimodal (though this bimodality is currently
under debate).

Parametric fits to account for these become more complex,
e.g.\ the `nuker law':
\[
    I(r) = I_{b}2^{(\beta-\gamma)/\alpha)}
    \left(\frac{r_{b}}{r}\right)^{\gamma}
    \left[1+\left(\frac{r_{b}}{r}\right)^{\alpha}\right]
    ^{\left(\gamma-\beta\right)/\alpha}
\]
where $\gamma$ is the slope of the inner power law, $\beta$ is
the slope of the outer power law, and $\alpha$ is the sharpness
of the break between them.

Profile type is correlated with luminosity. Luminous
ellipticals ten to have ``cuspy cores'' with a break radius and
shallower central density profile. Lower luminosity ellipticals
have power laws all the way in.

Galaixes with cusps may have ``extra light'', perhaps related to an accrection
event; note that at least some of these show distinct kinematic signature in
their cores.

\subsection{Intrinsic (3D) shapes of ellipticals}
\begin{itemize}
    \item oblate (2 long axes)
    \item prolate (1 long axis)
    \item triaxial (all axes different length).
\end{itemize}
True shapes are determined by by looking at distribution of ellipticiites
\begin{itemize}
    \item Distribution function is different for fainter and
        brighter Es.
    \item For bright giant Es, distribution is \emph{inconsistent} with
        either prolate or oblate itrinsic shapes: not enough circular
        galaxies\footnote{Tremblay and Merritt Fig 3}.
    \item For fainter Es, distribution is \emph{consistent} with
        oblate, prolate, or triaxial.
    \item Triaxiality is also inferred for some giant Es from
        observation of isophotal twisting, which you cant get from
        oblate or prolate shape\footnote{de Zeeuw, Fig 1}.
\end{itemize}

\subsection{Non-axisymmetric features in galaxies}
Often described by amplitudes of Fourier moments of
intensity distribution as a function of radius, e.g.
$\alpha_1$, $\alpha_2$, $\alpha_4$
\[
    I\left(r,\theta\right) = \sum{c_{m}\cos\left(m\theta\right)} +
    \sum{s_{m}\sin\left(m\theta\right)}
\]\[
    \alpha_{m} = \frac{\sqrt{c^{2}_{m} + s_{m}^{2}}}{I_{o}}
\]
``First even term above ellipses is $\alpha_4$ term:'' \mynotes{(WTF???)}
\begin{itemize}
    \item ``boxy" isophotes: $\alpha_4 < 0$;
        bright, slow, central cores, strong radio and x-ray.
    \item ``disky" isophotes: $\alpha_4 > 0$;
        faint, significant rotation-flattening, little
        radio or x-ray emission, steep cusp.
\end{itemize}
Deviations from elliptical fits are seen in slliptical/spheroidal systems,
e.g.\ NGC 821 v.\ NGC 2300. It is possible that isophotal deviations form
a continuous sequence\footnote{e.g.\ Kormendy and Bender classification}:
\begin{center}
S $\rightarrow$ S0 $\rightarrow$ disky Es $\rightarrow$ true Es $\rightarrow$ boxy Es.
\end{center}
It is also possible that true Es are just disky Es viewed
edge-on\footnote{Kormendy and Bender, figure 3}, and that there are two
distinct classes of Es. Some Es also show non-symmetric deviations from
smooth SB distributions, e.g.\ ripples, shells, and streams, especially
in outer regions.

\subsection{Kinematics}
Elliptical galaxies are kinematically \emph{hot}. The random motions of stars
is generally large compared to the organized motion (but this is not to say
that there is \emph{no} rotation).

The key kinematic quantity is \textbf{velocity dispersion}, $\sigma$.  Galaxies
can be characterized by \emph{central} $\sigma$, but $\sigma$ does vary with
radius.

Some ellipticals do have some rotation, mostly in lower luminosity systems.
The relative importance of \emph{organized} motion over \emph{random} motion
can be characterized by $v_{rot}/\sigma$.  Shapes are expected to be influenced
by rotation: for an oblate model with isotropic velocity distribution that is
flattened by rotation:
\[
    \frac{v_{rot}}{\sigma} = \sqrt{\frac{\epsilon}{\left(1-\epsilon\right)}}
\]
Giant ellipticals have less rotation than this, implying anisotropic velocity
dispersions to account for their shape (required for triaxial systems in any
case).  Low/medium luminosity (high SB) Es, however, \emph{may} be isotropic
with flattening caused by rotation\footnote{de Zeeuw, figure 3}.  Low SB Es
appear to have anisotropic $\sigma$, i.e.\ more eccentric than expected from
rotation; ``measured in LG Es 185 (factor of three low in $v_{rot}/\sigma$,
147, factor of 10 low)." \mynotes{seriously, wtf}.  Significant fraction of Es
may have dynamical subcomponents, e.g.\ velocity and dispersion for some
interesting cases.

\subsection{Relations between different parameters and different families}
\subsubsection{Photometric parameters}
SB-size relation, between SB and SB-luminosity relations
(``Kormendy relations'')\mynotes{???}.
Also, profile \emph{shape} (e.g. Sersic index) with luminosity.

\subsubsection{Kinematics vs.\ luminosity}
More luminous galaxies have less rotation, but higher velocity dispersions
(\textbf{Faber-Jackson relation}, which is roughly $L\sim\sigma^{4}$, but
lots of scatter).

\subsubsection{The Fundamental Plane}
Faber-Jackson relations and the Kormendy relation (between SB and L/size) are
manifestations of the \textbf{Fundamental Plane of elliptical galaxies}; there
is a correlation between residuals in the Faber-Jackson emphrelation and SB.
Galaxies do not populate the entire 3D space of $I$, $r$, and $\sigma$, but
instead populate only a \emph{plane} in this space.  There is a relation
between the \textbf{three fundamental global observables}: SB (or luminosity),
size, and velocity dispersion\footnote{original refs: Dressler et al. 1987;
Djorgovski and Davis 1987; Bender, Burstein and Faber 199x}.  The observed
relation given by Virgo ellipticals is:
\[
    r_{c} \propto \left(\sigma_{0}^{2}\right)^{0.7}I_{c}^{-0.85}
\]
What is the origin of relation? Three assumptions would have to be
true:
\begin{enumerate}
    \item Es are in virial equilibrium
    \item M/L varies systematically with luminosity
    \item Es form a ``homologous" family, all, e.g.\ with deVaucouleurs profiles.
\end{enumerate}
In this case, one expects:
\[
    L = c_{1}I_{c}r_{c}^{2}
\]\[
    M = c_{2}\frac{\sigma^{2}_{o}r_{e}}{G}
\]
where the first is a definition and the second is related to the \textbf{virial
theorem} (though this applies to the mean potential energy and mean kinetic
energy per unit mass, averaged over the entire system). The constants are
related to the shapes and other details (not necessarily constant among the
variety of Es).

Combining these, we get
\[
    r_e = \frac{c_{2}}{c_{1}}\left(\frac{M}{L}\right)^{-1}\sigma_{o}^{2}I_{c}^{-1}
\]
If $M/L \propto L^{~0.2}$, then the observed fundamental plane is recovered;
could arise from stellar populations or from variations in baryon to total
mass. Alternatively, if $M/L$ is constant with a structure that varies relative
to one or more of the fundamental variables (which we know it does, e.g.,
systematic variations of Sersic $n$ with luminosity, but not clear if this is
entire ``explanation'').

However, we still need to understand origin of assumptions; why should
parameters, e.g.\ mass-to-light vary smoothly with luminosity?  Recall that
this ratio includes dark matter.  If ellipticals have dark matter halos, the
luminous inner parts are not required to be in virial equilibrium.

There is relatively little scatter around the fundamental plane, implying that
the assumptions are reasonably valid over a large range of elliptical
properties, which implies some significant regularities in the galaxy formation
process.

Galaxies do not fully populate the entire plane defined by our relation.
Consequently,when the plane is projected onto the other two axes, there is a
correlation\footnote{Djorgovski figure 2}.

\paragraph{Luminosity (size) --- $\sigma$ plane:}
Luminosity is proportional to velocity dispersion as
$L\propto\sigma^{4}$, which is known as the \textbf{Faber-Jackson relation}.
However, since the locus of Es isn't perfectly linear and the plane defied by
the ellipticals isn't perpendicular to this dimension, the scatter around the
Faber-Jackson relation is larger than the scatter around the fundamental plane.
A new radius can be defined that incorporates SB such that the new radius vs.\
$\sigma$ views the fundamental plane edge-on. Such a size measurement is called
$D_{n}$, the isophotal \emph{diameter} of the B=20.75 isophote. This
$D_{N}-\sigma$ relation provides a very useful distance estimator (if the
fundamental plane really is fundamental).

\paragraph{SB --- size plane:}
Relation for \emph{smaller} galaxies:
\begin{itemize}
    \item \emph{higher} SB for \emph{normal} small Es
    \item \emph{lower} SB for \emph{diffuse} small Es
\end{itemize}
These are sometimes known as the \textbf{Kormendy relations} and are one of the
main bases for separating these two types of objects.

\paragraph{SB --- $\sigma$ plane:} Presumably related to
two underlying physical parameters:
\begin{enumerate}
    \item density
    \item virial temperature
\end{enumerate}
In this plane, one can only form galaxies where \emph{cooling is effective},
i.e.\ at larger densities and hotter temperatures. This restricts
the area in the space in which we can find galaxies.

Additional features of the galaxy formation process may introduce additional
restrictions into allowed locations of galaxies on the fundamental plane. Most
luminous ellipticals are located along one line (with some scatter) in the
fundamental plane, and most diffuse ellipticals are located along another.

\subsubsection{Isophotal deviations vs.\ luminosity and kinematics}
Deviations from elliptical shape is correlated with dynamics, such that slower
rotators are more likely to be ``boxy'', and faster rotators are more likely to
be ``disky''\footnote{Kormendy and Bender, figure 2}.

There appears to be two
kinds of ``boxiness'', one found in giant Es and one found in bulges, which in
fact are relatively rapid rotators.  It is likely that these come from
different origins: plausibly, mergers in the case of giant Es and bars/disk
evolution in the case of bulges.

The core properties are correlated with the global shapes and dynamics:
cuspy core galaxies are boxy and anisotropic, cuspy galaxies are
disky and rotating.

\subsubsection{Several types of ellipticals??}
Normal and low luminosity Es:
\begin{itemize}
    \item significant rotation
    \item nearly isotropic
    \item oblate spheroids
    \item cusps (no cores)
    \item disky isophotes
\end{itemize}
Giant Es:
\begin{itemize}
    \item non-rotating
    \item anisotropic (triaxial)
    \item less flattened
    \item cuspy cores
    \item boxy isophotes
\end{itemize}
Dwarf spheroidals (diffuse ellipticals?):
\begin{itemize}
    \item lower SB
    \item off the fundamental plane
    \item anisotropic (slow rotators for their ellipticity)
\end{itemize}
Bulges: multiple types? As an aside: globular clusters, which are
likely to be significantly different.

\subsection{Spectral energy distributions}
Note typical features in ellipticals: stellar absorption lines
(4000\AA{} break, Mg, Fe, etc.)
Es are generally red: implications are predominantly old population.
Relations:
\begin{itemize}
    \item Color-luminosity: more luminous Es are redder. However, as wel'll
        see, variations of color at this level can come from either metallicity
        or age, so interpretation isn't trivial.
    \item Mg line strenth-luminosity: more luminous Es have stronger lines.
        This correlation is even tighter when considering relation between
        central velocity dispersion and line strength (The Mg-$\sigma$ relation),
        over large range of $\sigma$. As we'll see, this variation could
        arise from variations in age or metallicity, or different heavy element
        abundance ratios. Of equal interest to the Mg-$\sigma$ relation is the
        quite small scatter around the relation.
  \end{itemize}
Some Es have signatures of a younger population: E+A (or k+a) galaxies:
galaxies with strong Balmer lines. These are relatively rare, but likely
are indicative of post-starburst galaxies with significant star formation
1 Gyr ago. These are not associated with a cluster environment. Some show
morphological signs of recent interactions.

Stellar populations within ellipticals: Es have gradients in color/line
strength, i.e.\ tend to be redder toward the center (but again, there is
some degeneracy between age and metallicity as possible reasons for this).

\subsection{Interstellar matter in ellipticals}
It is now known that Es have significant interstellar gas, seen not in
optical, but in X-ray emission $\rightarrow$ hot gas.
This makes sense since some gas is expected from stellar evolution,
although there is also the possibility of gas from the environment.

Some Es aso show evidence of colder gas and dust.  A surprisingly large
fraction ($\sim$ 50\%) show some evidence for dust their cores\footnote{Lauer
et al. 2005} These are typically small components by mass.

\section{Spiral/Disk galaxies}
\subsection{SB profiles}
\subsection{Size-L/SB-L relations}
\subsection{Vertical distributions in disk galaxies}
\subsection{Non-axisymmetric features in disk galaxies: bars and spiral arms}
\subsection{Kinematics}
spirals are kinematically cold, meaning the
random motion of stars is small compared to the organized motion
(rotation), though there is some dispersion of velocities.

\subsubsection{Tully-Fisher relation}:
kinematics(maximum rotational velocity)-luminosity relation.
\subsection{Typical spectra}
\subsection{Gas and dust}

\section{Galaxy parameter (scaling) relations}
\begin{description}[labelindent=0.2in]
    \item [Structural] luminosity-SB
        \begin{itemize}
            \item isophotal shape-luminosity relation for Es
                (boxy vs. disky).
        \end{itemize}
    \item [Kinematic] Faber-Jackson, Tully-Fisher
        \begin{itemize}
            \item ellipticals luminosity-rotation relation.
                \mynotes{The Faber-Jackson relation also applies to
                ellipticals, but this is the relation between luminosity
                and velocity dispersion, not rotational velocity\ldots?}
        \end{itemize}
    \item [Structural/kinematic] Fundamental Plane
        \begin{itemize}
            \item A global generalization to all galaxies: the fundamental
                manifold\footnote{Zaritsky et al. (2008)}, with an
                assumption about how to combine rotation and velocity
                dispersion to get a characteristic ``velocity'', all galaxies
                appear to lie in a plane.
        \end{itemize}
    \item [Stellar populations] color-magnitude, Mg-$\sigma$ for Es
        (stronger for more luminous Es)
    \item [Gas] HI-SB, L-metallicity
    \item [Black holes] M$_{\mathrm{BH}}$-$\sigma$
        \begin{itemize}
            \item Central black hold seems to be ubiquitious in galaxies
            \item BH masses can be estimated (with difficulty; see later).
                More luminous (spheroid) galaxies appear to have more
                massive black holes.
            \item Tighter relation exists with velocity
                dispersion:M$_{\mathrm{BH}}$-$\sigma$ relation
                (or M$_{\mathrm{spheroid}}$)\footnote{Kormendy and
                Bender, 2001}. Note that black hole mass seems to be
                more linked to spheroid than to total galaxy mass
                (although claims to the contrary have been made).
        \end{itemize}
\end{description}


\section{Environments of galaxies: clusters and cluster galaxies}
    \begin{itemize}
      \item Galaxies are not homogeneously distributed in space...
      Correlation function
      \item Groups and Clusters
      \item Abell cluster
      \item Galaxies in galaxy clusters (Boselli and Gaazzi, PASP 118,
      517, 2006)
      \item Distributed hot gas in clusters, with mass comparable to
      that found in galaxies (one individual galaxy?)
      \begin{itemize}
        \item X-ray observations
        \item temperatures are 10$^7$ - 10$^8$ K
        \item intracluster gas is enriched in heavy elements
        \item X-ray observations useful for probing cluster masses,
        under assumption of hydrostatic equilibrium.
        \begin{itemize}
          \item Need measurement of density and temperature profiles
          of X-ray gas
          \item Estimates can be made from X-ray luminosity and/or
          temperature (better): the $M - kT$ relation
          \item typical masses are 10$^{14}$ - 10$^{15} M_{\odot}$; a
          typical cluster mass function (Wen et al. 2010).
        \end{itemize}
      \end{itemize}
      \item Shapes of clusters, inhomogeneities: not all clusters are
      in equilibrium
    \end{itemize}

\section{Some galaxies to be familiar with}

\newpage
\begin{center}
    \fontsize{20}{22}\selectfont\textbf{Part II: The building blocks of galaxies}
\end{center}

\section{Stars and Stellar populations}
How to relate observables to intrinsic population characteristics:
\begin{itemize} % AA
    \item \textbf{Population characteristics:} distribution of masses, compositions
        and ages (star formation history).
    \item \textbf{Observe spectral energy distributions (or colors) of stars:}
        individual stars in very nearby galaxies, interated starlight
        for most galaxies.
\end{itemize}

\subsection{Review: stellar evolution}
\begin{itemize}
    \item Internal structure of stars: determined by mass, chemical
        composition, and age (Russell-Vogt theorem). Exclues
        non-spherical symmetric effects, e.g.\ rotation, magnetic
        fields, binarity, etc.
    \item Luminosity (radius) and effective temperature derived from
        equations of stellar structure: mass conservation, hydrostatic
        equilibrium, energy equation, energy transport, along
        with auxiliary relations: equation of state, opacity, nuclear
        reaction rates.
    \item Main stages of stellar evolution:
        \begin{itemize} % CC
            \item Hydrogen core burning: main sequence (MS)
            \item Hydrogen shell burning: giant branch (for lower mass
                stars)
            \item Helium core burning: horizontal branch, red clump,
                blue core helium burners. Note key transition around 2
                solar masses (depends on metallicity): shift to helium
                flash at lower masses.
            \item Helium shell burning: Asymtotic Giant Branch (AGB)
            \item Other nuclear burning for high mass stars
            \item White dwarf or supernova
        \end{itemize} % end CC
    \item Model tracks (evolution as a function of time for a given
      mass); for spherical symmetry, calculations are 1D.
    \item \textbf{Isochrones}: cross-section of poroperties at a
      fixed time across a range of masses. Some well-known groups that
      calculate evolutionary tracks/isochrones:
      \begin{itemize}
          \item Padova
          \item BASTI (Teramo)
          \item Dartmouth
          \item Yale-Yonsei
          \item Victoria-Regina
          \item Geneva
      \end{itemize}
    \item Uncertainties that lead to some differences between different
        calculated isochrones: convective overshoot, diffusion, convection,
        helium abundance, mass loss, etc. Generally, Uncertainties are larger
        for later stages of evolution. Additionally, there may be missing
        physics, e.g.\ rotation and magnetic fields, that would require a full
        3D treatment.
    \item Given effective temperature,
        surface gravity (from mass and radius), and composition,
        stellar atmospheres give observables: spectral energy
        distribution/colors. Some model atmospheres (need links!):
        \begin{itemize}
            \item {Kurucz}
            \item {MARCS}
        \end{itemize}
    \item Theoretical Color-Magnitude Diagrams (CMDs) $\rightarrow$
      observed; need distance and reddening/extinction
    \item Age effects (model isochrones) from Yi et al 2001.
    \item Metallicity effects: internal (opacity) and atmosphere
      (line blanketing) effect combine in the same direction to make
      more metal-rich populations redder: CMDs
      \begin{itemize}
          \item Metallicity terminology: often given as mass fractions of
              hydrogen (X), helium (Y), and heavier elements (Z).
          \item solar abundance: X=0.7, Y=0.28, Z=0.019 (roughly)
          \item also given by
              $$ \left[\frac{\textrm{Fe}}{\textrm{H}}\right] =
              \log\left[
              \frac{\left( {\textrm{Fe}}/{\textrm{H}} \right)}
              {\left({\textrm{Fe}}/{\textrm{H}}\right)_{\odot}}
              \right]$$
          \item Be aware that this is an oversimplification, as Z contains
              lots of different elements. More later.
      \end{itemize}
    \item Main sequence (MS)
        \begin{itemize}
            \item Very rough scaling relation between luminosity and mass:
                $ L \propto M^{3.5} $
            \item Main sequence shifts with metallicity: redder for
                higher metallicity
            \item Location also depends on helium abundance
        \end{itemize}
    \item Red Giant Branch (RGB)
        \begin{itemize}
            \item note that because of more rapid evolution after MS,
                RGB stars all have roughly the same mass.
            \item temperature of RGB depends on age/mass: younger and more
                massive stars are hotter. Temperature also depends on
                metallicity: latter is dominant effect in older populations
                (greater than 5 Gyr).
            \item At lower masses/larger ages, tip of RGB is close to
                constant bolometric luminosity regradless of age or
                metallicity. In observed plane, leads to roughly fixed
                tip luminosity \emph{if} observing at long wavelengths
                (e.g.\ I band): basis for Tip of Red Giant Brance (TRGB)
                distance indicator.
            \item RGB bump (RGBB from
                Bono et al 2001),
                location of which depends on mass/age and metallicity;
                arises when H burning shell crosses chemical discontinuity.
        \end{itemize}
    \item Horizontal Branch (HB), aka.\ Red Clump (RC) or more generally
        He core burning sequence
        \begin{itemize}
            \item High mass stars form blue He core-burning branch
                (also note red plume)
            \item Intermediate mass stars form red clump.
            \item Low mass stars form horizontal branch. Variable mass loss
                on RGB and at He flash gives a range of masses.
            \item Horizontal branch morphology depends on metallicity: more
                metal-poor populations have bluer HB.
            \item However, there is something else that also affects HB
                morphology, leading to the so-called second-parameter problem
                (e.g.\ {M3/M13}, from {Rey et al
                2001}). Possibilities: Age? He abundance? Heavy element
                abundances? Density? Rotation?
            \item RR Lyrae stars: in instability strip (caused by doubly
                ionized He, also includes Cepheids, delta Scuti stars, etc.)
                RR Lyrae stars are indicators of old metal-poor population.
                Periods of 0.5 days, but multiple groups (Oosterhoff classes)
                depending on stellar parameters$\ldots$
        \end{itemize}
    \item Asymtotic Giant Branch (AGB)
        \begin{itemize}
            \item For intermediate masses/ages, AGB is significantly more
                luminous than the TRGB, hence of potential critical importance
                to studies of integrated light.
            \item For lower masses (older populations), AGB tip comparable
                to TRGB and AGB asymtotically approaches RGB (hence its name).
        \end{itemize}
    \item Potential importance of binaries/interactions
        \begin{itemize}
            \item unresolved (but otherwise non-interacting) binaries:
                broaden sequences, depending on mass ratios: equal masses
                give the appearence of a star 0.75 mag brighter.
            \item interacting binary stars
            \item blue stragglers: possible stellar merger/interaction products?
                {M3 example} from Sandage 1953.
            \item supernovae type SNIa: arise from binaries, produce different
                heavy element abundances than core collapse SNe. (e?)
        \end{itemize}
    \item End stages of stellar evolution:
        WDs, neutron stars, black holes, and supernovae
        \begin{itemize}
            \item {projenitor-final mass relation}
                (from Binney and Merrifield)
            \item {White dwarf cooling sequence}
                (from {Hansen et al 2007}: potential for
                age dating).
            \item Supernovae: generate significant fraction of heavy elements
                (but not all). Significant energy input, thermal and mechanical.
        \end{itemize}
    \item Galactic globular clusters: cornerstone of understanding
      stellar evolution historically, as apparent examples of a
      ``simple stellar population (SSP)'', with all stars of the same
      age and abundance.
      \begin{itemize}
          \item However, it's not recognized that not all GCs are so simple.
          \item Some CMDs show clear evidence of multiple components, e.g.\
              {NGC2808}.
          \item Long history of evidence of abundance variations between
              different stars, typified by the Na-O anticorrelation;
              demonstrated through observations of main sequence stars that
              this is not a mixing effect.
          \item Two phenomena have recently been coupled, but not yet a clear
              understanding of how the multiple populations arise.
      \end{itemize}
  \end{itemize}

\subsection{More information in a CMD than is represented by isochrones}
  one can also consider relative numbers of stars at
      different locations: CMDs that incorporate number of stars are
      reffered to as Hess diagrams (e.g.\ {Fornax}
      from Battaglia et al., 2006).
      \begin{itemize}
          \item For a simple stellar population (SSP, a population with
              a single age and metllicity), the relative number of stars
              at each stage is determined by the initial mass function (IMF)
              and the age.
          \item IMF determinations and parameterizations:
              \begin{itemize}
                  \item ``classical'' Salpeter IMF (power law with
                      $ dN/dM \propto M^{-2.35}$) and {others}
                      (from {Pagel}). Note power law form:
                      $ dN/dM \propto M^{\alpha}  $,
                      or alternatively,
                      $ dN/d\log{M} \propto M^{\Gamma} \propto M^{\alpha + 1} $
                  \item Widely used determination of local IMF is by
                      {Droupa Tout, and Gilmore 1993}, who find
                      $$ dN/dM \propto M^{-2.7}\ \textrm{for}\ M > 1M_{\odot} $$
                      $$ dN/dM \propto M^{-2.2}\ \textrm{for}\ 0.5 < M < 1M_{\odot} $$
                      $$ dN/dM \propto M^{-1.3}\ \textrm{for}\ M < 0.5M_{\odot} $$
                  \item Chabrier IMF: log-normal form,
                      $ dN/dM \propto \exp(\log{M}-\log{M_o})^2  $
                  \item Empirically, no strong evidence for variations of the
                    IMF as functions of, e.g., metallicity.
                  \item No well-established theory for predicting the IMF
              \end{itemize}
      \end{itemize}

\subsection{Star Formation Histories (SFHs) from resolved stellar populations}
      \begin{itemize}
          \item In general, galaxies are not SSPs, i.e.\ their CMDs don't look
              like those of a cluster. Can fit Hess diagrams of resolved stellar
              populations with combinations of SSPs to derive constraints on SFH,
              e.g.\ {simulated galaxies} (from
              {Tolstoy et al}).
          \item Generally speaking, we want the star formation history:
              $SFH(t,Z,M)$ gives the number of stars (or stellar mass) at all
              combinations of age, metallicity, and mass; because of lack of
              observed IMF variation, usually IMF is separated out:
              $$ SFH(t,Z,M) = \xi(M)\psi(t,Z) $$
              where $\xi$ is the IMF.
          \item MW neighbors resolved down to oldest MS turnoff ($M \sim 5$),
              because typical distances give distance modulii $\lesssim$ 20.
          \item M31 and neighbors somewhat shallower ($m-M\sim$ 24.5)
              without large investment of telescope time.
          \item Results for MW:
              \begin{itemize}
                  \item {solar neighborhood Hipparcos sample}:
                      roughly constant {SFH}, but be aware
                      of dynamical effects and limited volume bias nearby
                      sample to younger populations. When corrected for, star
                      formation history in solar neighborhood likely to have
                      significantly declined (see, e.g.\
                      {Aumer \& Binney 2009}).
                  \item Age of oldest stars can be studied by WD sequence,
                      gives $t_{\textrm{oldest}} > 8$ Gyr. Also note presence
                      of disk RR Lyrae stars ($t > 10$ Gyr??)
                  \item Might consider studying age distribution of clusters,
                      as they are simpler populations, but note problems with
                      disruption of clusters that would lead to a bias to
                      younger clusters.
                  \item Bulge can also be studied, but note foreground population
                      and extinction issues. Predominantly old population,
                      e.g.\ {bulge CMD}, and
                      {luminosity function} (from {Ortolani
                      et al 1995}).
                  \item Halo (as defined kinematically/spatially) also
                      prdominantly old, both from field population and from
                      globular cluster population.
                  \item Historically, distinction between disk stars/open
                      clusters (population I) and halo stars/globular clusters
                      (population II), with pop I being younger and more metal
                      rich. Note, however, the pop II association with low
                      metallicity is now recognized not to be fundamental;
                      inner halo/bulge significantly more metal rich.
                  \item For metallicities, note that there is a significant
                      puzzle in that there is \emph{not} a strong age-metallicity
                      relation in the solar neighborhood (e.g.\ Orion has
                      roughly solar metallicity, but is nearly 5 billion years
                      younger): radial migration, inhomogeneous ISM, inflow
                      $\ldots$?
              \end{itemize}
          \item Results for Local Group (LG) dwarf galaxies
              \begin{itemize}
                  \item Carina dSph: striking evidence of episodic star
                      formation, but this is NOT characteristic.
                  \item Others (from Tolstoy et al); note range of SFHs.
                  \item LG dIrrs compilation (from Dolphin et al 2005)
                  \item LG dSphs (from Dolphin et al 2005); possible similarity
                      to dIrrs apart from lack of recent SF?
                  \item Many dwarfs show population gradients, e.g.\ from
                      RGB/HB morphology (e.g., Harbeck et al 2001), likely
                      metallicity gradients: some show evidence of age gradients,
                      with younger population in center.
              \end{itemize}
          \item Results for M33 (from Holtzman et al 2011); note significant age
              gradient, and significant component of younger stars.
          \item M31 halo (from Brown et al 2006); significantly different from
              MW bulge, with more of an intermediate-old population?
          \item Can do more distant galaxies if information from advanced stages
              (RC, HB) is reliable; (e.g.\ local volume SFH (from Williams et al,
              in prep), dwarf SFH (from Weisz et al 2011)).
          \item Even farther if info from AGB is reliable.
          \item General conclusions:
              \begin{itemize}
                  \item galaxies are not SSPs.
                  \item galaxies have gradients in their stellar populations
                  \item significant population of old stars in nearby galaxies?
                  \item more massive galaxies formed stars earlier??
              \end{itemize}
      \end{itemize}

\subsection{Integrated light from \emph{un}resolved stellar populations}
      \begin{itemize}
        \item What stars contribute the most light?
            \begin{itemize}
                \item Along the MS, use combination of M-L relation and IMF:
                    $$ L \propto (M^{-2.35}M^{3.5}) \propto M^{1.15} $$
                    Massive stars dominate light.
                \item Compare MSTO with evolved population: RGB has almost
                    same mass as MSTO, but significantly more luminous: relative
                    contributions depend on relative number of starsalong the
                    giant branch compared with MSTO stars, but one finds that
                    the luminous evolved populations dominate (Renzini fig 1.5).
                \item Since post-MS evolution is fast compared to MS evolution
                    for all masses, it is true that at any given time when the
                    most luminous stars are the most evolved stars, the luminosity
                    is given predominantly by stars of (nearly) a single mass
                    (true for all except youngest ages, see Renzini 1.1) and
                    luminosity vs.\ lifetime plots.
            \end{itemize}
        \item Integrated brightness: do SSPs get brighter or fainter as they age?
            \begin{itemize}
                \item At younger ages, have MS ``peel-off'' effect, but also
                    supergiants! (WTF$\ldots$)
                % ------------------page 6-----------------------------------%
                \item At older ages, have competing effects of IMF and rate of
                    evolution.
                    \begin{itemize}
                        \item To determine the luminosity of a stellar
                            population, consider the number of stars evolving
                            off the MS, or equivalently, dying, which is given
                            by the \emph{evolutionary flux}:
                            $$ b(t) = \psi(M_{\textrm{TO}})|\dot{M}|
                            \textrm{\ [stars\ yr}^{-1}]$$
                            where TO stands for turnoff, $\psi$ is the IMF,
                            and $|\dot{M}|$ is the time derivative of the
                            turnoff mass, e.g.\ Renzini 1.1. The key point is
                            that lower mass stars evolve more slowly.
                        \item The IMF is important, along with age, in
                            determining the luminosity evolution of a galaxy.
                        \item Once one reaches an age where most massive stars
                            have died, all stars of lower mass reach comparable
                            luminosity (tip of RGB), hence luminosity of
                            population depends on number of red giants.
                        \item For a typical IMF number of stars increases
                            slower toward lower mass than rate of generation
                            of turnoff stars decreases.
                        \item Luminosity of the turnoff also has some smaller
                            effect.
                    \end{itemize}
                \item Do integrated colors depend on the IMF?
                    \begin{itemize}
                        \item Need to know the relative contributions of each
                            stage of evolution, i.e., from stars of different
                            masses.
                        \item Number of stars in each post-MS stage is
                            determined predominatnly by the time spend in each
                            stage:
                            $$ N_j = b(t)t_j $$
                            because all post-MS stars have nearly the same mass.
                            So although the total flux is sensitive to the IMF,
                            the relative contributions of each stage are not,
                            at least for older populations (Renzini fig 1.5).
                        \item For all except the youngest population, the later
                            stages of evolution provide a majority of the light.
                            Evolved stages are nearly all the same mass.
                        \item Consequently, the integrated spectrum and the
                            relative contributions of various evolutionary
                            stages are nearly independent of the IMF\@. They
                            \emph{do} depend on the age and metallicity.
                    \end{itemize}
                \item Dependence of color and luminosity on time (from
                    Bruzual and Charlot 199?) for a SSP (single burst)
                    normalized to one solar mass.
                    \begin{itemize}
                        \item luminosity variation is a combination of IMF plus
                            rate of evolution for lower masses.
                        \item color evolution comes from changing mix of stellar
                            population, but this is just for a single (solar)
                            metallicity.
                        \item note RSG phase, then subsequent dimming and
                            reddening.
                        \item note this is bandpass dependent, with less dimming
                            (at older ages where the RGB is dominant) at longer
                            wavelengths.
                    \end{itemize}
                \item Implication: if range of ages is present, integrated light
                    is (significantly) weighted toward younger populations.
                % ------------------page 7-----------------------------------%
            \end{itemize}
        \item What stars contribute the most \emph{mass}?
            \begin{itemize}
                \item For all measured
                \item relative stellar M/L ratios
            \end{itemize}
        \item Can you estimate stellar mass from integrated light?
            \begin{itemize}
                \item Consider the
                \item For individual stars
                \item Absolute value of
                \item stellar M/L ratio depends on
                \item So, it is possible
                    \begin{itemize}
                        \item Variety of different
                        \item Some uncertainties
                        \item Larger
                        \item if IMF were to be variable
                    \end{itemize}
            \end{itemize}
        \item Star formation histories from integrated colors
            \begin{itemize}
                \item Issue: intgrated colors are affected by combination of age
                    distribution,
% ----------------------------------page 8-----------------------------------%
                    metallicty distribution, and reddening distribution.
                \item color variations with age:
                \item situation somewhat improved
                \item Even if age-metallicity
            \end{itemize}
        \item Given age-metallicty degeneracy from colors,
        \item Spectral evolution (BC fig 4) for a variety of SFRs
        \item Dependence of contributor on wavelength
        \item Potential problems with synthetic integrated spectra:
        \item What about spectral features?
            \begin{itemize}
                \item For populations
                \item For older populations
% ----------------------------------page 9-----------------------------------%
                \item typical optical spectra
            \end{itemize}
        \item Even with
        \item Main area of application
% ---------------------------------page 10-----------------------------------%
      \end{itemize}

\subsection{Chemical evolution}
Understanding chemical evolution provides another clue about
the formation history of galaxies. Since stars generate this evolution,
one can in principle learn something about star formation histories by
looking at the distribution of compositions. The more handle we have on
metallicites, the better we can contstrain ages. The \emph{distribution}
of metallicities within galaxies (recall gradients) potentially provide
important information about mode of galaxy formation. We'd like self-
consistency between SFH  and chemical abundances. Potential information
about gas entering and leaving system. Relative abundances of different
elements also may provide clues about formation history.

The basic picture of chemical evolution starts with some initial conditions,
then add SFH: (IMF + SFR). This results in chemical enrichment for stars as a
function of mass and metallicity. Also consider input and output of gas from
any particular region, with either primordial or modified composition.
\begin{itemize}
    \item ICs: No heavy elements and pure gas (no stars).
    \item SFH: birthrate function $\psi(m,t)$ gives number of stars formed with
        mass $m$ per unit volume, usually split into a separable function:
        \[
            B(m,t) = \xi(M)\psi(t)
        \]
        where $\xi(M)$ is the IMF and $\psi(t)$ is the SFR\@.  This implicitly
        assumes an IMF which is constant in time, which may or may not be true,
        but so far, there is no strong observational evidence against it
        (although note pop III issues).
\end{itemize}
For any particular star, use stellar evolution to compute the amount and
composition of mass returned to the ISM vs.\ amount locked up in stellar
remnants (Pagel fig 2, from Maeder 1992).  Note that the \emph{rate} of return
depends on stellar mass, especially for the case of SnIa, which also depends on
the binary fraction.

\textbf{\underline{nucleosynthesis}}: main element groups and their sources:
\begin{itemize}
    \item light elements: Big Bang Nucleosynthesis (BBN)
        plus subsequent destruction (?)
    \item $\alpha$ elements (even Z from O up\ldots???): massive stars (core
        collapse SN)
    \item Fe-peak elements: type Ia SN and core collapse SN
    \item $s$ and $r$ process (neutron capture, plux beta decay) elements: core
        collapse SN, AGB stars\ldots (need neutron capture diagram here)
\end{itemize} % end AAA
Note that infall might arise from primordial clouds, or from processed
material, e.g.\ mass-loss from halo stars. Outflow might come from SN winds,
and in this case, it's possible that the composition of outflow material might
be more enriched than the typical composition at any given time. So things can
get complicated.

\paragraph{Basic equations of chemical evolution:}
Allowing for all sorts of realistic effects makes it difficult to make
very simple predictions for the evolution of abundances. However, under
some simplifying assumptions, it is possible to do so. A simple chemical
evolution model serves as a useful baseline against which observations
can be compared to determine where the assumptions may break down.
In addition, understanding a simple model allows us to introduce and
become aquainted with terminology that is in widespread use.

Consider a total (galaxy) mass $M$, split into a gas mass
$g$ and stellar mass $s$. For a simple model, consider the
gas at any time to be well mixed. At any given time, we wish
to know the fraction abundance ($Z_{i}$) of element $i$.
Consider an inflow rate into the system, $F$ and outflow
(ejection) rate, $E$. Then we have:
\[
    M = g + s
\]\[
    \frac{\mathrm{d}M}{\mathrm{d}t} = F - E
\]\[
    \frac{\mathrm{d}g}{\mathrm{d}t} = F - E + e - \psi't
\]
where $\psi'$ is the SFR in units of mass per time,
\[
    \psi' = \psi\int\xi\textrm{d}m
\]
% ---------------------------------page 12-----------------------------------%
(so $\xi$ is normalized to a total of one), and $e$ is the
ejection rate of mass from stars,
\[
    e = \int_{m_{t}}^{m_{U}}{
        \left(m-m_{\mathrm{rem}}\right)\psi\left(t-\tau(m)\right)
        \xi\left(m\right)\mathrm{d}m}
\]
which is a sum over all stellar masses of the product of the SFR at the time of
formation of each mass with the mass returned to the ISM, weighted by the
IMF\@. The lower limit of integration is at the stellar mass which is dying at
time $t$; lower mass stars don't contribute because they haven't ejected any
mass yet.  We also have
\[
    \frac{\mathrm{d}s}{\mathrm{d}t} = \psi' - e
\]
wince the mass in stars increases by te number of stars formed,
but decreases by the amount of mass lost back to the ISM from
the previous generation of stars. For the elements, we have
\[
    \frac{\mathrm{d}(gZ_{i})}{\mathrm{d}t} =
    e_{Z}(i) - Z_{i}\psi + Z_{F}F - Z_{E}E
\]
($Z_i$ is the mass fraction of element $i$; hereafter we'll drop
the subscript for simplicity). The mass in each element
\emph{increases} by:
\begin{itemize}
    \item amount of mass released by previous generations
    \item amount of mass added by inflow
\end{itemize}
but \emph{decreases} by:
\begin{itemize}
    \item amount of mass locked up in new stars
    \item amount of mass lost to outflow
\end{itemize}
The term $e_{Z}$ is given by:
\[
    e_{Z} = \int_{m_{\tau}}^{m_{U}}{
        \left[(m-m_{rem})Z(t-\tau(m))+mq_{Z}\right]
        \psi(t-\tau(m))\xi(m)\textrm{d}m}
\]
where $q_{Z}$ represents the fractional mass of element $Z$ synthesized and
ejected during stellar evolution (so left term in brackets gives material which
returns unprocessed and right term gives newly synthesized contribution).  In a
simple model, the synthesized masses are independent of the metallicity of the
population (although we know this is not true for some elements, more on this
later).

To simplify, consider an approximation to these formulae called
the \textit{instantaneous recycling approximation} which assumes
that all elements are returned instantaneously - good for products
of massive stars, but less good for products from lower mass stars,
e.g.\ iron. Then we have
\[
    e = \psi\int{(m-m_{\textrm{rem}})\xi(m)\textrm{d}m}
    \equiv (1-\alpha)\psi
\]
where $\alpha$ is the \textit{lock-up fraction}: the fraction
of mass ``locked up'' in stars and remnants.
\[
    \alpha = 1-\int_{m_{\tau}}^{m_{U}}{{m'}\xi(m')\mathrm{d}{m'}} +
    \int_{m_{\tau}}^{m_{U}}{m_{\mathrm{rem}}(m)\xi(m)\mathrm{d}m}
\]
where the second term is the total mass in stars that have died, and
the third term is the mass from those stars that have been locked up
in remnants.

Alternatively, one can think of the \textit{Return fraction},
$R=1-\alpha$. This gives us
\[
    \frac{\mathrm{d}g}{\mathrm{d}t} = F-E+(1-\alpha)\psi-\psi=F-E-\alpha\psi
\]
We also have
\[
\]
We define the stellar \textit{yield}, $p_{Z}$ to be
\[
\]
so the yield gives the fraction of the remnant population synthesized
and released in each element. This gives
\[
\]
where the RHS terms are recycled material, new production, lock-up in
new stars, inflow, and outflow. Simplifying, we have
\[
\]
Using mass locked up in stars, $s$, as the independent variable
\[
\]
\[
\]
so we get
\[
\]
\[
\]
If we make a further assumption of a uniform wind, $Z_{E}=Z$, then
\[
\]
and finally,
\[
    g\frac{\mathrm{d}Z}{\mathrm{d}s} =
    \frac{\mathrm{d}(gZ)}{\mathrm{d}s} - Z\frac{\mathrm{d}g}{\mathrm{d}s} =
    p + (Z_{F} - Z)\frac{F}{\alpha\psi}
\]
\textcolor{bred}{This is a basic equation of chemical synthesis}.
The \emph{simplest} model for chemical evolution assumes no
inflow or outflow, a homogeneous system without any spatial
differentiation of metallicity, zero initial metallicity, and
yields which are independent of composition. This is known as the
\textit{Simple, one-zone model}.

In the Simple model, we have
\[
    \frac{\mathrm{d}g}{\mathrm{d}t} = -\frac{\mathrm{d}s}{\mathrm{d}t}
\]\[
    g\frac{\mathrm{d}Z}{\mathrm{d}s} = -g\frac{\mathrm{d}Z}{\mathrm{d}g} = p
\]
% ---------------------------------page 15-----------------------------------%
Solving for $Z(g)$, we get
\[
\]
where $\mu$ is the gas fraction
\[
\]
The average abundance in stars is
\[
\]
which tends to $p$ as $\mu$ becomes small, i.e.\ the gas fraction becomes
small. For larger gas fractions, the mean $<Z>$ will be lower than the yield.
Note that this is the mean metallicity weighted by stellar mass, not by
stellar luminosity or by number (but these could easily be derived).

\subsection{Measuring stellar abundances}
Stellar abundances in individual stars measured from absorption features.
Hot stars only allow measurements of very few elements, while cool stars
(T\textless3500 K) are complicated due to very large numbers of lines and
prominence of molecular features. When lines are well-separated, abundance
determination is possible from measurements of the \textit{equivalent width}.
When lines are more crowded, usually approached by spectral synthesis.

In all cases, to measure abundances, need to know:
\begin{description}[
            labelindent=0in,
            labelwidth=2in
            style=multiline,
            itemsep=2ex,
        ]
    \item [atmospheric parameters]
        \begin{itemize}
            \item Temperatures ($T_{e}$) may be derived from colors
                (unless reddening is an issue); also often derived from
                excitation equilibrium: require lines of different
                excitation potential to give the same abundance (e.g. FeI lines)
            \item Surface gravity ($\log{g}$) can be derived from luminosities,
                temperatures and masses, hence distances are required.
                Alternatively, use ionization equilibrium: require lines
                of different ionization states to give the same abundance
                (e.g., FeI vs.\ FeII).
            \item microturbulence
        \end{itemize}
    \item [atomic parameters]
        \begin{itemize}
            \item Astrophysical $gf$ values: Atomic parameters are determined
                from lab experiments, but not always available at high quality.
                $gf$ values are derived from measurements of stars in which
                abundance is presumed to be known from other measurements.
        \end{itemize}
\end{description}
Abundances in unresolved populations are challenging, for the reasons
previously discussed (age-metallicity degeneracy, blending of different
features, etc.) Present day abundances in galaxies best measured through
gas abundances (to be discussed).

Units for measuring abundances:
\begin{itemize}
    \item Stellar abundances: usually measured relative to the sun
        $[X/Y]$:
        \[ \log\left[\frac{(X/Y)}{(X/Y)_{\odot}}\right] \]
    \item $Z_{i}$: mass fraction of element $i$
    \item $\log(X/Y)$: log of ratio of mass fractions
    \item $12 + \log(Z_{i}/H)$: log of ratio of mass fraction to hydrogen,
        with 12 added (to keep the numbers positive?)
\end{itemize}


\paragraph{Abundances as function of time}
The simple model predicts an increase of the metallicity with time.
The rate of increase depends on the SFR.
For example, if we \emph{assume} that the SFR is proportional to the
gas mass, $\mathrm{d}s/\mathrm{d}t = \omega{g}$, where $\omega = constant$,
we get:
\[
    z = \omega{t}
\]
and the abundance is expected to increase linearly with time.

\paragraph{Relative abundances of different elements}
Simple model predicts that, at any time, elements will be found in relative
abundances given by the ratio of their yields. This is true to some extent
for $\alpha$ elments, e.g.\ in the solar neighborhood\footnote{from
Bensby et al. 2003}. Most other elements go in lockstep with O (Fig. 11)

This is not true for Fe. This is plausibly explained for by a significant
fraction of Fe coming from SnIa, which has a time delay, and so the simple
model with instantaneous recycling doesn't hold. This is important to realize,
because many direct ``metallicity'' determinations are make from Fe liens.
If SnIa is the right explanation, one would expect to a constant ratio
of Fe/O for systems in which star formation proceeded rapidly (i.e.\ finished
before SnIa go off), but in systems with more extended start formation,
Fe/O would increase at higher O abundance (Page l figure 10).
Note that $\alpha/Fe$ traces timescales of extended star formation,
which is not necessarily the same as age (you could have a short epoch
of star formation at a late time to give alpha-enhancement even at younger
ages). However, very old populations would be expected to have alpha-enhanced
populations, e.g.\ as observed in globular clusters.
With the SnIa interpretation, $\alpha/Fe$ provides a clock. Unfortunately,
the calibration of the clock is not extremely well known because of the
unknown precise nature of SNIa projenitors. It is usually considered
to be of order 1 Gyr but there is very plausibly a range of SNIa ages.

The constancy of element abundance ratio holds for \emph{primary}
elements i.e. those that are produced from primordial abundance.
In addition some elements are secondary which means that some element must
previously exist in order for the secondary element to be created.
An example is N, which is produced during the CNO cycle
in greater abundance with a larger initial abundance of C.
This is observed (Fig. 12)

More generally, the constancy of element abundance ratios could fail if
the yields are themselves a function of metallicity. This may be
expected to occur, e.g.\ for elements that are significantly contributed by mass
loss, since mass loss may increase with increasing metallicity.

In reality, lots of different elements appear \emph{not} to go
in lockstep with one another, presumably because of the origin of different
elements from different processes and in different types of stars.
These variations may hold important clues for tracing the origin of stellar
systems. $\alpha/Fe$ traces duration of star formation (relative to
SnIa timescale). Other element ratios may trace variations in the IMF,
since different mass stars contribute different fractions of heavy
elements, or stochastic effects from a given IMF for star formation occurring
in small parcels (e.g.\ clusters). Gives rise to the idea of
\textit{chemical tagging} \mynotes{???}. Example: Local Group
dwarf galaxies and the Galactic halo\footnote{from Tolstoy et al. 2009}.
Could the halo come to be built up by dwarfs? Caution: just because
\emph{present-day} dwarfs don't match doesn't mean previously existing
ones don't. Note SDSS-III APOGEE survey.

Mass-metallicity relation here? \url{http://arxiv.org/abs/1211.3418}

\section{Gas}
Intrinsic gas characteristics:
\begin{itemize}
    \item temperature
    \item density
    \item composition
    \item ionization field
    \item total gas mass
\end{itemize}
Gas in galaxies exists in multiple phases:
\begin{center}
\begin{tabular}{l l}
    phase & temperature\\
    \hline
    molecular clouds & $\lesssim$ 100 K\\
    cold, neutral (atmic) medium (CNM) & $\lesssim$ 100 K\\
    warm, neutral (atmic) medium (WNM) & $\lesssim$ 8000 K\\
    warm, ionized medium (WIM/DIG) & $\sim$ 8000 K\\
    hot, ionized medium (HIM) & $> 10^{5}$ K\\
\end{tabular}
\end{center}
Also denser ionized gas in HII regions, SN remnants, PNe.
CNM/WNM/WIM/HIM perhaps roughly in pressure equilibrium, but note
other support mechanisms such as turbulence, magnetic fields, etc.
If they are in pressure equilibrium, higher temperature components
have correspondingly lower densities.

\subsection{Observing and characterizing the gas}
Two general methods:
\paragraph{Emission}
Atomic lines
\begin{itemize}
    \item Permitted: In low temperature gas (T = 10000 K $\sim$ 1eV),
        the energy difference to excited states is typically too high
        for excitations to occur from collisions. An exception is
        H lines from recombination; these exist from UV through radio
        wavelengths.
    \item ``Forbidden'': Low transition probabilites; important in low
        density gas because collisions are less frequent, allowing time
        for radiative transitions to occur before collisional de-excitation.
        Wavelengths are typically in the optical and UV regimes. Note that
        many of these are doublets because of multiple angular momentum
        levels. Most prominent in optical: [OIII], [NII], [SII], [OII], [OI]
    \item Fine-structure lines, transitions ivolving spin alignment with
        orbital alignment. Typically FIR wavelengths
    \item Hyperfine-structure lines, transitions involving electron spin
        alignment with nuclear spin. Radio wavelengths.
\end{itemize}
\subparagraph{Molecular lines}
\begin{itemize}
    \item Vibrational transitions, typically IR wavelengths
    \item Rotational transitions, typically mm wavelengths. However,
        rotational transitions only strong for non-symmetric molecules
        that have a dipole moment (not H$_{2}$, although it can have
        some IR features if shock heated). CO emission is most prominent.
\end{itemize}
Emission in many cases is proportional to $\rho^{2}$. If the excitation
mechanism is collisional (i.e., requires two particles), the observational
quantity is the \textit{emission measure}: $\propto\int{n_{e}^{2}\mathrm{d}\ell}$.

\subsubsection{Absorption}
\paragraph{Atomic lines}
\begin{itemize}
    \item Typically from ground state, e.g.\ H Lyman transitions,
        Na 5890, CaII
    \item Most prominent in optical
\end{itemize}
\paragraph{Molecular lines}
\begin{itemize}
    \item Mostly UV, e.g.\ H$_{2}$ absorption
\end{itemize}
Absorption is generally more sensitive, but it does require a bright
background source, e.g.\ UV light from quasars. Absorption is
linearly proportional to density. The observational quantity is
\textit{column density}: $\propto\int{n\mathrm{d}\ell} $

\subsection{Heating and cooling} What determines temperature?
\subsubsection{Heating}
\begin{itemize}
    \item Radiative heating (both via ionization and heating through dust)
    \item Mechanical heating (shocks, jets, etc.)
    \item Gravitational heating
    \item Cosmic rays
\end{itemize}
\subsubsection{Cooling}
\begin{itemize}
    \item Bremsstrahlung (free-free)
    \item Line emission from heavier elements
    \item Molecules
    \item Cooling curves and contributions (Sutherland \& Dopita 1983)
\end{itemize}

\subsection{HI measurements: 21 cm line}
\begin{itemize}
    \item Traces neutral medium
    \item From spin-flip transition: very rare (10 Myr) but H is very
        numerous. Relative number in two energy states set by collisional
        equilibrium and proportional to statistical weights of two levels
        (3:1); upper levels are populated from (microwave background photons,
        collisional saturation?)
    \item HI flux at a given location gives column density (atoms/surface
        area):
        \[
        n_{HI} [\mathrm{cm}^{-2}] \approx 1.82\times10^{18}\int{
            \left[\frac{T_{b}(\nu)}{\mathrm{K}}\right]\mathrm{d}
            \left[\frac{v}{\mathrm{km\;s}^{-1}}\right]}
        \]
        Typical column densities in galaxies $\sim 10^{20}\;\mathrm{cm}^{-2}$
    \item Total HI flux (integrated over entire galaxy), along with distance,
        gives total HI mass:
        \[
            \frac{M_{HI}}{M_{\odot}} = 2.36\times10^{5}\left(
            \frac{D}{\mathrm{Mpc}}\right)^{2}F_{HI}
        \]
        for $F_{HI}$ in Jy km s$^{-1}$.
    \item 21-cm single dish observations have low spatial resolution.
        Higher resolution possibly with interferometric observations
        (e.g.\ VLA) but information on smooth distribution/total mass can
        be lost.
    \item HI can be observed to very low column densities
        ($\sim10^{13}\;\mathrm{cm}^{2}$) in absorption, but only for
        the Lyman series. Primary tool for studying the IGM:
        Lyman alpha clouds, epoch of reionization, quasar absorption
        line systems\ldots
    \item Typical column densities are\ldots
        Total mass in HI can be a significant fraction of total stellar
        mass, increasingly important in lower luminosity galaxies.
    \item Temperature might be inferred from width of line, but need to
        be careful about turbulence, cloud structure, etc. Estimate can
        also be made by comparing emission flux with absorption if you
        have a background source (radio galaxy).
\end{itemize}

\subsection{Molecular gas}
\subsection{Warm ionized gas}
\subsection{Hot gas}
If ``very'' hot, detectable in X rays: thermal Bremstrahlung (free-free)
cutoff frequency decreases with increasing temperature. At lower temps
(T $\sim$ 10$^{5}$ K), very hard to detect. Soft X rays are absorbed
by neutral hydrogen. Detectalbe in absorption by high ionized metals,
e.g.\ OVI, CIV, NV. Very hard to determine total mass, but mass may be
very significant, just from cosmological baryon fraction \mynotes{???}
\subsection{Denser ionized gas}
Easiest gas to observe in galaxies.
\subsection{Chemical evolution and abundances from gas}
\subsection{Distribution of gas within spirals}
Neutral gas primarily observed in blue sequence galaxies.
HI distribution usually extends beyond optical distribution of starlight,
while molecular is more centrally concentrated. In some galaxies,
CO traces H$\alpha$ well, but not in others. Hot gas is observed in
most galaxies.

\subsection{Gas as a kinematic tracer}
Measuring galaxy rotation - both optically and using HI (both spatially
resolved, e.g.\ VLA, and unresolved data).
\begin{itemize}
    \item Swaters et al. 2002
    \item Velocity-coded color (from NRAO)
    \item Spider diagrams
    \item PV diagrams: note spread of velocity at any given position;
        location of peak velocity may not give the \emph{maximum}
        velocity at that position
\end{itemize}
Rotation is often characterized by $v_{max}$: maximum of rotation curve
(sometimes hard to measure).  From unresolved HI profiles, rotation is
characterized by line widths, e.g.\ $W_{50}, W_{20}$ (width at 50\% and 20\% of
the peak line flux, respectively).

\subsection{Interaction between stars and gas}
\subsubsection{supernovae and mass loss from stars}


\newpage
\section{Dust}
\subsection{General importance of dust}
\begin{itemize}
    \item Can obscure light from stars/gas: affects inferences on
        populations, abundances, etc.
    \item Emits radiation which can be observed (mostly IR) to probue
        obscured emission.
    \item Important for ehemistry of ISM, e.g.\ created of H$_{2}$.
    \item Important for heating and cooling of ISM
    \item Probably \emph{not} important as a significant mass component;
        likely only a fraction of a percent of ISM by mass
\end{itemize}

\subsection{Extinction}
Foreground and internal. Extinction laws and their variation.

Extinction of light is characterized by an \textbf{extinction curve}, which gives
the relative amount of light lost as a function of wavelength. Note that
extinction curves are often in magnitudes (logarithmic). Generally,
shorter wavelengths are attenuated more, leading to the phenomenon of
reddening. This implies that the dust grain sizes are comparable to the
wavelength of the light. \textcolor{bred}{Existence of the 2200\AA{} bump.}
The total amount of extinction is proportional to the amount of reddening,
e.g.\ $R_{V} \equiv A_{V}/E(B-V)$.
More generally, $R_{\lambda} \equiv A_{\lambda}/E(B-V)$.

Variations in the extinction curve exist, both within the Milky Way
and in nearby galaxies\footnote{Gordon et al. 2003}. This variation is
more significant in the UV. Cardelli, Clayton, and Mathis suggest that
extinction curves can be parameterized by a single parameter $R_{V}$.
Probably related to dust properties, e.g.\ grain size.

Extinction has both scattering and absorption components. Scattering
leads to, e.g., reflection nebula and diffuse light. Measuring scattering
suggests optical albedo $\sim$ 0.5, so roughly equal amounts of scattering
and absorption. Absorption leads to heating and re-emission.

Discussions of extinction curves refer to the extinction of background light by
foreground dust (i.e. a foreground screen). Integrated galaxy light likely does
not fall into this scenario (apart from foreground Milky Way extinction).
Remember Burstein \& Heiles and Sclegel, Fink beiner \& Davis for MW maps.
Scattering is significant and leads to ``saturation''\footnote{Witt \&
Gordon, 2003}.

\subsubsection{Amount of gas, dust-to-gas ratio, and transparency of galaxies}
The amount of extinction seems to be well correlated with the amount of
neutral H
\[
    N(H) = 5.8\times10^{21}E(B-V)\quad
    \left[\mathrm{cm}^{-2}\;\mathrm{mag}^{-1}\right]
\]\[
    A_{V} = 5.3\times10^{-22}N_{H}\quad
    \left[\mathrm{mag}\;\mathrm{cm}^{2}\right]
\]
This suggests that dense (inner) regions of ISM are likely to be optically
thick, but not low density ones. Confirmed by observations of ``overlapping''
galaxies.

\subsection{Emission from dust}
\begin{itemize}
    \item Predominant heating is from starlight (some collisional
        heating in dense clouds)
    \item Amount of energy absorbed depends on flux, size, and albedo.
        Amount of energy radiated depends on temperature, size, and emissivity.
    \item Dust particles are not ideal blackbodies. In addition, they may not
        be in temperature equilibrium, especially small particles.
    \item MW dust
    \item Dust emits in far IR, e.g. MW emission and models. Note sharper
        emission features from PAHs around 10 microns.
    \item Dust emission traces star formation; luminous IR galaxies
        (LIRGs, ULIRGs, 10 to 100x IR emission than typical galaxy).
    \item Implications for higher redshift: sub-mm observations and
        the negative K-correction
\end{itemize}

\subsection{Composition}
\begin{itemize}
    \item Solids have broad or non-existent absorption features, so abundances
        from absorption are difficult or impossible.
    \item Some inferences from interstellar gas abundances:
        if stellar abundance ratios are assumed, than observed gas abundances
        show depletion (from Draine), e.g. of C, Mg, Si, Fe at a level
        greater than 50\%.
    \item Still can't identify uniquely what molecules the dust is in.
        Possibilities:
        \begin{itemize}
            \item silicates
            \item oxides
            \item carbon solids
            \item hydrocarbons, e.g. PAHs (polycyclic aromatic hydrocarbons)
            \item carbides
            \item metallic Fe
        \end{itemize}
    \item Some absorption features are observed:
        \begin{itemize}
            \item 2200\AA{} feature: originally thought to be graphite,
                now perhaps PAHs
            \item 9.7 and 18 microns: silicates? (from Draine)
            \item diffuese interstellar bands (DIBs) (from Draine)
        \end{itemize}
    \item As stated above, emission features are observed in IR from PAHs
\end{itemize}
Overall, plausible to expect that the \textbf{dust-to-gas ratio might be
a function of metallicity}.

\subsection{Creation/destruction}
Dust is apparently difficult to create in the ISM. It is thought to be
created in envelopes of cool stars $\rightarrow$ \textbf{the amount of dust
may also be a function of star formation.}

\newpage
\section{Central black holes}
(some references: Peterson, ``An Introduction to Active Galactic Nuclei'',
Ferrarese \& Ford (2005)).

Galaxies harbor central supermassive black holes: how do we know?
\subsection{AGN as indicators of central black holes}
Some fraction nearby galaxies who ``active'' nuclei:
\begin{itemize}
    \item Optical emission lines
    \item Power-law continua spanning most of the EM spectrum.
    \item Radio sources, often with jets
    \item AGN review\footnote{see book by Peterson, 1997}\footnote{see
        Whittle (UVa) lecture notes}
        \begin{itemize}
            \item Seyfert galaxies: Optical emission lines. Types 1
                (broad lines) and 2 (narrow lines). Distinguishable from
                HII regions, etc. from \textbf{emission line ratios}.
            \item LINERS: weaker lines, lower ionization
            \item BL Lac objects
            \item Quasars and QSOs
            \item Radio galaxies, e.g.\ Cen A, M87. Emission is from
                synchrotron emission. Energy source may be the same,
                but don't generally directly see nuclear activity.
                Fanaroff-Riley (FR) types 1 (low radio luminosity,
                edge-darkened) and 2 (high radio luminosity, edge-brightened).
            \item See tables in section 4b of Whittle notes with some spectra.
        \end{itemize}
    \item Frequency of Seyfert phenomenon 1-2\% but LINERS much more common;
        perhaps tens of percents of local galaxies show activity at some
        level.
\end{itemize}
Generally considered to be powered by accretion onto a nuclear black hole.
Arguments for the existence of black holes:
\begin{itemize}
    \item Luminosity: can be as high as $\sim10^{45-58}$ erg s$^{-1}$
        for some AGN. \textbf{Eddington Luminosity:}
        \[
            L_{e} = \frac{4{\pi}GMm_{p}c}{\sigma_{T}} =
            1.3\times10^{46}\frac{M}{10^{8}M_{\odot}}
            \left[\mathrm{erg\;s}^{-1}\right]
        \]
    \item Non-stellar continua
    \item Timing arguments. AGN show rapid variability with timescales of
        days (optical/radio) to hours (X-ray): implies small objects
        (small fraction of a pc).
\end{itemize}
\textbf{Overall physical picture:}
different active nucleii phenomena all arise from
a common physical scenario. Some \emph{intrinsic} differences give rise
to different phenomena, whereas some differences arise from viewing angle.
\begin{itemize}
    \item Unified model: several main regions:
        \begin{itemize}
            \item Accretion disk
            \item Broad line region (BLR)
            \item Narrow line regions (NLR)
            \item Dust torus
        \end{itemize}
    \item Alternative model\footnote{Elvis (2000)}
    \item Viewing angle differences: Unified model (and another)
    \item Some possible intrinsic causes of variation between types
        \begin{itemize}
            \item Accretion rates and accretion rate differences
            \item Differences from variations in gas suppy (quantity and/or
                efficiency of infall)
            \item Differences from different black hole masses?
        \end{itemize}
\end{itemize}
\subsection{Black holes in non-active galaxies}
It appears that most galaxies, even inactive ones, harbor central
supermassive black holes. This is expected at some level from number density
of quasars at higher redshift.
Detection of SMBHs\footnote{Ferrarese \& Ford}\footnote{Whittle (UVa) lecture}:
\begin{itemize}
    \item \textbf{Sphere of influence} (region where velocity generated by
        central mass is comparable to or larger than velocity determined
        from overall galaxy potential) is very small:
        \[
            r_{h} \sim \frac{GM_{BH}}{\sigma^{2}} \sim
            11.2\frac{M_{BH}}{10^{8}M_{\odot}}\left(
            \frac{\sigma}{200\mathrm{km\;s}^{-1}}\right)^{-2}\quad
            \left[\mathrm{pc}\right]
            \]
    \item MW galactic center (e.g. UCLA group video)
    \item Other galaxies: BHs inferred but strictly speaking, not absolutley
        required\footnote{Ferrarese \& Ford 2005}.
\end{itemize}
If they have BHs, why aren't they active? Gas supply? Advection dominated
accretion flow (ADAF)?

\subsection{Measuring black hole masses}
\begin{itemize}
    \item Proper motions
    \item Stellar and gas kinematics
    \item Gas kinematics
    \item Broad Fe K$\alpha$ (6.4 KeV X-ray line)
    \item Reverberation mapping
\end{itemize}
BH masses have now been made for several dozen galaxies. Initially, correlation
seen between BH mass and galaxy luminosity. Relation between BH mass and
central velocity dispersion significantly tighter. Origin not well understood.
M$_{BH}-\sigma$ relation (or M$_{spheroid}$).

\subsection{Importance of central black holes}

\newpage
\section{Galaxy spectral energy distributions}

\newpage
\section{Dark matter and galaxy masses}
Dark matter dominates mass in the universe. How much is there? How is
it distributed, both among and within galaxies? To what extent does it
regulate galaxy properties?

Note from cosmology: we expect
\begin{itemize}
    \item $\Omega_{m} \sim 0.27$
    \item $\Omega_{b} \sim 0.047$
\end{itemize}

\subsection{Observations of dark matter in galaxies}
The amount of dark matter in a galaxy is often characterized by the
mass-to-light ratio (M/L). \textbf{Note that there is a distinction
between \emph{stellar} M/L and \emph{total} M/L.}

\subsubsection{Galaxy rotation curves}
Rotation curves (RCs) probe dark matter content and profile. The general
observation of flat profiles (sometimes reffered to as ``isothermal'' profiles)
implies that $M \propto r$ and $\rho \propto r^{-2}$.

Recall that the shape of the rotation curve is moderately well-correlated
with the luminosity, in the sense that one gets steeper RCs for lower L
galaxies\footnote{Persic and Salucci, figure 4}. However there is significant
variation at each luminosity. Note that not all RCs are flat; there is a
variety of shapes.

Comparison of the luminosity profile with the rotation profile shows that
the \emph{luminous} mass alone is unable to produce the rotation profile,
therefore dark matter is required. The fraction of ``dark mass'' increases
with radius, and also increases with fixed radius with decreasing luminosity.

\paragraph{Dark matter in disk galaxies}
Given the luminosity profile, the shape of the RC shows that DM has to be
present, because there's no way to get high enough stellar mass-to-light ratios
in the outer region.  However, determining the amount of dark matter in disk
galaxies requires knowledge of the M/L of the stellar population, which is
unknown (recall that M/L depends on the SFH, especially the IMF).  The stellar
M/L can't be derived from the RC because the density distribution of the DM is
unknown.  It's possible that the presence of dust makes this more challenging
to infer.

A common assumption is the \textbf{``maximum-disk hypothesis''}, in which
the innermost parts of the RC are assumed to be driven by the luminous mass
alone. These regions then give a M/L, which is assumed to be constant as
a function of radius. Combined with the outer roation curve, mass profiles
for the luminous and dark components are derived. However, data is often
degenerate in different ratios of disk to DM\footnote{see van der Kruit
example, from van der Kruit Galaxy Masses 2009 talk}.
The applicability of this hypothesis is debateable. For example,
for an exponential disk:
\[
    V_{max} \propto \sqrt{\frac{M}{r_{d}}}
    \]
so we might expect a trend at fixed L (if we extrapolate from M to L)
with scale length: not observed in Tully-Fisher relation.

The \emph{total} M/L in disk galaxies
has been found to be $\sim$ 10-50 in regions probed by RCs.

\paragraph{Dark matter in LSB galaxies}
The RCs for LSB galaxies confirm a high M/L. Compare two galaxies at
identical positions in TF relation and see that the LSB galaxy is much
more DM dominated\footnote{from de Blok \& McGaugh}. This suggests that
RCs are correlated not only with luminosity, but also with surface
brightness.

LSB RCs are of particular interest because they may provide the most
direct probes of DM distribution. The shapes of their RCs are currently under
active debate as to what they imply for DM distribution.
Numerical DM simulations predict a ``cuspy'' distribution of DM,
with $\rho_{DM} \propto r^{-1}$ in the inner regions, e.g.\ the
Navarro-Frenk-White (NFW) profile:
\[
    \rho = \frac{\rho_{0}}{\frac{r}{r_{s}}\left(1+\frac{r}{r_{s}}\right)^{2}}
    \]
family of profiles characterized by the concentration $r_{200}/r_{s}$.

However, there may also be an effect of baryons on DM (adiabatic compression,
or expansion?), since in inner regions, gravity from baryons may not be
negligible. Some observations\footnote{Swaters et al 2003} suggest more of
a core. The interpretation of relatively low velocities in LSB centers can
be complicated by a variety of effects, such as projection, beam smearing,
non-circular velocities, etc. The ``core vs.\ cusp'' debate is still active.

DM halos are expected to extend far beyond what we can see from physical
tracers (even HI gas) within the galaxy embedded in their centers.
Typical virial radii (from simulations) are expected to be several hundred
kpc. So total masses are not well probed by rotation curves.

\paragraph{Dark matter in ellipticals}
Theoretically, the presnece of DM in ellipticals is exptected. However, it
is harder to observe than in spirals because of the lack of an ``easy''
dynamical tracer, like a rotating component. THey also have higher SB,
so there is a larger contribution by baryones to kinematics\footnote{see
review by Gerhard 2006}.

For non-rotating components, the velocity dispersion profiel is insufficient
to measure mases without knowledge of \textit{velosity ellipsoid},
parameterized by \textit{velocity anisotropy}:
\[
    \beta \equiv 1 - \frac{<v_{\theta}^{2}>}{<v_{r}^{2}>}
    \]
Or more generally, distribution of orbits\footnote{e.g.\ velocity dispersion
profiles, from van der Marel (1994)}. Inclination/projection also plays a
role\ldots

Potential tracers:
\begin{itemize}
    \item Velocity distribution of stars (integrated light). Need to use
        line profile shapes, 2D distributions (integral field)
    \item Planetary nebulae
    \item Globular clusters
    \item Gas disks (rare, but do exist)
\end{itemize}
Note that even for non-rotating systems, distribution of mass often described
by the \textit{circular velocity}:
\[
    v_{c} \equiv \sqrt{\frac{GM(r)}{r}}
    \]
From stellar velocity distributions, most ellipticals show evidence of
flat circular velocity curves, indicating the presence of DM in the inner
regions, but not always a lot.

Modelling suggests that a simple estimate\footnote{Cappellari et al. (2006)}
\[
    M/L \sim 5\frac{\sigma_{e}^{2}r_{e}}{GL}
    \]
gives a reasonable rough estimate; c.f virial relation \mynotes{(???)}
and relation for
central velocity dispersion, e.g.\ for isothermal sphere.
There has been some debate about PN observations at larger radii, with some
observations gsuggesting declining circular velocities. However, understanding
of velocity anisotropy is important to the conclusion.

Indpendent mass estimate proveded by X-ray halos under the sasumption of
hydrostatic equilibrium (requires measuring densities and temperatures),
which, when they exist, generally indicate the presence of dark matter
halos\footnote{from Sato et al. 2000}
\[
    M(r) = -\frac{k_{B}T(r)r}{{\mu}m_{p}G}
    \left[\frac{\mathrm{d}\ln{\rho}}{\mathrm{d}\ln{r}} +
    \frac{\mathrm{d}\ln{T}}{\mathrm{d}\ln{r}}\right]
    \]
Bottom line: ellipcals probably have dark matter.

\subsubsection{Gravitational lensing}
Another way of probing mass. There is a distinction between \textit{strong}
and \textit{weak} lensing\footnote{both from Treu talk at Galaxy Masses 2009}.
Weak lensing is measured statistically.

\subsection{Total masses of galaxies}
\subsection{Halo abundance matching}

\end{document}
