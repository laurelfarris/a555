\documentclass{article}
%\usepackage[left=1in, right=1in, top=1in, bottom=1in]{geometry}
\renewcommand\familydefault{\sfdefault}
\usepackage[margin=1.5in]{geometry}
\setlength{\parindent}{0em}

\usepackage{hyperref}
\usepackage{framed}
\usepackage{amsmath}
\usepackage{amssymb}
\usepackage{marvosym}
\usepackage{graphicx}
\usepackage{tcolorbox}
\usepackage{lipsum}
\usepackage{enumitem}
\usepackage{xcolor}
\usepackage{ragged2e}
%\usepackage{mathptmx}
\usepackage{mathpazo}

\usepackage{fancyhdr}
\pagestyle{headings}  % Put section name in top left, page number in top right
%\pagestyle{fancy}
%\fancyhf{}  % Clear all headers and footers (including default page number).
%\setlength{\headheight}{15pt}
%\renewcommand{\headrulewidth}{0pt} % remove the header rule
%\lhead{text} % Top left
%\rhead{text} % Top right
%\cfoot{text} % Top center
\rfoot{\thepage} % Page numbers on bottom right corner

\usepackage[symbol]{footmisc}
\usepackage{perpage}
\MakePerPage{footnote}
%\renewcommand{\footnoterule}{\kern-3pt\hrule width\textwidth height 0.4pt\kern 2pt}

\definecolor{bred}{rgb}{0.8, 0.0, 0.0}
\definecolor{cadmiumgreen}{rgb}{0.0, 0.42, 0.24}
\definecolor{mypur}{rgb}{0.4, 0.22, 0.33}
\definecolor{myblue}{rgb}{0.0, 0.2, 0.6}
\definecolor{mygray}{rgb}{0.90, 0.90, 0.90}
\definecolor{mypurple}{rgb}{0.41, 0.16, 0.38}
\definecolor{pinegreen}{rgb}{0.0, 0.47, 0.44}

\definecolor{hl}{rgb}{0.61, 0.87, 1.0}
\newcommand{\test}[1]{%
    \begin{center}
        {\parbox{0.9\textwidth}{\emph{\centering #1}}}
        %\colorbox{hl}{\parbox{0.9\textwidth}{\emph{\centering #1}}}
    \end{center}}

\newcommand{\mynotes}[1]{\textcolor{cadet}{\textit{#1}}}
\usepackage{tikz}\usetikzlibrary{backgrounds, shapes.misc}
\newcommand*\circled[1]{\tikz[baseline=(char.base)]
            \node[draw=black,shape=rounded rectangle,draw,inner sep=10pt] (char) {#1};}

%\setcounter{secnumdepth}{1}
\usepackage{titlesec}
%\usepackage[compact]{titlesec}
%   [<shape>]{<format>}{<label>}{<sep>}{<before-code>}[<after-code>]
\titleformat{\section}%
  {\fontsize{16}{18}\selectfont\bfseries\color{myblue}}
  {\fontsize{46}{50}\selectfont\color{mypur}\arabic{section}\color{black}$\vert$}
  {0em}{}
\titleformat{\subsection}%
  {\fontsize{14}{16}\selectfont\bfseries\color{mypur}}
  {\color{myblue}\circled{\arabic{section}.\arabic{subsection}}}
  {0.5em}{}
  [\vspace{-2.5pt}{\color{mygray}\titlerule[5pt]}]
  %[\vspace{-20pt}\colorbox{mygray}{% \begin{minipage}{\textwidth}% %\vspace*{2pt}%Space before \hfill %\vspace*{2pt}%Space after \end{minipage}}]
\titleformat{\subsubsection}%
  {\fontsize{13}{14}\selectfont\bfseries\color{mypur}}
  {\color{myblue}\arabic{section}.\arabic{subsection}.\arabic{subsubsection}}
  {1em}{}
  %[\vspace{-2.5pt}{\color{mygray}\titlerule[3pt]}]
\titleformat{\paragraph}%
  {\fontsize{11}{12}\selectfont\bfseries\color{myblue}}
  {}
  {0.5em}{}
\titleformat{\subparagraph}%
  {\fontsize{11}{12}\selectfont\itshape\color{myblue}}
  {}
  {0.5em}{}

%\titlespacing*{⟨command⟩}{⟨left⟩}{⟨before-sep⟩}{⟨after-sep⟩}[⟨right-sep⟩]
\titlespacing*{\section}{-0.75in}{0ex}{0ex}
\titlespacing*{\subsection}{-0.25in}{0ex}{0ex}[-0.25in]
\titlespacing*{\subsubsection}{0pt}{2ex}{-1ex}
\titlespacing*{\paragraph}{0pt}{1ex}{-2ex}
\titlespacing*{\subparagraph}{0pt}{1ex}{-2ex}

% Lists
\setlist[itemize]{noitemsep, topsep=0ex}
\setlist[itemize, 1]{label=$\vcenter{\hbox{\footnotesize$\bullet$}}$}
\setlist[itemize, 2]{label=$\vcenter{\hbox{\footnotesize$\circ$}}$}
%\renewcommand{\labelitemi}{$\vcenter{\hbox{\tiny$\bullet$}}$}
%\renewcommand{\labelitemii}{$\vcenter{\hbox{\tiny$\circ$}}$}
\setlist[enumerate]{noitemsep}
\setlist[description]{noitemsep, labelindent=0.5in}
\definecolor{cadet}{rgb}{0.33, 0.41, 0.47}

\renewcommand{\descriptionlabel}[1]{%
    \bfseries\textcolor{cadet}{#1}
}
\usepackage{hyperref}
\hypersetup{
    colorlinks=true,
    urlcolor=myblue,
    linkbordercolor=myblue,
    linkcolor=black,}
%\urlstyle{same}
\newcommand{\myref}[1]{\textcolor{pinegreen}{\S{} \ref{#1}}}


%---------------------------------------------------------------------------%
\begin{document}
\setlength{\parskip}{0pt}
\tableofcontents
\newpage
\setlength{\parskip}{10pt}
\hypersetup{colorlinks=true, urlcolor=myblue, linkcolor=pinegreen,}

\section{Introduction}
Galaxies are defined as gravitationally bound collections of stars,
gas, dust, dark matter, and black holes.

Goals:
\begin{itemize}
    \item Understand general current picture of galaxies and galaxy evolution
    \item Basic understanding of how galaxies are observed
    \item Review statistical properties of the galaxy population, and specific properties of different types of galaxies
    \item Understand physical tools by which we learn and characterize different components of galaxies
    \item If time permits: what we can learn from the Milky Way
\end{itemize}

ASTR 555 intended to provide basic overview of galaxies and their components.
ASTR 616 to focus on galaxy formation and evolution.

\subsection{History}
\textcolor{red}{Incomplete!}
\begin{enumerate}
    \item 1700's: Messier objects, 39 of
        which are actually galaxies (total of 110 in the final catalogue).
    \item 1864: GC, 1888: NGC (William Herschel and son John)
    \item 1920's: `Great Debate' between Curtis and Shapley on whether
        or not galaxies were located within the MW\@. Resolved by Hubbles
        discovery of Cepheids in M31.
    \item 1980's: Importance of environment recognized:
        \textbf{morphology/density relation}; `Nature vs. Nurture'.
    \item 1990's: Techniques for finding and confirming high redshift
        galaxies (z>2): \textbf{Lyman break galaxies}:
        \begin{itemize}
            \item Lyman limit
            \item Rydberg formula: $\frac{1}{\lambda}=
                R\Big(\frac{1}{n_u^2}-\frac{1}{n_l^2}\Big) $
        \end{itemize}
        First large scale surveys, both nearby and at medium redshift. HST
        imaging of distant galaxies. N-body (dark matter) simulations.
    \item 2000's:Precision Cosmology and LCDM, outside optical
        wavelengths: IR (Spitzer) and sub-mm (JCMT).
    \item 2010's: Extended gas halos in galaxies. Possibly detection of
        DM and dark energy.
\end{enumerate}

\subsection{Approaches}
Multifaceted approach to studying galaxy formation and evolution:
\subsubsection{Nearby galaxies}
Galaxy ``archaeology'': study nearby galaxies in detail,
attempt to understand processes that led to their current
appearance.
\begin{description}[labelindent=2em, labelwidth=8em, leftmargin=12em]
    \item [Advantages] can resolve structure, individual stars
        in nearest galaxies, high S/N observations
    \item [Disadvantages] some information may be erased by physical processes
        (e.g.\ merging), degeneracies in integrated light
\end{description}
\subsubsection{Distant galaxies}
Look at galaxy samples at different lookback times, study distribution of
properties (galaxy population) as a function of time.
\begin{description}[labelindent=2em, labelwidth=8em, leftmargin=12em]
    \item [Advantages] direct probe of different stages.
        \href{http://astronomy.nmsu.edu/holtz/a555/html/diagrams/misc/tz_lcdm.htm}
        {Relationship between lookback time and redshift}.
    \item [Disadvantages] brightness/selection effects, lack of detail,
        difficulty in associating objects at one redshift to those at another.
\end{description}
\subsubsection{Physics of galaxy formation}
\begin{description}[labelindent=2em, labelwidth=8em, leftmargin=12em]
    \item [Advantages] some physics (e.g.\ gravity) is well understood
    \item [Disadvantages] some physics (e.g.\ star formation) is not well
        understood. Dynamic range of the problem is huge:
        \begin{itemize}
            \item Dynamic range in distances, from stellar scales
                to largest scale structure.
            \item Dynamic range in mass, from stellar scales
                (1 $M_{\odot}$) to
                largest scale structure ($10^6-10^{15} M_{\odot}$)
        \end{itemize}
\end{description}

\subsection{Overview of galaxies and galaxy formation}
Cosmological context of galaxy formation and evolution:
\begin{itemize}
    \item Baryonic matter (and some leptons), non-baryonic matter, dark energy
    \item \href{http://astronomy.nmsu.edu/holtz/a555/resources/Cosmological_Composition_1.png}
        {Current composition of the universe}
    \item \href{http://astronomy.nmsu.edu/holtz/a555/resources/Cosmological_Composition_2.png}
        {Past composition of the universe}
\end{itemize}

\subsubsection{Main components:}
\begin{description}
    \item [Dark Matter] (DM); usually non-baryonic, though even some baryonic
        matter can be hard to see, such as brown dwarfs. DM dominates mass of
        galaxies, and is currently detectable only by its gravitational effect
        on motions, or background light (lensing).
    \item [Stars] Observed properties depend primarily on mass, age,
        and composition. Variety leads to multiple luminosities and colors
        in galaxies.
    \item [gas] molecular, atomic, and ionized gas phases. Low density gas
        can be hard to detect.
    \item [dust] mass of ISM varies widely between galaxies. Observed via
        emisison and absorption.
    \item [Central (supermassive) black holes]
        Observed indirectly via gravitational effects, accretion
\end{description}

\subsubsection{Formation and evolution}
How do galaxies come to appear as they do?
Important processes (not necessarily in chronological order):
\begin{itemize}
    \item Gravitational collapse (of dark matter, and later, baryons)
        in cosmological framework.
        \begin{itemize}
            \item How big are initial lumps at different size scales?
            \item How much angular momentum?
            \item How fast do lumps grow?
        \end{itemize}
    \item Condensation of gas and cooling
        \begin{itemize}
            \item ``hot'' vs. ``cold'' accretion
        \end{itemize}
    \item Star Formation (not well understood)
        \begin{itemize}
            \item Under what conditions do stars form?
            \item What types (masses) of stars form?
            \item Drives chemical evolution, which may impact
                cooling and future star formation
        \end{itemize}
    \item Black hole formation
        \begin{itemize}
            \item Primordial formation vs.\ formation from early stars
            \item How common?
        \end{itemize}
    \item Feedback/mass loss
        \begin{itemize}
            \item How much energy? Mechanical or thermal?
                Does mass escape or just delay accretion?
            \item What objects generate it, and how? Possibilities:
                Winds, supernovae, galactic nucleii.
        \end{itemize}
    \item Continued accretion from IGM
        \begin{itemize}
            \item How much?
            \item What mode?
            \item What composition?
        \end{itemize}
    \item Merging:
        \begin{itemize}
            \item halo merging; dynamical friction
            \item galaxy merging: gas-rich (``wet'') vs.\ gas-poor (``dry'')
        \end{itemize}
    \item Other environmental processes:
        \begin{itemize}
            \item Cluster (group?) environment: ram pressure, tides
        \end{itemize}
    \item Dynamical evolution
        \begin{itemize}
            \item Dynamical instabilities; e.g.\ bars and spiral arms
            \item Migration
            \item Internal vs.\ external triggers
        \end{itemize}
\end{itemize}
These processes have \emph{characteristic timescales}, and the relation between
them may influence how galaxies form, evolve, and appear.\footnote{
    see a \href{http://astronomy.nmsu.edu/holtz/a555/resources/mofig1.1.gif}
    {schematic flowchart} and some of the
    \href{http://astronomy.nmsu.edu/holtz/a555/resources/mofig1.1.gif}
    {complicated links} from Mo et al.}

\subsubsection{Timescales}

\paragraph{Hubble time:}
\[
    t \propto H_{0}^{-1}
    \]
Cosmological densities:
critical density today is given by
\[
    \rho_{crit} = \frac{3H_{0}^{2}}{8\pi{G}}
    \approx 10^{-29}\;[\mathrm{g\;cm}^{-3}]
    \approx 5.5\times10^{-6}\;[\mathrm{cm}^{-3}]
    \]
$\rho_{crit}$ increases with redshift. ``Collapsed'' halo has
$\overline{\rho} \approx 200\rho_{crit}$.

\paragraph{Gravitational timescales}
Orbital time:
\[
    t_{orb} = \left( \frac{3\pi}{G\overline{\rho}} \right)^{1/2}
    \]
Free-fall time (time for unsupported cloud to collapse\ldots doesn't happen!):
\[
    t_{ff} = \left( \frac{3\pi}{32G\overline{\rho}} \right)
    \sim 2.1\times10^{9}\quad [f^{1/2}n_{-3}^{-1/2}\; \mathrm{yr}]
    \]
\begin{itemize}
    \item $n_{-3}$ = gas density [10$^{-3}$ cm$^{-3}$]
    \item $f$ = gas fraction of the cloud
\end{itemize}

\paragraph{Cooling timescale}
\[
    t_{cool} = \frac{\frac{3}{2} nkT}{n^{2}\Lambda(T)}
    = 3.3\times10^{9}\quad \left[
        \frac{T_{6}}{n_{-3}\Lambda_{-23}} \mathrm{yr}
        \right]
    \]
\begin{itemize}
    \item $\Lambda_{-23}$ =
        cooling curve in units of 10$^{-23}$ erg cm$^{-3}$ s$^{-1}$.
        $\Lambda_{-23}$ = 1 is roughly the minimum cooling rate for a
        primordial plasma at $T > 10^{4}$ K.
        \footnote{See Mo, van den Bosch, \& White for derivation/discussion of above).}
\end{itemize}
What determines gas temperature?
A gas cloud is heated during the process of gravitational collapse:
\[
    T_{vir} \approx 3.6\times10^{5}\quad \left[
        \mathrm{K} \left( \frac{v_{c}}{100\mathrm{km/s}} \right)^{2}
        \right]
    \]
\begin{itemize}
    \item $v_{c} = \sqrt{GM/R}$ characterizes the mass and radius of the halo.
\end{itemize}

\paragraph{Star-formation/gas consumption time}
\[
    t_{sf} \propto \frac{M_{gas}}{\dot{M}_{sf}}
    \]
Typical SF rates today are $\sim$ 1 M$_{\odot}$ yr$^{-1}$, but can be
much higher in ``starbursts''.

\paragraph{Chemical enrichment time}
Depends on element, but for many, massive star lifetime, $\sim 10^{7}$ yr.

\paragraph{Merging time}
Orbital timescales between halos.

\paragraph{Dynamical friction time}
Only important for massive subhalos.
\[
    t_{df} \frac{t_{hubble}}{10} M_{main}/M_{sat}\quad [\mathrm{yr}]
    \]
(Note error in Moe et al.).

\test{Know what the main components of galaxies are. Be able to list and qualitatively
describe the (many) different processes that may be important in galaxy formation.
Understand and be able to describe some of the different timescales that may be
important in galaxy formation.}

\subsection{Toy example}
Overly simple, but illustrates the consequences of this scenario.\footnote{
    Rees (1995)}
\begin{itemize}
    \item Individual galaxies appear to have a maximum size of order
        10$^{11}$ - 10$^{12} M_{\odot}$.
    \item Can be crudely explained by understanding of dissipation
    \item Self-gravitating cloud has two timescales:
        \begin{enumerate}
            \item dynamical, or free-fall:
                \[
                    t_{dyn} \sim (G\rho)^{-1/2}
                \]
            \item cooling time:
                \[
                    t_{cool} \sim \frac{nkT_{g}}{n^{2}\Lambda(T)}
                \]
                where $\Lambda$ is the cooling function
        \end{enumerate}
    \item If $t_{cool} > t_{dyn}$, then a cloud can be in quasi-static
        equilibrium, i.e.\ cooling is unimportant.
    \item  If $t_{dyn} > t_{cool}$,
        the cloud cools, kinetic energy is converted to radiation, and
        the cloud collapses.
\end{itemize}
Given a cooling curve for primordial composition, the relevant timescales can
be calculated, and show that collapse is unlikely to occur for $ M >
10^{12}M_{\odot}$. This implies that dissipation is important, at least for
objects we observe as galaxies (e.g.\ luminous objects).

This argument is only a suggestion for a number of reasons. One of these is
that halos do not actually have uniform density, so there's no such thing as a
single cooling time for the entire halo.

\subsection{Questions}
\begin{itemize}
    \item When do each of these steps happen and what are their
        relative importances?
    \item What sets the masses of galaxies? Sizes?
        ($10^{6} - 10^{12}M_{\odot}$) Luminosities?
    \item What sets the distributions of numbers of galaxies as a
        function of mass/luminosity?
    \item Does the ratio of baryonic mass/total mass change for
        different galaxies?
    \item What triggers star formation in galaxies?
    \item What is responsible for the rango of galaxy morphology?
    \item How much of present structure is determined by initial
        conditions, e.g.\ initial overdensity, angular momentum (and what
        are those initial conditions)?
    \item How much does present appearance depend on basic physics
        within galaxies, e.g.\ dynamics and chemical evolution?
    \item How much depends on environment, e.g.\ mergers and
        interactions, background radiation?
    \item Does the relative importance of these effects (initial
        conditions, internal evolution, environment) vary for different
        galaxies?
\end{itemize}
\newpage


\newpage
\section{Observing galaxies}
\subsection{Imaging}
\subsubsection{Surface brightness}
\textbf{Surface Brightness (SB)} is the most fundamental observable for
galaxies. It is the basic measured property for imaging of a \emph{resolved}
object. SB is independent of distance \emph{until geometry of the universe
becomes important}:
\[
    SB \propto (1+z)^{-4}
    \]
Units:
\begin{itemize}
    \item erg cm$^{-2}$ s$^{-1}$ sterradian$^{-1}$
    \item mag arcsec$^{-2}$
\end{itemize}
When summing SBs, flux units must be used (if you're adding and
subtracting magnitude units, you're doing something wrong \Smiley).
Differences in magnitudes correspond to \emph{ratios} in fluxes:
\[
    m_{1} - m_{2} = -2.5\log\frac{F_{1}}{F_{2}}
    \]
In general, flux is a function of wavelength, so SB is measured in different
bandpasses, e.g.\ UBVRI or SDSS (ugriz). The ratio of the flux of the same
object in different bandpasses gives an estimate of the \textbf{spectral slope}
between bandpasses; often expressed as a \emph{color index}, the difference in
magnitude between the two bandpasses, such as U-R.

Relevance of SB distribution: How are stars distributed in galaxies? There is
not a direct correlation between stellar luminosities and stellar masses, nor
is there one between stellar mass and mass in other components.

Typical SB values in galactic \emph{centers} are:
\begin{description}[align=right,labelwidth=5em]
    \item [Spiral] $V \sim 20 - 21$ mag arcsec$^{-2}$
        (not much brighter than typical dark sky)
    \item [Elliptical] $V \sim 16 - 17$ mag arcsec$^{-2}$
\end{description}
In principle, SB is measured directly from a 2D detector as an arbitrary
function of location. Galaxies are relatively faint; more than half the light
from galaxies comes from regions with $SB < SB_{\mathrm{sky}}$ so $S/N$ is
difficult to obtain. It's challenging to observe anything fainter than 24-25
mag arcsec$^{-2}$ in unresolved light (although note recent discovery of
``ultra-diffuse galaxies'', \href{https://arxiv.org/abs/1506.01712} {Subaru
(Koda et al.)}, \href{http://arxiv.org/abs/1410.8141} {Dragonfly (van Dokkum et
al)}). Can observe to significantly lower SB (30 mag
arcsec$^{-2}$) for nearby objects in which you can resolve stars, i.e. add up
their individual brightnesses.

\paragraph{Observational challenges:}
\begin{description}[align=right, labelindent=2em, labelwidth=5em]
    \item [seeing] affects SB distribution in center (see section on cores)
    \item [sky determination] affects SB distribution in outer regions
        (see \href{http://astronomy.nmsu.edu/holtz/a555/html/diagrams/a616/sky.htm}
        {here}).
\end{description}

Since SB is the key observable, there can be strong \textit{selection effects}
against low SB objects. If all galaxies had the same SB profile (they don't)
then low SB galaxies are \emph{strongly} biased against in either
magnitude-limited or size-limited catalogs (if mag/size is determined isophotally).
The apparent ``size'' of a galaxy will depend on its SB.

\textbf{Isophotes} are contours of constant SB.
They are often well (though not perfectly) represented by ellipses since
most galaxies are symmetric at a significant level.
Elliptical contours fit spirals and ellipticals for different reasons
\footnote{Some basics of techniques for ellipse fitting:
    Kent, ApJ 266, 562 (1983); Lauer, ApJ 311, 34}:
\begin{description}[align=right, labelwidth=5em]
    \item [Spiral] viewing angle and disk thickness
    \item [Elliptical] intrinsic ellipticity (for example,
        \href{http://astronomy.nmsu.edu/holtz/a555/resources/n1226.h.jpg}
        {NGC1226} and
        \href{http://astronomy.nmsu.edu/holtz/a555/resources/n1316.h.jpg}
        {NGC1316}).
\end{description}
Some basics of techniques for ellipse fitting:
\begin{itemize}
    \item Fourier moments
    \item Model fitting
\end{itemize}
In reality, galaxies have more complex features, e.g.\ asymmetries,
departures from elliptical isophotes, bars, spiral arms, jets, etc.\
(see below).

For a purely axisymmetric object, SB distribution reduces to a 1D
(major axis) SB profile.
SB profiles are often parameterized, with distributions in the form
of a
\href{http://astronomy.nmsu.edu/holtz/a555/html/diagrams/a616/sersic.htm}
{\textit{Sersic Profile}}\footnote{plot from
\href{http://adsabs.harvard.edu/abs/2003ApJ...582..689M}
{MacArthur et al. (2003)}}
(though there are other forms that can be used as well).
The Sersic Profile is given by
\[
    \Sigma(r) = \Sigma_{e}\exp \left( -b_{n} \left[ \left(
    \frac{r}{r_{e}} \right) ^{1/n}-1 \right] \right)
    \]
\begin{itemize}
    \item $\Sigma(r)$ = SB at radius $r$
    \item $r_{e}$ = \textit{effective radius} (aka.\ half-light radius),
        the radius enclosing half the
        total light if the model is extrapolated to infinity.
    \item $\Sigma_{e}$ = SB at $r_{e}$
        ($\Sigma_{0} \sim 2000\Sigma_{e}$\ldots \mynotes{where $\Sigma_{0}$ =
        SB at galaxy center?})
    \item $b_{n} \approx 2n - 0.324$ and is determined from the definition of
        $r_{e}$
    \item $n$ depends on the type of galaxy ($n=4$ for ellipticals and $n=1$
        for disks).
\end{itemize}

\paragraph{SB profile for disk galaxies}
Disks (and low luminosity ellipticals) are usually well represented \emph{on
average} by an exponential:
\[
    \Sigma(r) = \Sigma_{s}\exp\left(-\frac{r}{r_{s}}\right)
    \quad \left[ \mathrm{erg\; cm^{-2}\; s^{-1}\; sterradian^{-1}} \right]
    \]
\[
    m(r) = m_{s} + kr
    \quad \mathrm{[magnitudes]}
    \]

\paragraph{SB profile for spheroids}
Spheroids are historically characterized by
\textbf{deVaucouleurs profile}, also known as the ``$r^{1/4}$'' law:
\[
    \Sigma(r) = \Sigma_e\exp\left(-7.67\left[\left(
    \frac{r}{r_e}\right)^{1/4}-1\right] \right)
    \quad\mathrm{[flux\;units]}
    \]
\[
    m(r) = m_{0} + kr^{1/4}
    \quad\mathrm{[magnitudes]}
    \]
Many galaxies appear to have SB profiles that are well fit by a combination of multiple
components, e.g., a bulge and a disk.
The contributions from the different components can be split
via bulge-disk decomposition (e.g.
MacArthur, Courteau, \& Holtzman, ApJ 582, 689 (2003)). Issues:
\begin{itemize}
    \item covariance between parameters
    \item 1D vs. 2D
    \item validity of models
\end{itemize}

\subsubsection{Sizes of galaxies}
Generally, galaxies do not appear to have sharp edges (although SB can
drop sharply in some disk galaxies).
\begin{itemize}
    \item Isophotal ($D_{25}$ in RC is B=25.0 \mynotes{???}):
        Beware for low SB galaxies and for galaxies
        at non-negligible redshift\ldots\mynotes{?????}
    \item Half-light radii (e.g.\ $R_{e}$); requires extrapolation for
        total brightness
    \item Petrosian radius: radius at which the \emph{local}
        SB at that radius drops to some fraction of the \emph{mean} SB within
        that radius.  (in an annulus, SDSS used 0.8$r$ - 1.25$r$, with a
        fraction of 0.2)
\end{itemize}
Typical angular sizes:
\begin{itemize}
    \item Nearby galaxies: arcminutes
    \item Distant galaxies: arcseconds
\end{itemize}
Linear sizes can be derived from distance estimate
(which depends on cosmology). Typical physical size (of the luminous
component) is $\sim$ 1-30 kpc.

\subsubsection{Integrated brightness}\label{intb}
Issue: galaxies don't have edges. Types of mags:
\begin{description}[align=right, labelwidth=0.5in,
        labelindent=0.25in, leftmargin=1in]
    \item [Metric:] brightness within an aperture of fixed angular or physical
        size. Be careful if galaxies have a range of sizes.
    \item [Isophotal:] brightness within a specified SB contour, e.g.\ Holmberg
        mags, which is the brightness within $\mu_{pg} = 26.5$. \mynotes{wtf is
        $\mu_{pg}$??} Be careful if galaxies have a range of SB profiles, as
        sample will be strongly biased against lower SB object.
    \item [Model:] require assumption about outer SB profile
    \item [Petrosian:]
        mag within some fixed number of Petrosian
        radii (SDSS uses 2).
        Advantages with regard to SB (distance) and seeing effects. Note
        \href{http://astronomy.nmsu.edu/holtz/a555/images/petrogal.htm}
        {relation of Petrosian mag to model mag} depends on SB profile.
        \footnote{\url{http://classic.sdss.org/dr5/algorithms/photometry.html\#mag_petro}}
\end{description}
Typical integrated brightnesses of galaxies, and star-galaxy separation
and crossover in number counts at $V \sim 21$.
When determining integrated colors of a galaxy, be sure to use the same
region of integration for both bandpasses.
For example, use the petrosian radius as determined in the highest S/N bandpass.

Other issues in getting luminosities of galaxies:
\begin{itemize}
    \item Foreground extinction, which depends strongly on Galactic latitude
        \footnote{\href{http://adsabs.harvard.edu/cgi-bin/nph-bib_query?bibcode=1984ApJS...54...33B}
            {Burstein \& Heiles, ApJS 54,33 (1984)};
            \href{http://adsabs.harvard.edu/cgi-bin/nph-bib_query?bibcode=1998ApJ...500..525S}
            {Schlegel, Finkbeiner, \& Davis, ApJ 500,525 (1998)}}
    \item Internal extinction/inclination, which can be significant
        (up to 1.5 mag in RC3 prescription \mynotes{(???)})
    \item If looking at a population, luminosities need to be compared in
        the same bandpass. This is an issue if looking over a range of
        redshifts; a fixed observed wavelength range corresponds to a
        different rest frame wavelength range at different redshifts.
        \textbf{K-corrections} are used to account for this at some
        level\footnote{\href{http://arxiv.org/abs/astro-ph/0210394}
            {Hogg et al. astro-ph/0210394};
            \href{http://adsabs.harvard.edu/cgi-bin/nph-bib_query?bibcode=1996ApJ...467...38K}
            {Kinney et al. ApJ 467.38 (1996)};
            \href{http://adsabs.harvard.edu/cgi-bin/nph-bib_query?bibcode=1980ApJS...43..393C}
            {Coleman, Wu, Weedman, ApJS 43,393 (1980)}}.
        K-corrections are defined in magnitudes:
        \[
            m_{R} = M_{Q} + DM + K_{QR}
            \]
        where the object is observed in bandpass $R$, and you are interested in
        absolute magnitude in rest-frame bandpass $Q$ (which might be $R$,
        the classical K-correction).

        For significant redshift, measure in different bandpass, so that K
        correction is less sensitive to assumptions about spectrum
        (i.e., $Q$ not the same as $R$).

        If you ``know'' (make assumptions about) the spectrum of the object, you
        can do this:
        \href{http://adsabs.harvard.edu/cgi-bin/nph-bib_query?bibcode=1996ApJ...467...38K}
        {Kinney et al, ApJ 467, 38 (1996)},
        \href{http://adsabs.harvard.edu/cgi-bin/nph-bib_query?bibcode=1980ApJS...43..393C}
        {Coleman, Wu, Weedman, ApJS 43, 393 (1980)}.
        Often, multiple bandpasses (colors) are used to determine best matching
        template.
\end{itemize}

\subsection{Spectroscopy}
The other fundamental observable is the spectral energy distribution (SED),
or spectrum, of a galaxy. Spectra are often obtained over some aperture,
or long slit; however, note increasing use of integral field spectroscopy
(e.g.\ SDSS MaNGA).

Galaxy spectra are \textit{composite}:
\begin{itemize}
    \item Stars
        \begin{itemize}
            \item multiple velocities
            \item luminosity weighted: total number of
                stars (for each stellar type), times luminosity per star
        \end{itemize}
        Generally, individual stellar lines can't be isolated.
    \item Gas emission
    \item Dust absorption/emission
\end{itemize}

The overall shape and features in \href{http://astronomy.nmsu.edu/holtz/a555/resources/galaxyspectra.gif}
{SEDs} (see also \href{http://astronomy.nmsu.edu/nicole/teaching/ASTR505/lectures/quickview.html}
{here}) provide important information about the nature of the luminous
components, i.e.\ physical properties of the stellar population and the gas
component.

\subsubsection{Velocities}
Spectra allow the study of velocities via the Doppler shift:
\[
    \frac{\Delta\lambda}{\lambda} = \frac{v}{c}
    \]
which measures both the bulk velocity of the galaxy and the distribution of
stellar velocities within the galaxy. Note that the shift in wavelength is
proportional to the emitted wavelength; lines are shifted and broadened more at
longer wavelengths. As a result, it is common to work with spectra to that are
evenly sampled in $\log \lambda$ rather than $\lambda$, because  $\Delta
\log\lambda \propto \Delta v$.

\paragraph{Bulk velocities}
\begin{itemize}
    \item Mean velocities of galaxies indicate that they are receding, with
        velocity proportional to distance $\rightarrow$ expansion of the
        universe.
    \item There is a dispersion of velocities around the mean relation; the
        deviation is called the \textbf{peculiar velocity}.  In the field,
        typical peculiar velocities are a few hundred km s$^{-1}$.  In
        clusters, they can be a thousand km s$^{-1}$.
\end{itemize}

\paragraph{Internal velocities}
\href{http://astronomy.nmsu.edu/holtz/a555/resources/intvel.gif}
{Internal velocities} can be organized or random.
\begin{description}
    \item [Organized] (e.g.\ rotation in disk galaxies). Note effects of
        inclination, which need to be measured to get intrinsic rotation
        velocity ($v_{obs} = v\sin{i}$).
        Also note that populations in disk galaxies are not entirely
        kinematically ``cold"; there is some dispersion around the rotational
        velocity, typically at the level of 10s of km s$^{-1}$.
    \item [Random] broader lines that represent the line-of-sight
        velocity distribution (LOSVD) of stars within a galaxy.
        \mynotes{This broadening would be the same at all points in the galaxy.}
        \[
            F(\lambda) \propto \int{
                F (v_{los}) S \left( \lambda - \frac{v_{los}\lambda}{c} \right)
                \mathrm{d}v_{los}}
            \]
        \begin{itemize}
            \item $S$ = spectrum of an individual star
            \item $F$ = fractional distribution of stars at
                different LOS velocities.
        \end{itemize}
\end{description}

Lines are usually fairly well represented
by a Gaussian velocity profile, characterized by the mean velocity:
\[
    \overline{v} = \int{v_{los}F(v_{los})\mathrm{d}v_{los}}
    \]
and the velocity dispersion:
\[
    \sigma = \int{ \left(v_{los} - \overline{v}_{los} \right)
    F\left(v_{los}\right) \mathrm{d}v_{los}}
    \]
e.g., central velocity dispersion or, more generally, the velocity
dispersion profile.

Deviations from a Gaussian can be measured (at high S/N),
and are usually characterized
by higher order moments: \textit{skew} ($h_{3}$) and \textit{kurtosis}
($h_{4}$):
\[
    F\left(v_{los}\right) \propto \exp\frac{-\omega^{2}}{2}\left[
        1 + \sum{h_{k}H_{k}(\omega)}\right]
    \]
where $\omega \equiv (v_{los}-\overline{v})/\sigma$ and
$H_{k}$ are Gauss-Hermite functions.

Extracting information about velocity distribution requires:
\begin{itemize}
    \item Some model of the underlying composite spectrum: stellar absorption
        lines are not infinitely narrow, and blends are common: use some
        template, either from observations of a stellar library or atmospheric
        models
    \item An understanding of the instrumental broadening, i.e., the line
        spread function (LSF)
    \item for a rough approximation, the instrumental broadening may be close
        to Gaussian, and if one just wants to characterize the velocity
        dispersion (i.e. Gaussian profile), one would determine the Gaussian
        width needed to match template to observed data, and subtract the
        instrumental Gaussian width in quadrature, since two Gaussian
        convolutions (LSF and velocity distribution) are equivalent to a single
        convolution with a Gaussian with width
        \[
            \sigma_{tot}^{2} = \sigma_{LSF}^{2} + \sigma_{LOSVD}^{2}
            \]
        Measuring velocity dispersion significantly smaller than the
        instrumental resolution is challenging, and at some level, becomes
        impossible.
    \item sophisticated techniques exist for extracting velocity distribution
        information, e.g. penalized pixel fitting
%        (\href{}
        {pPXF}).
\end{itemize}

Some
\href{http://astronomy.nmsu.edu/holtz/a555/resources/examplesh3h4.gif}
{examples of velocity data}. What kind of galaxies are these?

\underline{\textbf{Relevance}}:
Internal velocities can be used to probe mass distribution within
galaxies (assuming that gravity drives the motions).
For a rotationally supported system:
\[
    v(r) = \sqrt{\frac{GM(r)}{r}}
    \]
For a collisionless system of \mynotes{(independent)} particles,
the distribution of velocities at
any given location characterized by the velocity ellipsoid
(with radial, azimuthal, and vertical components),
and Jeans equation leads to:
\[
    \frac{\mathrm{d}\rho\sigma^{2}}{\rho{\mathrm{d}r}} +
    \frac{2\beta\sigma^{2}}{r} = -g(r)
    \]
where $\beta$ is the \textbf{velocity anisotropy}:
\[
    \beta \equiv \left(
        1 - \frac{\sigma_{\theta}^{2}+\sigma_{\phi}^{2}}{2\sigma_{r}^{2}}
    \right)
    \]
though $\beta$ is not generally known. So mass modeling of kinematically
``hot'' systems is not so easy.

Mean velocities of galaxies indicate that they are receding, with velocity
proportional to distance $\rightarrow$ expansion of the universe. There is a
dispersion of velocities around the mean relation; deviation is called the
peculiar velocity. In the field, typical peculiar velocities are a few hundred
km/s, in clusters, can be a thousand km/s.

\subsection{Distances to galaxies}
Some properties of galaxies, such as luminosity and linear size,
require knowledge of distances to galaxies:
\[
    M - m = -5\log{d_{L}} + 5
    \]
\begin{itemize}
    \item $d_{L}$ = luminosity distance [pc]
    \item $M$ = absolute magnitude (or SB)
    \item $m$ = apparent magnitude
\end{itemize}

\paragraph{Various techniques for getting distances}
%\footnote{see \href{http://adsabs.harvard.edu/abs/2010ARA\%26A..48..673F}
%{Freedman \& Madore, ARAA review}}
\begin{itemize}
    \item Variable stars: Cepheids/RR Lyrae stars through period-luminosity
        (period-luminosity-color/metallicity) relation.
    \item RGB tip Nearby galaxies with old stellar populations where
        individual stars can be resolved.
    \item Geometric techniques
        \begin{itemize}
            \item masers (proper motion compared with radial velocity in central disks)
            \item exclipsing binaries
        \end{itemize}
    \item SB fluctuations From imprint of Poisson statistics. Requires objects
        with a ``standard'' population (i.e.\ stellar luminosity function).
    \item Planetary nebulae luminosity function
    \item More luminous standard candles: type-Ia supernova
        (``standardizable'')
    \item Scaling relations between velocity and luminosity
        \mynotes{for relatively nearby galaxies}
        \begin{itemize}
            \item Spirals: \textbf{Tully-Fisher relation} between maximum rotational velocity and luminosity
            \item Ellipticals: fundamental plane relation between velocity
                dispersion, SB, and physical size
                ($D_{n}-\sigma$) relation.
        \end{itemize}
\end{itemize}
%See \S\ref{} for more details.

\paragraph{Redshift}
Galaxies expand with everything else in the cosmic flow.
\[
    z = \frac{\Delta\lambda}{\lambda}
    = \frac{\lambda_{obs} - \lambda_{rest}}{\lambda_{rest}}
    \qquad\qquad
    \lambda_{obs} = \lambda_{rest} (1 + z)
    \]
\begin{itemize}
    \item At \emph{low} $z$, $z \approx v/c, v = H_{0}d$.
        Determining $H_{0}$ requires independent distance measurement;
        uncertainties are often parameterized by using $H_{0} = 100h$
        km s$^{-1}$ Mpc $^{-1}$, and distance-dependent quantities given
        in terms of $h$. Current best estimate of $H_{0}$ is $\sim$ 70 km s$^{-1}$.
    \item At \emph{high} $z$, need knowledge of cosmological
        parameters to get distance from redshift (i.e.\ redshift-distance
        relation.\footnote{see, e.g.\
        \href{http://ned.ipac.caltech.edu/level5/Hogg/Hogg_contents.html}
        {Hogg, Distance Measures in Cosmology}})
\end{itemize}

Note that observed velocity needs to be corrected for earth motion (to
heliocentric or barycentric velocity), motion of the Sun (to LSR), rotation of
Galaxy (to GSR), and motion of Galaxy within the Local Group.

Galaxies also move around relative to the uniform expansion because of local
gravitational attraction from concentrations of mass. The deviation of the
velocity from the smooth flow is called the \textit{peculiar velocity} (as
mentioned above). Typical peculiar velocities are on the order of several
hundred km s$^{-1}$, but depend on environment; c.f.\ Milky Way and galaxy
clusters; if these are ignored, introduces spurious features in distance maps
(\href{http://astronomy.nmsu.edu/holtz/a555/resources/cfa.html} {``fingers of
God'', redshift space distortions}). \mynotes{(``c.f.'' means roughly ``with
the exception of'', at least in this context.)}

Hubble's law and the smooth Hubble flow (at what velocity? what redshift?)

At $z \gtrsim 0.1$, peculiar velocities lead to relatively small distance
errors. At lower redshifts, might estimate distances using model of mass
distribution and velocity field, but this is likely not well determined.

\paragraph{Spectroscopic redshift}
Determining redshift spectroscopically for distant (faint) galaxies is
observationally time-consuming. Accurate positions of individual lines
require reasonable S/N; emission line galaxies are significantly easier.
It is possible to get good redshifts from low
S/N observations by taking advantage of multiple spectral features,
e.g., by cross-correlation of spectrum against a template
in $\log\lambda$ space.

\paragraph{Photometric redshift}
Lower accuracy distances are possible from so-called \textit{photometric
redshifts} \mynotes{(though can't disperse light.)}
Use low resolution spectral
information, i.e.\ multiband photometry to constrain redshift.\footnote{
    e.g. \href{http://astronomy.nmsu.edu/holtz/a555/resources/photoz.png}
    {plot} (from Niemack et al 2009, ApJ)}
The different intrinsic SEDs of galaxies, internal reddening, and
redshift need to be separated.
Accuracy of results obviously depends on quality of photometry
and number of bandpasses. Typical accuracies (checked against spectroscopic
redshifts) give
\[
    \frac{\Delta{z}}{\left(1+z\right)} \sim 0.1
    \]
(sometimes better); note typical $\sim$ 10\% outlier fraction
(varies for different types of galaxies).

\subsection{Morphological classification}
\href{https://ned.ipac.caltech.edu/level5/Sept11/Buta/frames.html}
{Level 5: GALAXY MORPHOLOGY - Ronald J. Buta (2013)}, also
\href{http://people.virginia.edu/~dmw8f/astr5630/Topic02/Lecture_2.html}
{Whittle notes}.

Historically, galaxies were considered in terms of their morphology, i.e.\
the Hubble sequence. However, it is not clear to what extent
morphological classification traces underlying physics, and descriptive
morphology may be biased by things that are not fundamental. Still, it
is widely used, so important to understand a bit.

\subsubsection{Morphological systems}
Good reference: Sandage in Galaxies and the Universe, 1975.
Some pictures:
\begin{itemize}
    \item \href{http://astronomy.nmsu.edu/holtz/a555/html/diagrams/a616/ellips.htm}
        {{ellipticals}}
    \item \href{http://astronomy.nmsu.edu/holtz/a555/html/diagrams/a616/s0.htm}
        {{S0s}}, also known as lenticulars
    \item \href{http://astronomy.nmsu.edu/holtz/a555/html/diagrams/a616/spirals.htm}
        {{spirals}}
    \item \href{http://people.virginia.edu/~dmw8f/astr5630/Topic02/t2_hubble_tfork.html}
        {{Tuning fork diagram}}
\end{itemize}

\subsubsection{Hubble classification}
\begin{description}
    \item [ellipticals:] given as $En$, where
        \[
            n = 10\left( \frac{1-b}{a} \right)
            \]
        $E0$ (sphere) $\rightarrow$ $E7$ (skinny).
        No distinction between dwarf ellipticals and dwarf spheroidals
    \item [spirals:] barred or unbarred (SBa,SBb,SBc; or Sa,Sb,Sc;
        respectively) where a,b,c denote size of bulge (bulge-to-disk ratio:
        B/D) in decreasing order. Also classified according to tightness of
        arms and the degree to which arms are resolved into individual HII
        regions (note that the latter two are essentially impossible to judge
        for edge-ons). These criteria are not perfectly correlated, and
        classifiers might emphasize different criteria.
        Note that spiral arms are generally higher contrast in surface
        brightness than they are in mass, so they may be somewhat misleading;
        also note that origin of spiral arms may not be fundamentally related
        to global galaxy properties.
    \item [S0s:] (aka lenticulars, or disks); intermediate, no spiral
        structure but have disk system, split into $S0_{1}$, $S0_{2}$, $S0_{3}$,
        depending on the amount of dust.
    \item [irregulars:]
        \begin{description}
            \item [Irr I] (Magellenic irregulars with lots of distinct HII regions)
            \item [Irr II] (lack the resolution into distinct HII regions).
        \end{description}
    \item \href{http://people.virginia.edu//~dmw8f/astr5630/Topic02/t2_Hubble_orientation.jpg}
        {Some pictorial examples}
\end{description}

\subsubsection{deVaucouleurs/RC3 classification}
\href{http://astronomy.nmsu.edu/holtz/a555/html/diagrams/a616/rc3class.htm}
{Diagram}.
See also \href{http://astronomy.as.virginia.edu}{here}.
\begin{itemize}
    \item Extends Hubble to later spiral types Sd, Sm, and finally Im.
    \item Extra classes around $S0$ ($S0^{-}, S0^{+}$)
    \item Allows for intermediate between barred and unbarred:
        \begin{itemize}
            \item SA: normal spirals
            \item SB: barred spirals
            \item SAB: transition
        \end{itemize}
    \item Adds extra distinction for \emph{ring} vs.\ \emph{s} shaped.
    \item Note that location along spiral dequence may not be the same as in
        ``Hubble classification'' (more based on B/D).
    \item Some \href{http://astronomy.as.virginia.edu}
        {{pictorial examples}}
    \item \href{http://astronomy.nmsu.edu/holtz/a555/html/diagrams/a616/numtype.htm}
        {{Numerical galaxy types (T)}}
\end{itemize}
Note that morphological classification often depends on multiple
characteristics, and therefore it can be somewhat subjective.
Quantitative classification schemes have been worked on
(see \myref{sssec:quantitative}),
but are not widely used for nearby galaxies.

Morphology depends on wavelength
(e.g.\ \href{http://ned.ipac.caltech.edu/level5/Kuchinski/frames.html}
{O'Connell 9609101}).
Be careful about morphological classification at higher redshift
(or at least a direct comparison with lower redshift galaxies)
because you may be looking at morphology at a different wavelength
(morphological \textit{K-correction:} correction to magnitude (or flux) due
to redshift). \mynotes{K-correction is introduced in a previous section... should put
in a reference or something.}

Based on global characteristics, the Hubble morphological classification of
ellipticals is probably not fundamental.
Viewing angle must play a part at some level.
It may be more meaningful to classify
by \emph{isophotal shape}\footnote{
    \href{http://astronomy.as.virginia.edu}{(Kormendy and Bender classification)}}
or kinematically by $v/\sigma$.

For the spiral sequence, there are correlations of luminosity, surface
brightness, rotational velocity and gas fraction with Hubble type. However,
each category has a broad range of global observables, and the categories overlap
significantly.\footnote{Some representative data from
\href{http://adsabs.harvard.edu/cgi-bin/nph-bib_query?bibcode=1994ARA\%26A..32..115}
{Roberts and Haynes review (ARAA 1994)}}:
\begin{itemize}
    \item Considering mean or median values, there is little trend in radius,
        $L_{B}$ \mynotes{(Bolometric luminosity?)}, and total mass for
        $S0$-$Sc$, but later types tend to be smaller, less massive, and less
        luminous
        (\href{http://astronomy.nmsu.edu/holtz/a555/html/diagrams/a616/rh2.htm}
        {{RH figure 2}}).
        However, note that LSB \mynotes{(Low Surface Brightness?)} galaxies
        tend to be later types, and since these are not included, there may be
        a bias here.
    \item Typical SB may be lower for latest types, even without inclusion of
        LSB galaxies; CSB\mynotes{(?????)} much higher for large ellipticals
        than spirals. Mass surface density appears to decrease monotonically
        with morphological class in spirals
        (\href{http://astronomy.nmsu.edu/holtz/a555/html/diagrams/a616/rh3.htm}
        {{RH figure 3}}).
    \item Cold gas is absent in ellipticals, and neutral hydrogen (HI) content
        appears to increase monotonically with type for spirals. It is possible
        that molecular gas content decreases with type, but not certain.
        (\href{http://astronomy.nmsu.edu/holtz/a555/html/diagrams/a616/rh4.htm}
        {{RH figure 4}}).
\end{itemize}
As stated above, all of the aforementioned classification schemes are
subjective at some level. Note general problem of human classification on
large-scales, e.g. with millions of galaxies, although also note Galaxy Zoo.
\mynotes{(?????????????????)}

\subsubsection{Quantitative classification schemes}\label{sssec:quantitative}
Some of the more quantitative schemes that have been
proposed, both parametric and non-parametric, are:
\begin{itemize}
    \item \textbf{Bulge-to-disk ratio}, using B/D decomposition\footnote{
        \href{http://adsabs.harvard.edu/cgi-bin/nph-bib_query?bibcode=2003ApJ...582..689}
        {MacArthur, Courteau, \& Holtzman, ApJ 582, 689 (2003)}}.
        Issues: covariance between parameters, 1D vs.\ 2D, and validity of models.
    \item \textbf{Global profile fit}, e.g.\ Sersic index
    \item \textbf{Concentration}, e.g.\ SDSS $r_{90}/r_{50}$, $r_{80}/r_{20}$,
        using circular apertures. Elliptical apertures can also be considered,
        e.g.\ based on \emph{second order moments}.
        Related to B/D. Sensitivity to seeing/distance.
\end{itemize}

\subsubsection{Non-symmetric galaxies}
Many of the schemes used for nearby galaxies fail for non-symmetric galaxies,
for which other indicators have been suggested:
\begin{itemize}
    \item \textbf{Asymmetry:}\footnote{\href{http://adsabs.harvard.edu/cgi-bin/nph-bib_query?bibcode=1996ApJS..107....1A}
        {Abraham et al, ApJS 107, 1, 1996}}
        Rotate about center, self-subtract,
        \[
            A = \frac{0.5\vert\mathrm{subtracted}\vert}{\mathrm{total}}
            \]
        Possible indicator of mergers.
    \item \textbf{Clumpiness:}\footnote{\href{http://adsabs.harvard.edu/cgi-bin/bib_query?2003ApJS..147....1}
        {Conselice ApJS 147, 1 (2003)}} e.g.\ S. Subtract smoothed version of
        image from original image, ratio of flux in subtracted image to flux in
        original images gives S. Possible indicator of star formation.
    \item \textbf{\href{http://astronomy.nmsu.edu/holtz/a555/images/gini.htm}
        {Gini coefficient:}}\footnote{\href{http://adsabs.harvard.edu/cgi-bin/nph-bib_query?bibcode=2003ApJ...588..218A}
        {Abraham et al, ApJ 588, 218 (2003)}}
        Applied to sorted list of pixel values. May be more stable at low
        resolution, S/N.
    \item \textbf{M20:} second order moment of brightest 20\% of
        galaxy light
        \footnote{\href{http://adsabs.harvard.edu/cgi-bin/nph-bib_query?bibcode=2004AJ....128..163L}
        {Lotz et al., AJ 128, 163 (2004)}}
        May be more stable at low resolution, S/N.
\end{itemize}

\subsection{Galaxy catalogs/resources}
\subsubsection{Large area/all sky surveys}
\subsubsection{Smaller region surveys}
\subsubsection{Spectroscopic surveys}
\subsubsection{``Classic catalogs''}
\subsubsection{Modern digital catalogs}
\subsubsection{Web tools}
\subsubsection{Selection effects}
\begin{itemize}
    \item There is generally no such thing as a catalog that contains
        ``everything" since there will always be objects that fall below the
        detection criterion for inclusion
    \item a ``complete" catalog is a catalog that has all objects that
        satisfy well-defined selection criteria, but this still does not
        mean that you can't get biased results from it, you still have to
        consider the effects of the selection function
    \item Consider a plot from a redshift survey of galaxy absolute
        magnitude vs redshift
    \item For magnitude limited samples, more luminous objects will be
        overrepresented: Malmquist bias (but beware that sometimes people
        refer to different things with this name!)
    \item Also may need to be careful about systematic biases arising from
        uncertainties in measured magnitudes, distances, if population is
        not uniform in underlying distribution of absolute mags and
        distances sampled - which it isn't! Luminosity function rises
        towards fainter mags, so more galaxies are scattered bright than
        are scattered faint. In distance, more galaxies are located farther
        than nearer just because of volume element, so more galaxies
        scattered into sample (with corresponding error in absolute mag)
        than scattered out of it
    \item Note many biases are possible aside from a simple mag cut, e.g.,
        color bias, surface brightness bias, etc.
        Whenever you see a relation that shows a correlation, consider
        selection effects!
    \item Best approach to selection biases may be to ``forward model",
        i.e. model underlying distribution through selection function and
        compare to observed properties.
\end{itemize}

\newpage
\section{Overview of nearby galaxy population}
Consider some of the main quantifiable observable properties of nearby
galaxies.

\subsection{Statistical properties of galaxies}
Consider relative number of galaxies with different parameters,
e.g.\ color or luminosity.

\subsubsection{Colors of galaxies}
Consider the distribution of SEDs of galaxies, to first order represented by
their color. There is a striking bimodality \href{http://astronomy.nmsu.edu/holtz/a555/images/strateva.htm}
{in color}\footnote{quantified from SDSS by \href{http://adsabs.harvard.edu/cgi-bin/bib_query?2001AJ....122.1861S}
{Strateva et al., AJ 122, 1861 (2001)}}:
\begin{itemize}
    \item Widley used terminology of red and blue sequences
        (with ``green valley'' in between).
    \item Red sequence is tighter than blue sequence, so latter is
        sometimes called the ``blue cloud''.
    \item Red sequence extends to higher luminosities, blue to lower
        luminosities, though there is significant overlap.
\end{itemize}
At some level, this bimodality is nothing more than the bimodality between
ellipticals/early-type spirals without much current star formation (red) and
later-type spirals with current star formation (blue), although dust, bulges,
and metallicity all play a role. The correspondance with morphology is
supported by correlation with structural parameters, e.g.\ \href{http://astronomy.nmsu.edu/holtz/a555/resources/blanton_araa_correlations.htm}
{multidimensional correlations}.\footnote{\href{http://adsabs.harvard.edu/abs/2009ARA\%26A..47..159B}
    {Blanton and Moustakis ARAA 2009}}

The key point is that color is easily observed and quantified. Given likely
differences in stellar populations, the relation between luminosity and stellar
mass is probably different for the two different sequences (red and blue). At a
rough level, the color allows the stellar mass to be estimated from the
luminosity, though some colors have more information about this than others,
and some bandpasses for luminosity are more affected by differences in the
stellar populations (more later).

Note that \href{http://astronomy.nmsu.edu/holtz/a555/images/mnr_11081_f7.htm}
{the relation shifts when expressed in terms of stellar mass},\footnote{\href{http://adsabs.harvard.edu/abs/2006MNRAS.373..469B}
    {Baldry et al. 2006}
}
and that there appears to be a transistion
mass around $10^{10}$ M$_{\odot}$ between the two sequences.

Bimodality exists in all environments, with comparable sequences, but different
relative numbers; see \href{http://astronomy.nmsu.edu/holtz/a555/resources/MNR_11081_f9.gif}
{color-density relation}.\footnote{\href{adsabs.harvard.edu/abs/2006MNRAS.373..469B}
{Baldry et al 2006}}
Note individual
sequences themselves appear relatively consistent between environments.\footnote{
See also \href{adsabs.harvard.edu/abs/2010ApJ...721..193P}
{Peng et al 2010, Fig 6}}

\test{Understand what is meant by the bimodality in galaxy colors, and be
able to reproduce what the color-magnitude relation of galaxies looks like.
Know the terminology: blue cloud, red sequence, green valley.}

\subsubsection{Luminosity function}
The luminosity function (LF) gives the number density of galaxies as a
function of luminosity, and can be expressed in either luminosity or
absolute magnitude units.
\begin{description}[labelwidth=3em, labelindent=0.25in]
    \item [$\Phi(L)$] = number density of galaxies with
        luminosity between $L$ and $L$ + $\mathrm{d}L$
    \item [$\Phi(M)$] = number density of galaxies with absolute
        magnitude between $M$ and $M$ + $\mathrm{d}M$
\end{description}
The integral over the LF gives the total number density of galaxies (at all
luminosities). The LF is the first step toward understanding a fundamental
question of galaxy formation: What sets the \emph{range} of galaxy luminosities
and the relative \emph{numbers} of objects at each luminosity? (though mass may
be a more fundamental characteristic than luminosity\ldots) The LF is also an
important cosmological probe for evolution of the galaxy population (more
later).

\paragraph{Measuring the luminosity function}
Measurement of the LF requires distance measurements (to get luminosity), and
critically depends on knowledge of the \textbf{selection function}.
The simplest idea for correction is to weight counts by $V_{max}$, the largest
volume to which the galaxy could have been observed given the selection
function. However, there is an issue with large scale structure, so
$\cfrac{V}{V_{max}} \rightarrow$ test for uniformity and understanding
of selection function\footnote{
    Significantly more sophisticated methods have been developed; for more details, see
    \href{http://adsabs.harvard.edu/cgi-bin/nph-bib_query?bibcode=1988ARA\%26A..26..509}
    {Binggelli et al (ARAA 26, 26, 1988).}
}
In general, need to be \emph{very} careful about understanding selection effects.
\href{http://astronomy.nmsu.edu/holtz/a555/resources/lf_selection.gif}
{Model (and verify) them}.

%------------------------------------------------------------------------
\mynotes{For magnitude limited samples, more luminous objects will be overrepresented.
This is called the Malmquist bias\footnote{Be aware that this name is sometimes
used to refer to different things.} Also may need to be careful about
systematic biases arising from uncertainties in measured magnitudes and
distances, if population is not uniform in underlying distribution of absoluate
mags and distances sampled (which it isn't). The LF rises toward fainter
magnitudes, so more galaxies are scattered bright than are scattered faint
(WTF??). In distance, more galaxies are located farther than nearer
because of volume element (seriously, WTF), so more galaxies
scattered into sample (with corresponding error in absolute magnitude) than
scattered out of it. Note that many biases are possible aside from a simple
magnitude cut, e.g.\ color bias, SB bias, etc.}
%------------------------------------------------------------------------

\paragraph{Observed luminosity function, functional fits, and parameters}
\begin{itemize}
    \item Absolute magnitudes ($M_{V}$) range from -22 to -8.
    \item Nomenclature: ``dwarf'' galaxies vs.\ ``normal'' galaxies.
    \item What objects dominate by number? What objects dominate the
        total luminosity?
    \item Different types have clearly different LFs
    \item Since proportion of a given type depends on environment, so
        must the general LF.
\end{itemize}

LFs are usually well characterized by a \textbf{Schecter Function}
\[
    \phi(L) = \frac{\phi_{*}}{L_{*}} \left( \frac{L}{L_{*}} \right) ^{\alpha}
    \exp\left( -\frac{L}{L_{*}} \right)
    \]
\[
    \phi(M) \propto 10^{ -0.4 \left( \alpha + 1 \right)
    \left( M - M^{*} \right) }
    \exp \left[ -10^{ 0.4(M^{*} - M) } \right]
    \]
where $\phi(L)$ is the number of galxies with luminosity between
$L$ and $L + dL$.  $\phi_{*}$, $L_{*}$, and $\alpha$ (the faint-end slope)
are the
\href{http://astronomy.nmsu.edu/holtz/a555/resources/schechter.gif}
{three parameters}. The Schecter function is only an approximation, and
is purely empirical.

A possibly useful feature of the Schecter function is that it can be
integrated (for $\alpha > -2$) to get \textit{total luminosity density}:
\[
    j = \phi_{*}L_{*}\Gamma\left(\alpha + 2\right)
    \]
\mynotes{$\Gamma$ = ?}

%Typical ``local'' parameters\footnote{
%    \href{http://astronomy.nmsu.edu/holtz/a555/resources/monterotab2.gif}
%    {parameters for different bandpasses} from
%    \href{http://adsabs.harvard.edu/abs/2009MNRAS.399.1106M}
%    {Montero-Dorta \& Prada (2009)}
%}
(e.g. SDSS at $z\sim0.1$) are:
\begin{itemize}
    \item $\phi^{*}\sim0.015h^{3}$ Mpc$^{-3}$
    \item $M_{B}^{*}\sim-19.5$
    \item $M_{R}^{*}\sim-20.5$
    \item $\alpha=-1$ to $-1.5$
\end{itemize}
As noted above, there is a dependence on type.\footnote{
    e.g.\ \href{http://astronomy.nmsu.edu/holtz/a555/resources/lf_environment.gif}
    {Driver \& Propris},
    \href{http://astronomy.nmsu.edu/holtz/a555/html/diagrams/a616/marzke3.htm}
    {Marzke, figure 3}
}

So what is the implied luminosity density? And (getting ahead of ourselves)
what is the implied stellar mass density, c.f. cosmological?

There has been significant \href{http://astronomy.nmsu.edu/holtz/a555/images/blantontab3.htm}
{discussian} over detailed shape and normalization of
LF\@. There are probably both observational issues (e.g.\ selection functions)
and astrophysical ones (e.g.\ large scale structure, ``cosmic variance'').

\paragraph{Origin of the luminosity function}
relative number of galaxies at different masses, if there is a correlation
between mass and luminosity. Related to primordial distribution? compare galaxy
LF with dark halo LF, figure from Whittle cooling at highest masses galaxy
``efficiency" at higher and lower galaxy masses: feedback? affecting numbers or
luminosities?

\paragraph{Evolution of the luminosity function}
LF evolution contains a lot of information about galaxy evolution
(understand galaxies by ``looking back in time'').
Some possibilities:
\begin{itemize}
    \item No evolution: distribution of galaxy luminosities is unchanging.
    \item Passive evolution: number and internal (stellar) makeup of
        galaxies is unchanging \mynotes{(no star formation)},
        but luminosity evolves as stars evolve.
    \item Luminosity evolution: stellar makeup of galaxies changes with
        time (star formation), leading to luminosity changes.
    \item Number (density) evolution: number of galaxies changes with time.
        Galaxies may be either created (formed) or destroyed (e.g.\
        mergers) as a function of luminosity.
\end{itemize}
There is a large amount of work on this, but the LF definitely shows
evolution\footnote{Faber et al. (2007)}. All samples show luminosity
evolution, but red galaxies show number density evolution. \href{http://astronomy.nmsu.edu/holtz/a555/images/faberdeepf7.htm}
{Schechter parameters}: ``quenching'' of galaxies: moving galaxies from blue
to red sequence?

Higher redshift results: the UV luminosity function: (links)

\newpage
\subsection{Elliptical/Spheriodal galaxies}
\subsubsection{Surface brightness profiles}
Ellipticals are probably better fit by Sersic profiles than by
deVaucouleurs\footnote{Caon et al. 1993}.
\mynotes{Previously, stated that \emph{sphericals} are characterized
by deVaucouleurs, not ellipticals.}
\begin{itemize}
    \item Correlation of Sersic indices with other parameters
        suggests that there is something physical going on, but be aware that
        there may be fit degeneracies.
    \item Low luminosity Es (and the bulges in spirals) may be better
        represented by exponentials.
    \item \href{http://astronomy.nmsu.edu/holtz/a555/resources/caonfig3.gif}
        {Correlation of Sersic indices with other parameters} suggests that
        there is something physical going on (but be aware of fit degeneracies).
    \item Slope of SB profile (Sersic $n$) appears to be correlated with
        luminosity
    \item SB vs L turns over at intermediate luminosity:
        two families of shperoidal systems? Or continuous trend?\footnote{
            See \href{http://adsabs.harvard.edu/abs/2009ApJS..182..216K}
            {Kormendy et al. (2009)}}
        \mynotes{$\rightarrow$ \textbf{Cuspy cores}}.
    \item Ellipticals exist over wide range of sizes, luminosities, and SBs.
\end{itemize}
Other profiles that have been used:
\begin{itemize}
    \item King model (truncated Gaussian for velocity distribution). This
        is the only profile with theoretical basis; the others are
        empirical - Hubble law doesn't converge.
    \item Hubble profile:
        \[
            \Sigma(r) = \frac{ \Sigma_{s} }
            {\left( 1 + \frac{r}{r_{s}} \right)^{2}}
            \]
        where $\Sigma_{s}$ = 0.25$\Sigma_{0}$
\end{itemize}
None are perfect matches to data over all scales\footnote{see Burkert 1993 for
comparison with deVaucouleurs law}.

Inner regions of spheroidal galaxies deviate from Sersic profiles fit to outer
regions. (Outer regions can also deviate from Sersic profiles, especially
notable in \href{http://astronomy.nmsu.edu/holtz/a555/resources/cDprof1.gif}
{central cluster (cD) galaxies}.)
Inner regions are of particular interest because they may reflect
dissipational collapse of low angular momentum material\footnote{See, e.g.\ NGC
4472}. Initially, it seemed like galaxies had cores, but this is partly an
effect of seeing. Higher spatial resolution (HST) shows that galaxies have both
flat and steep (cuspy) inner profiles\footnote{e.g.\ NGC720 vs.\ NGC4621}.
Inner profiles seem to be roughly bimodal (though this bimodality is currently
under debate).

Parametric fits to account for these become more complex,
e.g.\ the ``nuker law'':
\[
    I(r) = I_{b}2^{ (\beta-\gamma) / \alpha) }
    \left( \frac{r_{b}}{r} \right) ^{\gamma}
    \left[ 1 + \left( \frac{r_{b}}{r} \right) ^{\alpha} \right]
    ^{ \left( \gamma - \beta \right) / \alpha }
    \]
where $\gamma$ is the slope of the inner power law, $\beta$ is the slope of
the outer power law, and $\alpha$ is the sharpness of the break between
them.

Profile type is \href{http://astronomy.nmsu.edu/holtz/a555/images/lauerfig5.htm}
correlated with luminosity. Luminous ellipticals tend to
have ``cuspy cores'' with a break radius and shallower central density
profile. Lower luminosity ellipticals have power laws all the way in.
Galaxies with cusps may have ``extra light'', perhaps related to an
accretion event; note that at least some of these show distinct kinematic
signature in their cores.

The existence of cores may be indicative of a previous merging event, with
merging black holes ``scouring'' the core.

\subsubsection{Intrinsic (3D) shapes of ellipticals}
\begin{itemize}
    \item oblate (2 long axes)
    \item prolate (1 long axis)
    \item triaxial (all axes different length).
\end{itemize}
True shapes are determined by by looking at distribution of ellipticities.
\begin{itemize}
    \item Distribution function is different for fainter and
        brighter Es.
    \item For bright giant Es, distribution is \emph{inconsistent} with
        either prolate or oblate itrinsic shapes: not enough circular
        galaxies\footnote{Tremblay and Merritt Fig 3}.
    \item For fainter Es, distribution is \emph{consistent} with
        oblate, prolate, or triaxial.
    \item Triaxiality is also inferred for some giant Es from
        observation of isophotal twisting, which you cant get from
        oblate or prolate shape\footnote{de Zeeuw, Fig 1}.
\end{itemize}

\subsubsection{Non-axisymmetric features in galaxies}
Often described by amplitudes of Fourier moments of intensity distribution as a
function of radius, e.g. $\alpha_{1}$, $\alpha_{2}$, $\alpha_{4}$

%\[
%    I\left(r,\theta\right) = \sum{c_{m}\cos\left(m\theta\right)} +
%    \sum{s_{m}\sin\left(m\theta\right)}
%    \]
%\[
%    \alpha_{m} = \frac{\sqrt{c^{2}_{m} + s_{m}^{2}}}{I_{o}}
%    \]

\[
    R_{iso}(\phi) - R_{ell}(\phi)
    = a_{0}
    + \sum{ a_{n} \cos{n\theta}  }
    + \sum{ b_{n} \sin{n\theta}  }
    \]
($\phi$ measured from major axis?)

``First even term above ellipses is $\alpha_{4}$ term:'' \mynotes{(WTF???)}
\begin{itemize}
    \item ``boxy'' isophotes: $\alpha_{4} < 0$;
        bright, slow, central cores, strong radio and x-ray.
    \item ``disky'' isophotes: $\alpha_{4} > 0$;
        faint, significant rotation-flattening, little
        radio or x-ray emission, steep cusp.
\end{itemize}
Good diagram showing the difference: \url{http://astronomy.nmsu.edu/holtz/a555/resources/mofig2.15.gif}

Deviations from elliptical fits are seen in elliptical/spheroidal systems,
e.g.\ NGC 821 v.\ NGC 2300. It is possible that isophotal deviations form
a continuous sequence\footnote{e.g.\ Kormendy and Bender classification}:
\begin{center}
    S $\rightarrow$ S0 $\rightarrow$ disky Es $\rightarrow$
    true Es $\rightarrow$ boxy Es
\end{center}
It is also possible that true Es are just disky Es viewed
edge-on\footnote{Kormendy and Bender, figure 3}, and that there are two
distinct classes of Es. Some Es also show non-symmetric deviations from
smooth SB distributions, e.g.\ ripples, shells, and streams, especially
in outer regions.

\subsubsection{Kinematics}
Elliptical galaxies are kinematically \emph{hot}. The random motions of stars
is generally large compared to the organized motion (but this is not to say
that there is no rotation at all).

The key kinematic quantity is \textbf{velocity dispersion}, $\sigma$.
Galaxies can be characterized by \emph{central} $\sigma$, but $\sigma$ does
vary with radius.

Some ellipticals have some rotation, mostly in lower luminosity systems.
The relative importance of \emph{organized} motion over \emph{random} motion
can be characterized by $v_{rot}/\sigma$. Shapes are expected to be influenced
by rotation: for an oblate model with isotropic velocity distribution that is
flattened by rotation:
\[
    \frac{v_{rot}}{\sigma} = \sqrt{\frac{\epsilon}{\left(1-\epsilon\right)}}
    \]
Giant ellipticals have less rotation than this, implying anisotropic velocity
dispersions to account for their shape (required for triaxial systems in any
case). Low/medium luminosity (high SB) Es, however, \emph{may} be isotropic
with flattening caused by rotation
\footnote{
    \href{}
    {de Zeeuw, figure 3}
}.

Low SB Es appear to have anisotropic $\sigma$, i.e.\ more eccentric than
expected from rotation; ``measured in LG Es 185 (factor of three low in
$v_{rot}/\sigma$, 147, factor of 10 low).'' \mynotes{seriously, wtf}.

Significant fraction of Es may have dynamical subcomponents, e.g.\ velocity
and dispersion for some interesting cases.

\subsubsection{Relations between different parameters and different families
of ellipticals}

\paragraph{Photometric parameters}
SB-size relation, between SB and SB-luminosity relations
(``Kormendy relations'')\mynotes{???}.
Also, profile \emph{shape} (e.g. Sersic index) with luminosity.

\paragraph{Kinematics vs.\ luminosity}
More luminous galaxies have less rotation, but higher velocity dispersions
(\textbf{Faber-Jackson relation}, which is roughly $L\sim\sigma^{4}$, but
lots of scatter).

\paragraph{The Fundamental Plane}
Faber-Jackson relations and the Kormendy relation (between SB and L/size) are
manifestations of the
\href{http://astronomy.nmsu.edu/holtz/a555/images/djorfig2.htm}
{\textbf{Fundamental Plane of elliptical galaxies}}; there
is a correlation between residuals in the Faber-Jackson emphrelation and SB.
Galaxies do not populate the entire 3D space of $I$, $r$, and $\sigma$, but
instead populate only a \emph{plane} in this space.
\begin{itemize}
    \item There is a relation between the \textbf{three fundamental global
        observables}: SB (or luminosity), size, and velocity
        dispersion\footnote{original refs: Dressler et al. 1987; Djorgovski
        and Davis 1987; Bender, Burstein and Faber 199x}.
    \item The observed relation given by Virgo ellipticals is:
        \[
            r_{e} \propto \left( \sigma_{0}^{2} \right) ^{0.7}I_{e}^{-0.85}
            \]
        What is the origin of relation?
        Might expect something like this if
        the following three assumptions are true:
        \begin{enumerate}
            \item Es are in virial equilibrium
            \item M/L varies systematically with luminosity
            \item Es form a ``homologous'' family, all, e.g.\ with
                deVaucouleurs profiles.
        \end{enumerate}
        In this case, one expects:
        \[
            L = c_{1}I_{c}r_{c}^{2}
            \]
        \[
            M = c_{2}\frac{\sigma^{2}_{o}r_{e}}{G}
            \]
        where the first is a definition and the second is related to the
        \textbf{virial theorem} (though this applies to the mean potential
        energy and mean kinetic energy per unit mass, averaged over the
        entire system). The constants are related to the shapes and other
        details (not necessarily constant among the variety of Es).

        Combining these, we get
        \[
            r_{e} = \frac{c_{2}}{c_{1}}
            \left( \frac{M}{L} \right)^{-1} \sigma_{0}^{2}I_{c}^{-1}
            \]
        If $M/L \propto L^{~0.2}$, then the observed fundamental plane is
        recovered; could arise from stellar populations or from variations
        in baryon to total mass. Alternatively, if $M/L$ is constant with a
        structure that varies relative to one or more of the fundamental
        variables (which we know it does, e.g., systematic variations of
        Sersic $n$ with luminosity, but not clear if this is entire
        ``explanation'').

        However, we still need to understand origin of assumptions; why
        should parameters, e.g.\ mass-to-light vary smoothly with
        luminosity?  Recall that this ratio includes dark matter.  If
        ellipticals have dark matter halos, the luminous inner parts are
        not required to be in virial equilibrium.

        There is relatively little scatter around the fundamental plane,
        implying that the assumptions are reasonably valid over a large
        range of elliptical properties, which implies some significant
        regularities in the galaxy formation process.

    \item Galaxies do not fully populate the entire plane defined by our
        relation.
        \begin{itemize}
            \item Consequently, when the plane is projected onto the other
                two axes, there is a correlation.\footnote{
                    Djorgovski figure 2
                }
            \item In the luminosity (size) --- $\sigma$ plane,
                Luminosity is proportional to velocity dispersion as
                $L\propto\sigma^{4}$,
                which is known as the \textbf{Faber-Jackson relation}.
                However, since the locus of Es isn't perfectly linear and
                the plane defied by the ellipticals isn't perpendicular to
                this dimension, the scatter around the Faber-Jackson
                relation is larger than the scatter around the fundamental
                plane. A new radius can be defined that incorporates SB
                such that the new radius vs.\ $\sigma$ views the
                fundamental plane edge-on. Such a size measurement is
                called $D_{n}$, the isophotal \emph{diameter} of the
                B=20.75 isophote. This $D_{N}-\sigma$ relation provides a
                very useful distance estimator (if the fundamental plane
                really is fundamental).
            \item In the SB --- size plane, relation for \emph{smaller} galaxies:
                \begin{itemize}
                    \item \emph{higher} SB for \emph{normal} small Es
                    \item \emph{lower} SB for \emph{diffuse} small Es
                \end{itemize}
                These are sometimes known as the \textbf{Kormendy
                relations} and are one of the main bases for separating
                these two types of objects.
            \item The SB --- $\sigma$ plane is presumably related to
                two underlying physical parameters:
                \begin{enumerate}
                    \item density
                    \item virial temperature
                \end{enumerate}
                In this plane, one can only form galaxies where
                \emph{cooling is effective}, i.e.\ at larger densities and
                hotter temperatures. This restricts the area in the space
                in which we can find galaxies.
            \item Additional features of the galaxy formation process may
                introduce additional restrictions into allowed locations of
                galaxies on the fundamental plane. Most luminous
                ellipticals are located along one line (with some scatter)
                in the fundamental plane, and most diffuse ellipticals are
                located along another.
            \item For more recent descussion, see
        \end{itemize}
    \item Isophotal deviations vs.\ luminosity and kinematics
        \begin{itemize}
            \item Deviations from elliptical shape is correlated with
                dynamics, such that slower rotators are more likely to be
                ``boxy'', and faster rotators are more likely to be
                ``disky''\footnote{Kormendy and Bender, figure 2}.
            \item There appears to be two kinds of ``boxiness'', one found
                in giant Es and one found in bulges, which in fact are
                relatively rapid rotators.  It is likely that these come
                from different origins: plausibly, mergers in the case of
                giant Es and bars/disk evolution in the case of bulges.
            \item The core properties are correlated with the global shapes
                and dynamics: cuspy core galaxies are boxy and anisotropic,
                cuspy galaxies are disky and rotating.
        \end{itemize}
    \item Several types of ellipticals??
        Normal and low luminosity Es:
        \begin{itemize}
            \item significant rotation
            \item nearly isotropic
            \item oblate spheroids
            \item cusps (no cores)
            \item disky isophotes
        \end{itemize}
        Giant Es:
        \begin{itemize}
            \item non-rotating
            \item anisotropic (triaxial)
            \item less flattened
            \item cuspy cores
            \item boxy isophotes
        \end{itemize}
        Dwarf spheroidals (diffuse ellipticals?):
        \begin{itemize}
            \item lower SB
            \item off the fundamental plane
            \item anisotropic (slow rotators for their ellipticity)
        \end{itemize}
        Bulges: multiple types? As an aside: globular clusters, which are
        likely to be significantly different.
    \item ellipticals as extension of disk galaxies to larger B/D ratio?
        see, e.g. Cappellari 2013 figure
\end{itemize}

\subsubsection{Spectral energy distributions}
Note typical features in ellipticals: stellar absorption lines
(4000 \AA{} break, Mg, Fe, etc.)
Es are generally red: implications are predominantly old population.
Relations:
\begin{itemize}
    \item Color-luminosity: more luminous Es are redder. However, as wel'll
        see, variations of color at this level can come from either metallicity
        or age, so interpretation isn't trivial.
    \item Mg line strenth-luminosity: more luminous Es have stronger lines.
        This correlation is even tighter when considering relation between
        central velocity dispersion and line strength (The Mg-$\sigma$ relation),
        over large range of $\sigma$. As we'll see, this variation could
        arise from variations in age or metallicity, or different heavy element
        abundance ratios. Of equal interest to the Mg-$\sigma$ relation is the
        quite small scatter around the relation.
\end{itemize}
Some Es have signatures of a younger population: E+A (or k+a) galaxies:
galaxies with strong Balmer lines. These are relatively rare, but likely
are indicative of post-starburst galaxies with significant star formation
1 Gyr ago. These are not associated with a cluster environment. Some show
morphological signs of recent interactions.

Stellar populations within ellipticals: Es have gradients in color/line
strength, i.e.\ tend to be redder toward the center (but again, there is
some degeneracy between age and metallicity as possible reasons for this).

\subsubsection{Interstellar matter in ellipticals} It is now known that Es
have significant interstellar gas, seen not in optical, but in X-ray
emission $\rightarrow$ hot gas. This makes sense since some gas is expected
from stellar evolution, although there is also the possibility of gas from
the environment.

Some Es aso show evidence of colder gas and dust. A surprisingly large
fraction ($\sim$ 50\%) show some evidence for dust their
cores\footnote{Lauer et al. 2005} These are typically small components by
mass.


\subsection{Spiral/Disk galaxies}

\subsubsection{SB profiles of disk galaxies}
Most spiral galaxies are a combination of a disk plus a spheroid. For the SB
profile, it is common to ``decompose'' spiral into two components: exponential
disk + Sersic bulge.
(e.g. MacArthur, Courteau, \& Holtzman, ApJ 582, 689 (2003)).
\begin{description}
    \item [disk] scale lengths are typically several kpc. Different families of
        disk profiles exist, e.g., Freeman type I and type II profiles, may be
        related to a bar or dust. Some spirals appear to have truncated disks,
        easier to see when observing disks edge-on, with truncation around 4
        disk scale lengths. Disks are not always perfectly flat, e.g. warps in
        the outer regions.
        More luminous galaxies may have thicker disks
    \item [bulge] Bulges in spirals tend to be better fit with variable Sersic
        indices, with lower index toward later type, smaller bulges; may be a
        distinction between ``classical bulges'' and bulges that may be related
        to bars and originating from the disk.
\end{description}
Issues with this decomposition:
\begin{itemize}
    \item covariance between parameters
    \item 1D vs. 2D
    \item validity of models (is there anything fundamental about an
        exponential that guarantees extrapolating it into the center?)
\end{itemize}
There appears to be some correlation between bulge scale length and disk scale
length (Macarthur, Courteau, Holtzman ApJ 582, 689 (2003)), so perhaps
components are not independent.

Spiral galaxies have stellar halos: in Milky Way  $\rho \propto r^{-3}$
\mynotes{(stellar mass density in halo?)};
in M31 stars have been found out to 200 kpc. Increasingly, stellar halos are
being recognized to have significant substructure, e.g., stellar streams in
NGC 891 and M31, Milky Way ``field of streams'' and some models.

Size-luminosity and surface brightness-luminosity relations: more luminous
spirals are larger, but still have higher surface brightness.

Vertical distributions in disk galaxies: disks are not always perfectly flat,
nor are they infinitely thin. To first order, disk thickness appears to be
independent of radius. Typically, vertical luminosity distribution is
characterized by an exponential:
\[
    I(z) \propto \exp{-\vert z \vert/z_{s}}
    \]
although sometimes a more general form is used to allow for some flattening
toward disk center, characterized by
\[
    I(z) = \mathrm{sech}^{2/n} \left( \frac{n \vert z \vert} {2z_{d}} \right)
    \]
Typical scale heights are roughly 0.1 $\times$ disk scale length

In the Milky Way, and some other external galaxies, a single exponential is
not a good fit, and a sum of two exponentials has been used, leading to the
idea of a thin plus thick disk. Thick disk also visible in velocity and
chemical space. Much debate about whether these are distinct components or
not, and origin of the thick disk, e.g. recent Bovy et al papers.

\subsubsection{Non-axisymmetric features in disk galaxies}
\paragraph{Bars}
\begin{itemize}
    \item (Milky way structure, Bland-Hawthorn and Gerhard, 2016 ARAA)
    \item Many disk galaxies (1/3 to 1/2) have bars, with various morphologies
        (rings, etc.).
    \item Typical axis ratios $\sim$ 2.5 to 5.
    \item Bars represent significant variations in mass density by factors of 2-3
    \item Seen edge on, bars may look like bulges, e.g. peanut shaped bulges.
    \item They likely form as result of dynamical instability in disk, though
        it's not totally clear whether this instability is caused by an
        internal or an external trigger.
\end{itemize}

\paragraph{Spiral arms}
\begin{itemize}
    \item Spiral arms come in a variety of morphologies: ``grand-design'' vs.
        multitarm vs. flocculent (from Elmegreen et al, ApJ 2011)
    \item Arms are more prevalent in blue light, but exist in red light and gas
        as well, so not entirely from young stars
    \item Overall mass contrast in spiral arms not so large
    \item Spiral arms are generally ``trailing"; spiral winds behind the
        rotation
\end{itemize}

\subsubsection{Kinematics}
Spiral galaxies are kinematically cold: the random motion of
stars is generally small compared to the organized motion (rotation).

\paragraph{Measuring galaxy rotation}
Rotation can be measured optically and using HI. HI data can be spatially
resolved (e.g.\ VLA), or unresolved.

While spirals are often
characterized by rotation curves (from Vogt el al 2004), velocity fields
are 2D:
\begin{itemize}
    \item several different ways (from Swaters et al. 2002) to look at HI
        kinematic data
    \item velocity-coded color (from NRAO)
    \item Spider diagrams
    \item PV diagrams: note spread of velocity at any given position: location
        of peak velocity may not give the maximimum velocity at that position
    \item From unresolved HI profiles, rotation characterized by line widths,
        e.g. $W_{50}$, $W_{20}$ (width at 50, 20 percent of peak line flux)
    \item Note that curves aren't always perfectly regular, can have features
        related, e.g., to structure in the disk: the nature of these can be
        clearer in 2D velocity data, e.g. NGC 5383. Note that there can also be
        motions of gas within the disk relative to rotation, and some internal
        velocity dispersion, e.g., UGC8508
    \item there can be large scale systematic deviations from rotation, e.g.,
        streaming motions in bars Milky Way rotation curve from Klypin et al
        2002.
\end{itemize}
The key kinematic quantity is the \textit{rotation curve}, the \emph{maximum}
rotation velocity as a function of radius. However, observed velocities need to
be corrected for inclination of galaxy, usually inferred from the observed axis
ratio (which can be harder to get for non-axisymmetric disks).

On average, the amplitude and shape of a rotation curve is related to luminosity,
but with significant scatter.
See some spiral galaxy rotation curves (from (Persic, Salucci, \& Stel 1996)).

Extension of rotation curve to large radius is the primary indicator of dark
matter.
\[
    v_{rot} \propto \sqrt{ \frac{G M(r)}{r}}
    \]
This is a proportionality because the geometry is not spherical. Flat rotation
curves thus imply  $M(r) \propto r$, i.e. significantly more than implied by
exponentially declining stellar component Galaxies are often characterized by
the maximum rotational velocity, $v_{max}$, (which is sometimes hard to
measure), or by the velocity at some specified radius, e.g. 2.2$R_{d}$, the
radius of maximum velocity for an exponentially declining mass distribution.

Stars have ``random'' velocities, characterized by their velocity ellipsoid, on
top of the rotation field. This is generally small, so hard to observe in
external galaxies, but it is known to exist, e.g. in the solar neighborhood
(Binney \& Merrifield, from Dehnen \& Binney 1998). Note that velocity
ellipsoid is not isotropic. Velocity dispersion increases for redder (older, on
average) populations, leads to decreased rotation velocity for these
populations, called ``asymmetric drift''. Traditional interpretation is that
velocity dispersion increases due to gravitational interactions with
substructure in disks, e.g. spiral arms, molecular clouds. Alternate
interpretation is that older stars are born with higher vertical velocity
dispersion, e.g., from a thicker disk: upside-down galaxy formation (although
heating would still occur subsequently).

\paragraph{Tully-Fisher relation}
Kinematics and luminosity are related by the \textit{Tully-Fisher relation} for
spiral/disk galaxies (Courteau et al 2007). Luminosity is proportional to the
maximum rotation velocity roughly by $ L \propto v^{n}$, where $n = 3 - 4$. The
slope depends on wavelength, definitions of where velocity is measured, etc.
There is relatively small scatter; useful as a distance indicator.
Scatter does not appear to be well-correlated with other parameters
(e.g. radius or surface brightness); different from ellipticals.
Various ``explanations'' for the scaling relations have been proposed
(see e.g., appendices in Courteau et al 2007), along the lines of what
we discussed for fundamental plane, with same caveats and limitations.
Scaling relations provide a strong constraint for galaxy formation models.

\paragraph{Typical spectra}
Disk galaxies show optical emission lines and generally blue continuum emission
lines from HII regions, indicating current star formation. Since young stars
dominate light, it is more challenging to learn about older stars. Most disks
have color gradients: age and/or metallicity gradients; emission lines suggest
metallicity gradients. and disk colors. bulges: Peletier and Balcells
observations suggest correlation of bulge and disk colors: bulges are not
always red. Again, do bulges and disks share some common origin?

\paragraph{Star formation - mass relation}
The ``main sequence'' from Peng et al. (2010), independent of environment;
specific SFR (SF per unit mass) nearly independent of stellar mass.
Here, SFR from H$\alpha$.

\paragraph{Gas and dust in disk galaxies}
Most gas is neutral hydrogen, observed in HI.
HI disks are very extended, often well past starlight.
Gas and dust are strongly confined to the plane of the galaxy.

Gas mass fractions can be significant, ranging from 5 to 80\%.
Neutral HI gas fractions (from Catinella et al
2010): higher gas fractions for
\begin{enumerate}
    \item lower stellar mass
    \item lower stellar mass surface density
    \item lower concentration
    \item bluer color
\end{enumerate}
Since gas can be a significant component by mass, especially for lower
mass/luminosity galaxies, the \emph{baryonic} Tully-Fisher relation (from
Gurovich et al 2010, left is stellar mass, right includes gas mass), has been
considered. This seems to show even less scatter than the stellar T-F relation;
note that this requires estimate of stellar mass from luminosity (more later).

Gas composition in galaxies also varies from galaxy to galaxy and within
galaxies. Nonetheless one can consider ``mean'', or alternatively present-day,
metallicities of galaxies. Find that more luminous galaxies generally are more
metal-rich, as manifested by a luminosity-metallicity relation, or a
mass-metallicity relation.

\subsection{Summary: Galaxy parameter (scaling) relations}
Galaxies exhibit some regularities, i.e., systematic correlations of different
global properties, all of which need to be reproduced/understood in any model
for galaxy formation and evolution.
\begin{description}[style=nextline,labelindent=0.2in]
    \item [Structural] luminosity-SB, or luminosity-size ($L-r_{e}$)
        \begin{itemize}
            \item isophotal shape-luminosity relation for Es
                (boxy vs. disky).
        \end{itemize}
    \item [Kinematic] Faber-Jackson, Tully-Fisher
        \begin{itemize}
            \item luminosity-rotation relation for ellipticals
                \mynotes{The Faber-Jackson relation also applies to
                ellipticals, but this is the relation between luminosity
                and velocity dispersion, not rotational velocity\ldots?}
        \end{itemize}
    \item [Structural/kinematic] Fundamental Plane
        \begin{itemize}
            \item A global generalization to all galaxies: the fundamental
                manifold\footnote{Zaritsky et al. (2008)}, with an
                assumption about how to combine rotation and velocity
                dispersion to get a characteristic ``velocity'', all galaxies
                appear to lie in a plane.
        \end{itemize}
    \item [Stellar populations] color-magnitude, Mg-$\sigma$ for Es
    \item [Gas] HI-SB, Luminosity-metallicity
    \item [Black holes] M$_{BH}-\sigma$
        \begin{itemize}
            \item Central black hold seems to be ubiquitious in galaxies
            \item BH masses can be estimated (with difficulty; see later).
                More luminous (spheroid) galaxies appear to have more
                massive black holes.
            \item Tighter relation exists with velocity
                dispersion:M$_{BH}-\sigma$ relation
                (or M$_{\mathrm{spheroid}}$)\footnote{Kormendy and
                Bender, 2001}. Note that black hole mass seems to be
                more linked to spheroid than to total galaxy mass
                (although claims to the contrary have been made).
        \end{itemize}
\end{description}


\subsection{Environments of galaxies: clusters and cluster galaxies}
Galaxies are not homogeneously distributed in space, e.g. SDSS pie diagrams: to
z=0.15, to z=0.6, to z=5. Clustering is often described by the \textit{two
point correlation function}, which gives the excess number of galaxy pairs over
a random distribution as a function of radius.
\[
    \mathrm{d}P = n \left( 1 + \xi(r) \right) \;\mathrm{d}V
    \]
Typically, the observed correlation function is well fit by a power law:
\[
    \xi(r) = \left( \frac{r}{r_{0}} \right)^{-\gamma}; \qquad
    r_{0} \sim 5h^{-1}\; [\mathrm{Mpc}]
    \]
so at 5 Mpc separation, there are twice as many galaxies as there are for a
random distribution. The amplitude of the correlation function depends on
both the absolute magnitude and the color (see, e.g.\
Zehavi et al 2005, Figs 6 and 15).

Many galaxies reside in groups or clusters of galaxies, systems that are a few
Mpc across.

Groups:
\begin{itemize}
    \item Typically contain 3-30 relatively luminous galaxies (cf., the Local
        Group). \mynotes{``cf.'' = ``to bring together''\ldots see wikipedia}
    \item Typical sizes are $\sim$ 1-2 Mpc, but note the existence of compact
        groups, which have several massive galaxies in close proximity (perhaps
        interacting)
\end{itemize}

Clusters:
\begin{itemize}
    \item Contain dozens to hundreds of galaxies
\end{itemize}

Probably best known catalog of local galaxy clusters is Abell cluster, with richness classes 1 to 5, based on number of galaxies in cluster.

Galaxies in galaxy clusters (see e.g.Boselli and Gavazzi, PASP 118, 517 (2006))
\begin{itemize}
    \item morphology-density relation. (related, color-density relation, Baldry
        et al 2006) Note this applies to both high and low luminosity galaxies.
        (See Peng et al, e.g. Fig 6 for more recent summary?)
    \item S0 galaxies
    \item Anemic spirals: spirals with significantly lower gas fractions
        ($\sim$ 50\%) than comparable field spirals. Gas deficiency appears
        largely in HI (not molecular), and preferentially in outer regions of
        galaxies. However, SFR is also notably lower than in field.
\end{itemize}

Galaxy clusters contain distributed hot gas, with total mass comparable to that
found in galaxies.
\begin{itemize}
    \item observed with X-ray observations;
    \item observed with X-ray observations;
    \item intracluster gas is enriched in heavy elements, so it is not primordial gas
    \item X-ray observations useful for probing cluster masses, under assumption of hydrostatic equilibrium
        \begin{itemize}
            \item Most accurate determination requires measurement of density
                and temperature profiles of X-ray gas
            \item Mass estimates can be made from X-ray luminosity and/or
                temperature (better): the $M-kT$ relation.
            \item typical masses are $10^{14}$ to $10^{15}$ solar masses; a
                typical cluster mass function (from Wen et al. 2010)
        \end{itemize}
\end{itemize}
Galaxy clusters also have dark matter halos beyond those associated with the
galaxies.

Shapes of clusters, inhomogeneities: not all clusters are in equilibrium.


\subsection{Some galaxies to be familiar with}
\begin{itemize}
    \item Local group, see census of dwarfs in Mateo annual review 1998. Spirals: MW, M31, M33 (note NGC 604). M32. LMC (note 30 Dor/R136) / SMC. dIrrs, dSphs. See some pictures here, also here; c.f. spheroidal companions of M31, e.g. here. Local morphology-density relation
    \item Local neighborhood. See various pictures here, e.g., M81 group: M82 (dwarf starburst). Sculptor group: NGC 253 (nearest nuclear starburst). Centaurus group: Cen A (NGC 5128). Maffei group (Maffei 1). Leo group (NGC 3379/M105).
    \item Some other famous active galaxies: NGC 1068, NGC 4151, 3C273
    \item Nearest clusters. Virgo: M87, NGC 4472. Fornax. Coma.
    \item Large scale structure: Shapley supercluster
\end{itemize}


\newpage
\begin{center}
    \fontsize{20}{22}\selectfont\textbf{Part II: The building blocks of galaxies}
\end{center}

\section{Stars and Stellar populations}
How to relate observables to intrinsic population characteristics:
\begin{description}[labelwidth=14em, leftmargin=18em]
    \item [Population characteristics:] distribution of masses, compositions
        and ages (star formation history).
    \item [Stellar SEDs] (or colors) can be observed for
        individual stars in very nearby galaxies, or integrated starlight
        in most galaxies.
\end{description}

\subsection{Review: stellar evolution}
The \textbf{Russell-Vogt theorem} states that the
internal structure of stars is determined by mass, chemical
composition, and age. (This excludes
non-spherical symmetric effects, such as rotation, magnetic
fields, and binarity).

Luminosity (radius) and effective temperature are derived from
equations of stellar structure:
\begin{itemize}
    \item mass conservation
    \item hydrostatic equilibrium
    \item energy equation
    \item energy transport
\end{itemize}
along with auxiliary relations:
\begin{itemize}
    \item equation of state
    \item opacity
    \item nuclear reaction rates
\end{itemize}
\subsubsection{Main stages of stellar evolution}
\begin{itemize}
    \item Hydrogen core burning: main sequence (MS)
    \item Hydrogen shell burning: giant branch (for lower mass
        stars)
    \item Helium core burning:
        \begin{itemize}
            \item horizontal branch (low-mass stars)
            \item red clump (intermediate-mass stars)
            \item blue core helium burners (high-mass stars)
        \end{itemize}
        Note key transition around 2 M$_{\odot}$
        (depends on metallicity): shift to helium
        flash at lower masses.
    \item Helium shell burning: Asymtotic Giant Branch (AGB)
    \item Other nuclear burning for high mass stars
    \item White dwarf or supernova
\end{itemize}
Stellar evolution gives \textbf{model tracks}, evolution as a function of
time for a given mass. For spherical symmetry, calculations are 1D.

\subsubsection{Isochrones}
An \textit{isochrone} is a cross-section of properties at a
fixed time across a range of masses.
\mynotes{(Stars that were born from the same collapsing cloud, and are therefore
the same age.)}
Some well-known groups that calculate evolutionary tracks/isochrones:
\begin{itemize}
    \item Padova
    \item BASTI (Teramo)
    \item Dartmouth
    \item Yale-Yonsei
    \item Victoria-Regina
    \item Geneva
\end{itemize}

Although fairly well-understood, there are still some
uncertainties that lead to differences between different
calculated isochrones: convective overshoot, diffusion, convection,
helium abundance, mass loss, etc. Generally, uncertainties are larger
for later stages of evolution. Additionally, there may be missing
physics, e.g.\ rotation and magnetic fields, that would require a full
3D treatment.

Given effective temperature, surface gravity (from mass and radius), and
composition, stellar atmospheres give observables: spectral energy
distribution/colors. Some model atmospheres:
\begin{itemize}
    \item {Kurucz}
    \item {MARCS}
\end{itemize}

Theoretical Color-Magnitude Diagrams (CMDs) $\rightarrow$ observed; need
distance and reddening/extinction Age effects (model isochrones) from Yi et al
2001.  Metallicity effects: internal (opacity) and atmosphere (line blanketing)
effect combine in the same direction to make more metal-rich populations
redder: CMDs
\begin{itemize}
    \item Metallicity terminology: often given as mass fractions of
        hydrogen (X), helium (Y), and heavier elements (Z).
    \item solar abundance: X=0.7, Y=0.28, Z=0.019 (roughly)
    \item also given by
        \[
            \left[\frac{\textrm{Fe}}{\textrm{H}}\right] = \log\left[
            \frac{\left( {\textrm{Fe}}/{\textrm{H}} \right)}
            {\left({\textrm{Fe}}/{\textrm{H}}\right)_{\odot}}
            \right]
        \]
    \item Be aware that this is an oversimplification, as Z contains
        lots of different elements. More later.
\end{itemize}

Main sequence (MS)
\begin{itemize}
    \item Very rough scaling relation between luminosity and mass:
        $ L \propto M^{3.5} $
    \item Main sequence shifts with metallicity: redder for
        higher metallicity
    \item Location also depends on helium abundance
\end{itemize}

Red Giant Branch (RGB)
        \begin{itemize}
            \item note that because of more rapid evolution after MS,
                RGB stars all have roughly the same mass.
            \item temperature of RGB depends on age/mass: younger and more
                massive stars are hotter. Temperature also depends on
                metallicity: latter is dominant effect in older populations
                (greater than 5 Gyr).
            \item At lower masses/larger ages, tip of RGB is close to
                constant bolometric luminosity regradless of age or
                metallicity. In observed plane, leads to roughly fixed
                tip luminosity \emph{if} observing at long wavelengths
                (e.g.\ I band): basis for Tip of Red Giant Brance (TRGB)
                distance indicator.
            \item RGB bump (RGBB from
                Bono et al 2001),
                location of which depends on mass/age and metallicity;
                arises when H burning shell crosses chemical discontinuity.
        \end{itemize}
Horizontal Branch (HB), aka.\ Red Clump (RC) or more generally
        He core burning sequence
        \begin{itemize}
            \item High mass stars form blue He core-burning branch
                (also note red plume)
            \item Intermediate mass stars form red clump.
            \item Low mass stars form horizontal branch. Variable mass loss
                on RGB and at He flash gives a range of masses.
            \item Horizontal branch morphology depends on metallicity: more
                metal-poor populations have bluer HB.
            \item However, there is something else that also affects HB
                morphology, leading to the so-called second-parameter problem
                (e.g.\ {M3/M13}, from {Rey et al
                2001}). Possibilities: Age? He abundance? Heavy element
                abundances? Density? Rotation?
            \item RR Lyrae stars: in instability strip (caused by doubly
                ionized He, also includes Cepheids, delta Scuti stars, etc.)
                RR Lyrae stars are indicators of old metal-poor population.
                Periods of 0.5 days, but multiple groups (Oosterhoff classes)
                depending on stellar parameters$\ldots$
        \end{itemize}
Asymtotic Giant Branch (AGB)
        \begin{itemize}
            \item For intermediate masses/ages, AGB is significantly more
                luminous than the TRGB, hence of potential critical importance
                to studies of integrated light.
            \item For lower masses (older populations), AGB tip comparable
                to TRGB and AGB asymtotically approaches RGB (hence its name).
        \end{itemize}
Potential importance of binaries/interactions
        \begin{itemize}
            \item unresolved (but otherwise non-interacting) binaries:
                broaden sequences, depending on mass ratios: equal masses
                give the appearence of a star 0.75 mag brighter.
            \item interacting binary stars
            \item blue stragglers: possible stellar merger/interaction products?
                {M3 example} from Sandage 1953.
            \item supernovae type SNIa: arise from binaries, produce different
                heavy element abundances than core collapse SNe. (e?)
        \end{itemize}
End stages of stellar evolution:
        WDs, neutron stars, black holes, and supernovae
        \begin{itemize}
            \item {projenitor-final mass relation}
                (from Binney and Merrifield)
            \item {White dwarf cooling sequence}
                (from {Hansen et al 2007}: potential for
                age dating).
            \item Supernovae: generate significant fraction of heavy elements
                (but not all). Significant energy input, thermal and mechanical.
        \end{itemize}
Galactic globular clusters: cornerstone of understanding
      stellar evolution historically, as apparent examples of a
      ``simple stellar population (SSP)'', with all stars of the same
      age and abundance.
      \begin{itemize}
          \item However, it's not recognized that not all GCs are so simple.
          \item Some CMDs show clear evidence of multiple components, e.g.\
              {NGC2808}.
          \item Long history of evidence of abundance variations between
              different stars, typified by the Na-O anticorrelation;
              demonstrated through observations of main sequence stars that
              this is not a mixing effect.
          \item Two phenomena have recently been coupled, but not yet a clear
              understanding of how the multiple populations arise.
      \end{itemize}

\subsection{Simple Stellar Populations and the Initial Mass Function}
There is more information in a CMD than is represented by isochrones;
one can also consider the relative \textit{numbers} of stars at
different locations. CMDs that incorporate number of stars are
reffered to as \textbf{Hess diagrams} (e.g.\
\href{http://astronomy.nmsu.edu/holtz/a555/resources/fornaxhess.gif}
{Fornax} from Battaglia et al., 2006).

For a \textbf{simple stellar population (SSP)}, a population with a single age
and metallicity, the relative number of stars at each stage is
determined by the \textbf{initial mass function (IMF)} and the age.

IMF determinations and parameterizations:
\begin{itemize}
    \item ``classical'' \textbf{Salpeter} IMF (power law with
        $ dN/dM \propto M^{-2.35}$) and others
        (from {Pagel}). Note power law form:
        $ dN/dM \propto M^{\alpha}  $,
        or alternatively,
        $ dN/d\log{M} \propto M^{\Gamma} \propto M^{\alpha + 1} $
    \item Widely used determination of local IMF is by
        {Droupa Tout, and Gilmore 1993}, who find
        $$ dN/dM \propto M^{-2.7}\ \textrm{for}\ M > 1M_{\odot} $$
        $$ dN/dM \propto M^{-2.2}\ \textrm{for}\ 0.5 < M < 1M_{\odot} $$
        $$ dN/dM \propto M^{-1.3}\ \textrm{for}\ M < 0.5M_{\odot} $$
    \item Chabrier IMF: log-normal form,
        $ dN/dM \propto \exp(\log{M}-\log{M_o})^2  $
    \item Empirically, no strong evidence for variations of the
        IMF as functions of, e.g., metallicity.
    \item No well-established theory for predicting the IMF
\end{itemize}

\subsection{Resolved stellar populations}
In general, galaxies are not SSPs; their CMDs don't look like those of a
cluster. Hess diagrams of resolved stellar populations can be fit with
\emph{combinations} of SSPs to derive \emph{constraints} on SFH, e.g.\
\href{http://astronomy.nmsu.edu/holtz/a555/resources/modelpops.gif}
{simulated galaxies} (from \href{http://adsabs.harvard.edu/abs/2009ARA\%26A..47..371T}
{Tolstoy et al.}).

Generally speaking, we want the star formation history, represented by
$SFH(t,Z,M)$. This gives the number of stars (or stellar mass)
at all combinations of age ($t$), metallicity ($Z$), and mass ($M$).
Because there has been no variation observed in the IMF, it is usually
separated from the rest of the SFH:
\[
    SFH(t,Z,M) = \xi(M)\psi(t,Z)
    \]
where $\xi$ is the IMF.

MW neighbors resolved down to oldest MS turnoff ($M \sim 5$),
because typical distances give distance modulii $\lesssim$ 20.
M31 and neighbors somewhat shallower ($m-M \sim 24.5$)
without large investment of telescope time.

\subsubsection{Results}
\paragraph{MW}
\begin{itemize}
    \item {solar neighborhood Hipparcos sample}: roughly constant {SFH}, but be
        aware of dynamical effects and limited volume bias nearby sample to
        younger populations. When corrected for, star formation history in
        solar neighborhood likely to have significantly declined (see, e.g.\
        {Aumer \& Binney 2009}).
    \item Age of oldest stars can be studied by WD sequence, gives
        $t_{\textrm{oldest}} > 8$ Gyr. Also note presence of disk RR Lyrae
        stars ($t > 10$ Gyr??)
    \item Might consider studying age distribution of clusters, as they are
        simpler populations, but note problems with disruption of clusters that
        would lead to a bias to younger clusters.
    \item Bulge can also be studied, but note foreground population and
        extinction issues. Predominantly old population, e.g.\ {bulge CMD}, and
        {luminosity function} (from {Ortolani et al 1995}).
    \item Halo (as defined kinematically/spatially) also prdominantly old, both
        from field population and from globular cluster population.
    \item Historically, distinction between disk stars/open clusters
        (population I) and halo stars/globular clusters (population II), with
        pop I being younger and more metal rich. Note, however, the pop II
        association with low metallicity is now recognized not to be
        fundamental; inner halo/bulge significantly more metal rich.
    \item There is \emph{not} a strong age-metallicity relation in the solar
        neighborhood (Orion has roughly solar metallicity despite the fact that it is nearly
        5 billion years younger).
        Radial migration, inhomogeneous ISM, inflow\ldots?
\end{itemize}

\paragraph{Local Group (LG) dwarf galaxies}
\begin{itemize}
    \item Carina dSph: striking evidence of episodic star formation, but this is NOT characteristic.  \item Others (from Tolstoy et al); note range of SFHs.
    \item LG dIrrs compilation (from Dolphin et al 2005)
    \item LG dSphs (from Dolphin et al 2005); possible similarity to dIrrs
        apart from lack of recent SF?
    \item Many dwarfs show population gradients, e.g.\ from RGB/HB morphology
        (e.g., Harbeck et al 2001), likely metallicity gradients: some show
        evidence of age gradients, with younger population in center.
\end{itemize}

\paragraph{M33}
(from Holtzman et al 2011); note significant age
gradient, and significant component of younger stars.
\paragraph{M31 halo}
(from Brown et al 2006); significantly different from
MW bulge, with more of an intermediate-old population?

Can do more distant galaxies if information from advanced stages (RC, HB) is
reliable; (e.g.\ local volume SFH (from Williams et al, in prep), dwarf SFH
(from Weisz et al 2011)).

Even farther if info from AGB is reliable.

\subsubsection{General conclusions}
\begin{itemize}
    \item Galaxies are not SSPs.
    \item Galaxies have gradients in their stellar populations
    \item Significant population of old stars in nearby galaxies?
    \item More massive galaxies formed stars earlier??
\end{itemize}

\subsection{Unresolved stellar populations}
\subsubsection{What stars contribute the most light?}
\begin{itemize}
    \item Along the MS, use combination of M-L relation and IMF:
        \[
            L \propto \left(M^{-2.35}M^{3.5}\right) \propto M^{1.15}
            \]
        Massive stars dominate light.
    \item Compare MSTO with evolved population: stars on the RGB have almost
        the same mass as those on the MSTO, but they are significantly more
        luminous.  The relative contributions depend on the relative
        \emph{number} of stars along the giant branch compared to the MSTO
        stars, but ultimately it is the luminous evolved populations that
        dominate the total luminosity.
        \href{http://astronomy.nmsu.edu/holtz/a555/images/renzini5.htm}
        {(Renzini fig 1.5)}.
    \item Since post-MS evolution is fast compared to MS evolution for all
        masses, it is true that at any given time when the most luminous stars
        are the most evolved stars, the luminosity is given predominantly by
        stars of (nearly) a single mass (true for all except youngest ages, see
        \href{http://astronomy.nmsu.edu/holtz/a555/images/renzini1.htm}
        {Renzini 1.1}) and
        \href{http://astronomy.nmsu.edu/holtz/a555/resources/iso.html}
        {luminosity vs.\ lifetime plots}.
\end{itemize}
\subsubsection{Integrated brightness}
Do SSPs get brighter or fainter as they age?
\begin{itemize}
    \item At younger ages, have MS ``peel-off'' effect.
        \mynotes{As a cluster ages, the MSTO gets redder and fainter,
        since massive, bright stars (O, B, etc.) evolve off the MS first.}
        Also supergiants\mynotes{$\ldots$?}
    \item At older ages, have competing effects of IMF and rate of
        evolution.
        \begin{itemize}
            \item To determine the luminosity of a stellar population, consider
                the number of stars evolving off the MS (dying)
                which is given by the \textit{evolutionary flux}:
                \[
                    b(t) = \psi\left(M_{\textrm{TO}}\right)
                    |\dot{M}|\quad[\textrm{stars\;yr}^{-1}]
                    \]
                where TO stands for turnoff, $\psi$ is the IMF, and $|\dot{M}|$
                is the time derivative of the turnoff mass, e.g.\ Renzini 1.1.
                \textbf{The key point is that lower mass stars evolve more slowly.}
            \item The IMF is important, along with age, in
                determining the luminosity evolution of a galaxy.
            \item Once most massive stars have died,
                all stars of lower mass reach comparable luminosity (tip of
                RGB), hence \textbf{luminosity of population depends on number
                of red giants.}
            \item For a typical IMF, the rate at which the number of stars
                increases toward lower masses is slower than the rate at which
                the generation of turnoff stars decreases.
            \item The luminosity of the turnoff also has some smaller effect.
        \end{itemize}
\end{itemize}

\subsubsection{Integrated colors}
Do integrated colors depend on the IMF?
\begin{itemize}
    \item Need to know the relative contributions of each
        stage of evolution, i.e., from stars of different masses.
    \item Since all post-MS stars have nearly the same mass, the number of
        stars in each post-MS stage is determined predominantly by the time
        spent in each stage:
        \[
            N_{j} = b(t)t_{j}
            \]
        So although the total flux is sensitive to the IMF, the relative
        contributions of each stage are not sensitive to the IMF, at least for
        older populations
        \href{http://astronomy.nmsu.edu/holtz/a555/images/renzini5.htm}
        {(Renzini fig 1.5)}.
    \item For all except the youngest population, the later stages of evolution
        provide a majority of the light.  \textbf{Evolved stages are nearly all
        the same mass.}
    \item \textcolor{bred}{Consequently, the integrated spectrum and the
        relative contributions of various evolutionary stages are nearly
        \emph{independent} of the IMF\@.} They \emph{do} depend on the age and
        metallicity.
\end{itemize}

\subsubsection{Dependence of color and luminosity on time}
See \href{http://astronomy.nmsu.edu/holtz/a555/images/bc1.htm}
{figure 1} from \href{http://adsabs.harvard.edu/abs/1993ApJ...405..538B}
{Bruzual and Charlot (1993)}
for a SSP (single burst) normalized to one solar mass.
\begin{itemize}
    \item Luminosity variation is a combination of IMF plus
        rate of evolution for lower masses.
    \item Color evolution comes from a changing mix of stellar population, but
        this is just for a single (solar) metallicity.
    \item RSG phase, followed by subsequent dimming and reddening.
    \item Note that this is bandpass dependent, with less dimming
        at longer wavelengths (at older ages where the RGB is dominant).
\end{itemize}
\textbf{Implication:} if a range of ages is present, then integrated light
is (significantly) weighted toward younger populations.

\subsubsection{What stars contribute the most \emph{mass}?}
\begin{itemize}
    \item For all measured IMFs, the bulk of the mass is in relatively
        low mass stars ($< 1 M_{\odot}$).
    \item Relative stellar M/L ratios for different IMFs can differ
        by a factor of two or more. For example, the Salpeter IMF has much more
        mass the IMFs with low mass turnover or cutoff.
\end{itemize}

%\fbox{\parbox{\textwidth}
\begin{framed}
    \textbf{Exercises}
    \begin{itemize}
        \item consider four populations with combinations of alpha=-2.35,-1.35,
            age=3,10 Gyr. How do relative brightnesses compare?
            How do relative colors compare?
        \item now consider alpha=-2.35,-1.35 with contiuous SF from 3-10 Gyr ago.
            How do brightnesses and colors compare?
        \item how do stellar masses compare?
        \item consider two galaxies with L* and 2 L*, with same IMF and age?
            How do stellar masses compare? What if IMFs are different?
            What if ages are different?
    \end{itemize}
\end{framed}

\subsubsection{The mass-to-light ratio (M/L)}
Can you estimate stellar mass from integrated light?
\begin{itemize}
    \item Consider the \emph{stellar} M/L
        (not to be confused with \emph{total} M/L, which
        includes dust, gas, and dark matter, in addition to stars).
        The stellar M/L is defined in solar terms:
        \[
            \left(\frac{M}{L}\right)_{*} \equiv
            \frac{M_{*}/M_{\odot}}{L_{*}/L_{\odot}}
            \]
    \item For individual stars, M/L decreases as mass increases
        ($L \propto R^{2}$). So younger populations will have a lower M/L
        than older populations.
    \item \textbf{The absolute value of M/L of a population depends strongly
        on the IMF.} Relative values characterize different stellar populations
        for a fixed IMF.
    \item The stellar M/L depends on luminosity bandpass;
        there is less variation toward near-IR bandpasses
        (less sensitivity to younger stars), but still not insigificant.
        K-band luminosity is sometimes used as a proxy for stellar mass, but
        see \href{http://astronomy.nmsu.edu/holtz/a555/resources/belldejong.html}
        {Bell \& deJong results} from
        \href{http://adsabs.harvard.edu/abs/2001ApJ...550..212B}
        {Bell \& deJong (2001)}.
        \begin{itemize}
            \item More variation in stellar M/L at shorter wavelengths
            \item Stellar M/L does seem to be well correlated with color,
                however, this assumes no systematic variations in the IMF.
            \item Some variations from metallicity
            \item Some variations from reddening, although reddening and
                extinction act in opposite directions.
        \end{itemize}
    \item So, it is possible to get some information about stellar M/L given
        colors, but the level of uncertainty depends on bandpass and color
        information available.
        \begin{itemize}
            \item Variety of different stellar M/L techniques in use, e.g.
                using different colors, or spectral energy distributions,
                different assumptions about star formation history
            \item Some uncertainties in relative M/L for different populations
                even with colors
            \item Larger uncertainty in absolute value of M/L because of IMF
            \item if IMF were to be variable, then even relative M/L would be
                challenging
        \end{itemize}
\end{itemize}

\subsubsection{Star formation histories from integrated colors}
\begin{itemize}
    \item Issue: intgrated colors are affected by combination of age
        distribution,
        metallicty distribution, and reddening distribution.
    \item \href{http://astronomy.nmsu.edu/holtz/a555/resources/int.gif}
        {color variations with age}: note significant degeneracy between age
        and metallicity. Both older populations and metal-rich populations
        are redder. However, only young populations have very blue colors.
    \item The situation is somewhat improved with a longer wavelength
        baseline (e.g.\ IR colors), but still significant.
    \item Even if age-metallicity were resolved, there is still an issue
        of luminosity weighting for non-SSPs.
\end{itemize}

\subsubsection{Integrated spectra}
Given age-metallicty degeneracy from colors, measurements of spectral features
might help. One can consider a full spectral synthesis to get an ingegrated
spectrum for a population. Some existing evolutionary synthesis
compilations/codes:
%\begin{itemize}
%\end{itemize}

Spectral evolution (BC fig 4) for a variety of SFRs
%\begin{itemize}
%\end{itemize}

Dependence of contributor on wavelength

Potential problems with synthetic integrated spectra:
%\begin{itemize}
%\end{itemize}

What about spectral features?
\begin{itemize}
    \item For populations
    \item For older populations
    \item typical optical spectra
\end{itemize}

%  page 9
Even with
%\begin{itemize}
%\end{itemize}

Main area of application
%\begin{itemize}
%\end{itemize}





\subsection{Chemical evolution}
Understanding chemical evolution provides another clue about
the formation history of galaxies. Since stars generate this evolution,
one can in principle learn something about star formation histories by
looking at the distribution of compositions. The more we know about
metallicities, the better we can constrain ages. The \emph{distribution}
of metallicities within galaxies (recall gradients) potentially provides
important information about the mode of galaxy formation.
We'd like self-consistency between SFH and chemical abundances.
Potential information about gas entering and leaving system. Relative abundances of different
elements also may provide clues about formation history.

\subsubsection{Basic picture of chemical evolution}
The basic picture of chemical evolution starts with some initial conditions,
then add SFH: (IMF + SFR). This results in chemical enrichment for stars as a
function of mass and metallicity. Also may wish to consider input and output of
gas from any particular region, with either primordial or modified composition.
\begin{itemize}
    \item ICs: No heavy elements and pure gas (no stars).
    \item SFH: birthrate function $\psi(m,t)$ gives number of stars formed with
        mass $m$ per unit volume, usually split into a separable function:
        \[
            B(m,t) = \xi(M)\psi(t)
            \]
        where $\xi(M)$ is the IMF and $\psi(t)$ is the SFR\@. This implicitly
        assumes an IMF which is constant in time, which may or may not be true,
        but so far, there is no strong observational evidence against it
        (although note pop III issues).
\end{itemize}
For any particular star, use stellar evolution to compute the amount and
composition of mass returned to the ISM vs.\ amount locked up in stellar
remnants (Pagel fig 2, from Maeder 1992).  Note that the \emph{rate} of return
depends on stellar mass, especially for the case of SnIa, which also depends on
the binary fraction.

\textbf{\underline{nucleosynthesis}}: main element groups and their sources:
\begin{itemize}
    \item light elements: Big Bang Nucleosynthesis (BBN)
        plus subsequent destruction (?)
    \item $\alpha$ elements (even Z from O up\mynotes{\ldots???}):
        massive stars (core collapse SN)
    \item Fe-peak elements: type Ia SN and core collapse SN
    \item elements from \textit{slow} ($s$) and \textit{rapid} $r$ processes
        (neutron capture, plux beta decay):
        core collapse SN, AGB stars\ldots (need neutron capture diagram here)
\end{itemize}
Note that infall might arise from primordial clouds, or from processed
material, e.g.\ mass-loss from halo stars. Outflow might come from SN winds,
and in this case, it's possible that the composition of outflow material might
be more enriched than the typical composition at any given time. So things can
get complicated.

\subsubsection{Basic equations of chemical evolution:}
Allowing for all sorts of realistic effects makes it difficult to make
very simple predictions for the evolution of abundances. However, under
some simplifying assumptions, it is possible to do so. A simple chemical
evolution model serves as a useful baseline against which observations
can be compared to determine where the assumptions may break down.
In addition, understanding a simple model allows us to introduce and
become aquainted with terminology that is in widespread use.

Consider a total (galaxy) mass $M$, split into a gas mass
$g$ and stellar mass $s$. For a simple model, consider the
gas at any time to be well mixed. At any given time, we wish
to know the fraction abundance ($Z_{i}$) of element $i$.
Consider an inflow rate into the system, $F$ and outflow
(ejection) rate, $E$. Then we have:
\[
    M = g + s
    \]
\[
    \frac{\mathrm{d}M}{\mathrm{d}t} = F - E
    \]
\[
    \frac{\mathrm{d}g}{\mathrm{d}t} = F - E + e - \psi't
    \]
where $\psi'$ is the SFR in units of mass per time,
\[
    \psi' = \psi\int\xi\textrm{d}m\quad[\mathrm{g\;s}^{-1}]
    \]

(so $\xi$ is normalized to a total of one), and $e$ is the
ejection rate of mass from stars,
\[
    e = \int_{m_{t}}^{m_{U}}{
        \left(m-m_{\mathrm{rem}}\right)\psi\left(t-\tau(m)\right)
        \xi\left(m\right)\mathrm{d}m}
    \]
which is a sum over all stellar masses of the product of the SFR at the time of
formation of each mass with the mass returned to the ISM, weighted by the
IMF\@. The lower limit of integration is the stellar mass that is dying at
time $t$; lower mass stars don't contribute because they haven't ejected any
mass yet. We also have
\[
    \frac{\mathrm{d}s}{\mathrm{d}t} = \psi' - e
    \]
since the mass in stars increases by the number of stars formed, but decreases
by the amount of mass lost back to the ISM from the previous generation of
stars. For the elements, we have
\[
    \frac{\mathrm{d}(gZ_{i})}{\mathrm{d}t} =
    e_{Z}(i) - Z_{i}\psi + Z_{F}F - Z_{E}E
    \]
($Z_{i}$ is the mass fraction of element $i$; hereafter we'll drop
the subscript for simplicity). The mass in each element \emph{increases} by:
\begin{itemize}
    \item amount of mass released by previous generations
    \item amount of mass added by inflow
\end{itemize}
but \emph{decreases} by:
\begin{itemize}
    \item amount of mass locked up in new stars
    \item amount of mass lost to outflow
\end{itemize}
The term $e_{Z}$ is given by:
\[
    e_{Z} = \int_{m_{\tau}}^{m_{U}}{
        \left[(m-m_{rem})Z(t-\tau(m))+mq_{Z}\right]
        \psi(t-\tau(m))\xi(m)\textrm{d}m}
    \]
where $q_{Z}$ represents the fractional mass of element $Z$ synthesized and
ejected during stellar evolution (so left term in brackets gives material which
returns unprocessed and right term gives newly synthesized contribution).  In a
simple model, the synthesized masses are independent of the metallicity of the
population (although we know this is not true for some elements, more on this
later).

To simplify, consider an approximation to these formulae called the
\textit{instantaneous recycling approximation} which assumes that all elements
are returned instantaneously - good for products of massive stars, but less
good for products from lower mass stars, e.g.\ iron. Then we have
\[
    e = \psi\int{(m-m_{\textrm{rem}})\xi(m)\textrm{d}m}
    \equiv (1-\alpha)\psi
    \]
where $\alpha$ is the \textit{lock-up fraction}: the fraction
of mass ``locked up'' in stars and remnants.
\[
    \alpha = 1-\int_{m_{\tau}}^{m_{U}}{{m'}\xi(m')\mathrm{d}{m'}} +
    \int_{m_{\tau}}^{m_{U}}{m_{\mathrm{rem}}(m)\xi(m)\mathrm{d}m}
    \]
where the second term is the total mass in stars that have died, and
the third term is the mass from those stars that have been locked up
in remnants.

Alternatively, one can think of the \textit{Return fraction},
$R=1-\alpha$. This gives
\[
    \frac{\mathrm{d}g}{\mathrm{d}t} = F-E+(1-\alpha)\psi-\psi=F-E-\alpha\psi
    \]
We also have
\[
    e_{Z} = (1-\alpha)Z_{\psi} + \psi \int_{m_{L}}^{m_{U}}{
        mq_{Z}\xi(m)\mathrm{d}m}
    \]
We define the stellar \textit{yield}, $p_{Z}$ to be
\[
    p_{Z} = \frac{\int_{m_{L}}^{m_{U}} {mq_{Z}\xi(m) \mathrm{d}m}} {\alpha}
    \]
so the yield gives the fraction of the remnant population synthesized
and released in each element. This gives
\[
    \frac{\mathrm{d}(gZ)}{\mathrm{d}t} =
    (1-\alpha)Z\psi + \alpha\psi{p} -Z\psi + Z_{F}F - Z_{E}E
    \]
where the RHS terms are recycled material, new production, lock-up in
new stars, inflow, and outflow. Simplifying, we have
\[
    \frac{\mathrm{d}(gZ)}{\mathrm{d}t} =
    -\alpha{Z}\psi + \alpha\psi{p} + Z_{F}F - Z_{E}E
    \]
Using mass locked up in stars, $s$, as the independent variable:
\[
    s(t) = \alpha\int_{0}^{t} {\psi'(t')dt'}
    \]
\[
    \frac{\mathrm{d}s}{\mathrm{d}t} = \alpha\psi
    \]
so we get
\[
    \frac{\mathrm{d}g}{\mathrm{d}s} = \frac{F-E}{\alpha\psi} - 1
    \]
\[
    \frac{\mathrm{d}gZ}{\mathrm{d}s} = -Z + p +
    \frac{Z_{F}F}{\alpha\psi} -
    \frac{Z_{E}E}{\alpha\psi}
    \]
If we make a further assumption of a uniform wind, $Z_{E}=Z$, then
\[
    \frac{\mathrm{d}gZ}{\mathrm{d}s} = -Z\left( 1 + \frac{E}{\alpha\psi}\right) +
    p + \frac{Z_{F}F}{\alpha\psi}
    \]
and finally,
\[
    g\frac{\mathrm{d}Z}{\mathrm{d}s} =
    \frac{\mathrm{d}(gZ)}{\mathrm{d}s} - Z\frac{\mathrm{d}g}{\mathrm{d}s} =
    p + (Z_{F} - Z)\frac{F}{\alpha\psi}
    \]
\textbf{This is a basic equation of chemical synthesis}.
The \emph{simplest} model for chemical evolution assumes no
inflow or outflow, a homogeneous system without any spatial
differentiation of metallicity, zero initial metallicity, and
yields which are independent of composition. This is known as the
\textcolor{bred}{\textbf{Simple, one-zone model}}.

In the Simple model, we have
\[
    \frac{\mathrm{d}g}{\mathrm{d}t} = -\frac{\mathrm{d}s}{\mathrm{d}t}
    \]
\[
    g\frac{\mathrm{d}Z}{\mathrm{d}s} = -g\frac{\mathrm{d}Z}{\mathrm{d}g} = p
    \]

Solving for $Z(g)$, we get
\[
    \]
where $\mu$ is the gas fraction
\[
    \]
The average abundance in stars is
\[
    \]
which tends to $p$ as $\mu$ becomes small, i.e.\ the gas fraction becomes
small. For larger gas fractions, the mean $<Z>$ will be lower than the yield.
Note that this is the mean metallicity weighted by stellar mass, not by
stellar luminosity or by number (but these could easily be derived).

\subsection{Measuring stellar abundances}
Stellar abundances in individual stars measured from absorption features.
Hot stars only allow measurements of very few elements, while cool stars
(T\textless3500 K) are complicated due to very large numbers of lines and
prominence of molecular features. When lines are well-separated, abundance
determination is possible from measurements of the \textit{equivalent width}.
When lines are more crowded, usually approached by spectral synthesis.

In all cases, to measure abundances, need to know:
\begin{itemize}
    \item atmospheric parameters
        \begin{itemize}
            \item Temperatures ($T_{e}$) may be derived from colors
                (unless reddening is an issue); also often derived from
                excitation equilibrium: require lines of different
                excitation potential to give the same abundance (e.g. FeI lines)
            \item Surface gravity ($\log{g}$) can be derived from luminosities,
                temperatures and masses, hence distances are required.
                Alternatively, use ionization equilibrium: require lines
                of different ionization states to give the same abundance
                (e.g., FeI vs.\ FeII).
            \item microturbulence
        \end{itemize}
    \item atomic parameters
        \begin{itemize}
            \item Astrophysical $gf$ values: Atomic parameters are determined
                from lab experiments, but not always available at high quality.
                $gf$ values are derived from measurements of stars in which
                abundance is presumed to be known from other measurements.
        \end{itemize}
\end{itemize}
Abundances in unresolved populations are challenging, for the reasons
previously discussed (age-metallicity degeneracy, blending of different
features, etc.) Present day abundances in galaxies best measured through
gas abundances (to be discussed).

\subsubsection{Units for measuring abundances}
\begin{itemize}
    \item Stellar abundances: usually measured relative to the sun
        $[X/Y]$:
        \[ \log\left[\frac{(X/Y)}{(X/Y)_{\odot}}\right] \]
    \item $Z_{i}$: mass fraction of element $i$
    \item $\log(X/Y)$: log of ratio of mass fractions
    \item $12 + \log(Z_{i}/H)$: log of ratio of mass fraction to hydrogen,
        with 12 added (to keep the numbers positive?)
\end{itemize}

\subsubsection{Abundances as function of time}
The simple model predicts an increase of the metallicity with time.
The rate of increase depends on the SFR.
For example, if we \emph{assume} that the SFR is proportional to the
gas mass, $\mathrm{d}s/\mathrm{d}t = \omega{g}$, where $\omega = constant$,
we get:
\[
    z = \omega{t}
    \]
and the abundance is expected to increase linearly with time.

\subsubsection{Relative abundances of different elements}
Simple model predicts that, at any time, elements will be found in relative
abundances given by the ratio of their yields. This is true to some extent
for $\alpha$ elments, e.g.\ in the solar neighborhood\footnote{from
Bensby et al. 2003}. Most other elements go in lockstep with O (Fig. 11)

This is not true for Fe. This is plausibly explained for by a significant
fraction of Fe coming from SnIa, which has a time delay, and so the simple
model with instantaneous recycling doesn't hold. This is important to realize,
because many direct ``metallicity'' determinations are make from Fe liens.
If SnIa is the right explanation, one would expect to a constant ratio
of Fe/O for systems in which star formation proceeded rapidly (i.e.\ finished
before SnIa go off), but in systems with more extended start formation,
Fe/O would increase at higher O abundance (Page l figure 10).
Note that $\alpha/Fe$ traces timescales of extended star formation,
which is not necessarily the same as age (you could have a short epoch
of star formation at a late time to give alpha-enhancement even at younger
ages). However, very old populations would be expected to have alpha-enhanced
populations, e.g.\ as observed in globular clusters.
With the SnIa interpretation, $\alpha/Fe$ provides a clock. Unfortunately,
the calibration of the clock is not extremely well known because of the
unknown precise nature of SNIa projenitors. It is usually considered
to be of order 1 Gyr but there is very plausibly a range of SNIa ages.

The constancy of element abundance ratio holds for \emph{primary}
elements i.e. those that are produced from primordial abundance.
In addition some elements are secondary which means that some element must
previously exist in order for the secondary element to be created.
An example is N, which is produced during the CNO cycle
in greater abundance with a larger initial abundance of C.
This is observed (Fig. 12)

More generally, the constancy of element abundance ratios could fail if
the yields are themselves a function of metallicity. This may be
expected to occur, e.g.\ for elements that are significantly contributed by mass
loss, since mass loss may increase with increasing metallicity.

In reality, lots of different elements appear \emph{not} to go
in lockstep with one another, presumably because of the origin of different
elements from different processes and in different types of stars.
These variations may hold important clues for tracing the origin of stellar
systems. $\alpha/Fe$ traces duration of star formation (relative to
SnIa timescale). Other element ratios may trace variations in the IMF,
since different mass stars contribute different fractions of heavy
elements, or stochastic effects from a given IMF for star formation occurring
in small parcels (e.g.\ clusters). Gives rise to the idea of
\textit{chemical tagging} \mynotes{???}. Example: Local Group
dwarf galaxies and the Galactic halo\footnote{from Tolstoy et al. 2009}.
Could the halo come to be built up by dwarfs? Caution: just because
\emph{present-day} dwarfs don't match doesn't mean previously existing
ones don't. Note SDSS-III APOGEE survey.

Mass-metallicity relation here? \url{http://arxiv.org/abs/1211.3418}

\section{Gas}
Intrinsic gas characteristics:
\begin{itemize}
    \item temperature
    \item density
    \item composition
    \item ionization field
    \item total gas mass
\end{itemize}
Gas in galaxies exists in multiple phases:
\begin{description}[labelwidth=20em, ]
    \item [molecular clouds] T $\lesssim$ 100 K
    \item [cold, neutral (atomic) medium (CNM)] T $\lesssim$ 100 K
    \item [warm, neutral (atomic) medium (WNM)] T $\lesssim$ 8000 K
    \item [warm, ionized medium (WIM/DIG)] T $\gtrsim$ 8000 K
        (DIG = diffuse ionized medium)
    \item [hot, ionized medium (HIM)] T $ > 10^{5}$ K
    \item [denser, ionized gas] $n \approx x$
        \begin{itemize}
            \item HII regions
            \item AGN
            \item SN remnants
            \item PNe
        \end{itemize}
\end{description}
CNM/WNM/WIM/HIM perhaps roughly in pressure equilibrium, but there are
other support mechanisms such as turbulence and magnetic fields.
If they are in pressure equilibrium, higher temperature components
have correspondingly lower densities.

\subsection{Observing and characterizing the gas}
Two general methods:
\paragraph{1. Emission}
\subparagraph{Atomic lines}
\begin{description}[labelwidth=5em, leftmargin=8em ]
    \item [Permitted] In low temperature gas (T = 10000 K $\sim$ 1eV),
        the energy difference to excited states is typically too high
        for excitations to occur from collisions. An exception is
        H lines from recombination; these exist from UV through radio
        wavelengths.
    \item [``Forbidden''] Low transition probabilities; important in low
        density gas because collisions are less frequent, allowing time
        for radiative transitions to occur before collisional de-excitation.
        Wavelengths are typically in the optical and UV regimes. Note that
        many of these are \textit{doublets because of multiple angular momentum
        levels}. Most prominent in optical: [OIII], [NII], [SII], [OII], [OI]
    \item [Fine-structure] transitions involving spin alignment with
        orbital alignment. Typically FIR wavelengths
    \item [Hyperfine-structure] transitions involving electron spin
        alignment with nuclear spin. Radio wavelengths.
\end{description}
\subparagraph{Molecular lines}
\begin{itemize}
    \item Vibrational transitions, typically IR wavelengths
    \item Rotational transitions, typically mm wavelengths. However,
        rotational transitions only strong for non-symmetric molecules
        that have a dipole moment (not H$_{2}$, although it can have
        some IR features if shock heated). CO emission is most prominent.
    \item \href{http://astronomy.nmsu.edu/holtz/a555/resources/irism.gif}
        {example spectrum}.
\end{itemize}
Emission in many cases is proportional to density \emph{squared}.
If the excitation mechanism is collisional (i.e., requires two particles),
the observational quantity is the \textit{emission measure}:
\[
    EM \propto \int{n_{e}^{2}\mathrm{d}\ell}
    \]

\paragraph{2. Absorption}
\subparagraph{Atomic lines}
Typically from ground state, e.g.\ H Lyman transitions, Na 5890, CaII
most prominent in optical

\subparagraph{Molecular lines}
Also mostly UV, e.g.\ H$_{2}$
\href{http://astronomy.nmsu.edu/holtz/a555/resources/uvabsorp.htm}
{absorption}

Absorption is generally more sensitive, but it does require a bright
background source, e.g.\ UV light from quasars. Absorption is
\emph{linearly} proportional to density. The observational quantity is
\textit{column density}:
\[
    CD \propto \int{n\mathrm{d}\ell}
    \]

\subsection{Heating and cooling}
What determines gas temperature?
\subsubsection{Heating}
\begin{itemize}
    \item Radiative heating (both via ionization and via heating through dust)
    \item Mechanical heating (shocks, jets, etc.)
    \item Gravitational heating
    \item Cosmic rays
\end{itemize}
\subsubsection{Cooling}
\begin{itemize}
    \item Bremsstrahlung (free-free)
    \item Line emission from heavier elements, molecules
    \item Cooling curves and contributions
        \href{http://adsabs.harvard.edu/abs/1993ApJS...88..253S}
        {(Sutherland \& Dopita, 1983)}
\end{itemize}

\subsection{Cold gas: HI measurements with 21 cm line}
\begin{itemize}
    \item Traces neutral medium
    \item From spin-flip transition: very rare (10 Myr) but H is very
        numerous. Relative number in two energy states set by collisional
        equilibrium and proportional to statistical weights of two levels
        (3:1); upper levels are populated from (microwave background photons,
        collisional saturation?)
    \item HI flux at a given location gives column density (atoms/surface
        area):
        \[
        n_{HI}\; [\mathrm{cm}^{-2}] \approx 1.82\times10^{18}\int{
            \left[\frac{T_{b}(\nu)}{\mathrm{K}}\right]\mathrm{d}
            \left[\frac{v}{\mathrm{km\;s}^{-1}}\right]}
        \]
        Typical column densities in galaxies are
        $\sim 10^{20}$ cm$^{-2}$
    \item Total HI flux (integrated over entire galaxy), along with distance,
        gives total HI mass:
        \[
            \frac{M_{HI}}{M_{\odot}} = 2.36\times10^{5}\left(
            \frac{D}{\mathrm{Mpc}}\right)^{2}F_{HI}
            \]
        for $F_{HI}$ in Jy km s$^{-1}$.
    \item 21-cm single dish observations have low spatial resolution.
        Higher resolution possibly with interferometric observations
        (e.g.\ VLA) but information on smooth distribution/total mass can
        be lost.
    \item In absorption, HI can be observed to very low column densities
        ($\sim10^{13}$ cm$^{2}$), but only for
        the Lyman series. Primary tool for studying the IGM:
        \begin{itemize}
            \item Lyman alpha clouds
            \item epoch of reionization
            \item quasar absorption line systems
        \end{itemize}
    \item Typical column densities are $10^{13} - 10^{21}$ cm$^{-2}$
    \item Total mass in HI can be a significant fraction of total stellar
        mass, increasingly important in lower luminosity galaxies.
    \item Temperature might be inferred from width of line, but need to
        be careful about turbulence, cloud structure, etc. Estimate can
        also be made by comparing emission flux with absorption if you
        have a background source (radio galaxy).
\end{itemize}

\subsection{Molecular gas}
\begin{itemize}
    \item Most dominant molecule is $H_{2}$, but since this has no net dipole moment,
        it doesn't have strong rotational transition emission.
        There are some mid-IR lines from shock-heated $H_{2}$
    \item $H_2$ can be observed in absorption in the far UV
    \item Molecular gas in emission most often traced by CO. However, relating
        CO emission strength to total amount of molecular gas is not
        necessarily trivial. The X (CO) factor
        relates observed strength of CO to $H_{2}$ column density:
        \[
            X \equiv \frac{
                N(H_{2})/\mathrm{cm}^{-2}}{
                I_{CO}/\mathrm{km\; s}^{-1}}
            \sim 2 \times 10^{20}
            \]
        However, there are many issues with this, and almost certainly
        depends on metallicity.
    \item Overall mass contribution\ldots less than atomic, but significant?
\end{itemize}

\subsection{Warm ionized gas}
Also known as ``Diffuse Ionized Gas'', or ``DIG''.
\begin{itemize}
    \item Visible in H$\alpha$
    \item Heating source likely photoionization, visible through
        recombination.
    \item Hard to estimate masses
    \item Pulsar dispersion gives measurement of $\int{n_{e}\mathrm{d}\ell}$,
        which can help a lot.
\end{itemize}

\subsection{Hot gas}
\begin{itemize}
    \item If ``very'' hot \mynotes{(number please?)}, detectable in X rays:
        thermal Bremstrahlung (free-free) cutoff frequency decreases with
        increasing temperature.
    \item  At lower temperatures (T $\sim$ 10$^{5}$ K), very hard to detect:
        soft X rays are absorbed by neutral hydrogen.
    \item Detectable in absorption by highly ionized metals,
        e.g.\ OVI, CIV, NV.
    \item Very hard to determine total mass, but mass may be
        very significant, just from cosmological baryon fraction \mynotes{???}
\end{itemize}

\subsection{Denser ionized gas}
\begin{itemize}
    \item Easiest gas to observe in galaxies.
\end{itemize}
\subsection{Chemical evolution and abundances from gas}
\subsection{Distribution of gas within spirals}
Neutral gas primarily observed in blue sequence galaxies. HI distribution
usually extends beyond optical distribution of starlight, while molecular is
more centrally concentrated. In some galaxies, CO traces H$\alpha$ well, but
not in others. Distribution of HI is very flat compared to distribution of
molecular gas (and stars). Extended hot gas is observed in most galaxies.

\subsection{Gas as a kinematic tracer}
\mynotes{(This section not in online notes anymore\ldots)}

Measuring galaxy rotation - both optically and using HI (both spatially
resolved, e.g.\ VLA, and unresolved data).
\begin{itemize}
    \item Swaters et al. 2002
    \item Velocity-coded color (from NRAO)
    \item Spider diagrams
    \item PV diagrams: note spread of velocity at any given position;
        location of peak velocity may not give the \emph{maximum}
        velocity at that position
\end{itemize}
Rotation is often characterized by $v_{max}$: maximum of rotation curve
(sometimes hard to measure).  From unresolved HI profiles, rotation is
characterized by line widths, e.g.\ $W_{50}, W_{20}$ (width at 50\% and 20\% of
the peak line flux, respectively).


\subsection{Interactions between gas and stars}
\subsubsection{Star formation}
Due to the lack of detailed knowledge of star formation, and lack of sufficient
resolution, star formation is parameterized within models, e.g.\ based on
stability arguments given gas densities and temperatures. Parameterization
is tuned by need to match characteristic observed relations of observed star
formation.

Star formation parameterizations: want to relate SFR to
some observed property.
\begin{itemize}
    \item Measure current star formation rate. Three main methods:
        \begin{enumerate}
            \item $H\alpha$ emission (but note possible issues with assumed
                IMF, leakage)
            \item UV as more direct SFR indicator (but associated
                uncertainty from extinction)
            \item IR emission from dust heated by young stars
        \end{enumerate}
        Typical galaxy SFRs are 0-100 M$_{\odot}$ yr$^{-1}$.
        The \textit{specific star formation rate (SSFR)} is
        the SFR per unit \textit{mass} (usually stellar).
        A typical galaxy has SSFR $\sim 10^{-10}$ yr$^{-1}$.
        For nearby resolved galaxies, we can
        also consider the surface density of star formation.
    \item Relate star formation with gas properties: Simplest quantity is
        gas density, observed as gas surface density. Can consider density
        of neutral gas, molecular gas, or a combination of all phases.
        Molecular might be considered to be the most relevant, but it is
        difficult to observe (also, recall profiles of H$\alpha$ vs gas
        density shown earlier).  Also, molecular gas forms from neutral
        gas.
    \item Schmidt law:
        \[
            SF \propto \rho^{n} \propto \Sigma^{N}
            \]
        (where $\Sigma^{N}$ is relevant as it is the observable).  Schmidt
        (1959) suggested $N \sim 2$ based on observations of HI and stars
        in Milky way. Note that a fixed star formation efficiency would
        have SFR proportional to gas mass.
    \item Measured relations show a more complex behavior (Kennicutt Fig
        8). Slope in regions of high gas density roughly comparable between
        galaxies (Schmidt law), but different galaxies have different
        thresholds for SF. Molecular gas may be more relevant above
        threshold, see e.g. Mo Fig 9.4
    \item Kennicutt SF law: has a threshold density, then a power law
        dependence
\end{itemize}
several ``scenarios'' might give observed trends: consider two possible
qualitative physical mechanisms behind observed relations.
\begin{enumerate}
    \item star formation related to growth of instabilities suggests
        \[
            \rho_{SFR} \propto \frac{\rho_{gas}}{(G\rho_{gas})^{-0.5}} \propto
            \rho^{1.5}
            \]
        we observe surface density, not density. For \emph{fixed} scale height of
        gas, these are directly proportional, but not for \emph{variable} scale
        height
    \item Alternatively, perhaps SF is related to the local dynamical
        timescale (e.g., SF triggered by passage through spiral arms). Then
        \[
            \Sigma_{SFR}
            \propto \frac{\Sigma_{gas}}{(\tau_{dyn})}
            \propto \Sigma\Omega_{gas}
            \]
\end{enumerate}
Both give comparable SFRs (Kennicutt Figure 7).

For thresholds, threshold density might be expected from consideration of
stability of disks - below critical density, disks are stable against large
scale perturbations; for a thin isothermal gas disk,
$\Sigma_{c}(r) = \alpha \kappa(r) \sigma / 3.36$G,
where $\sigma$ is the gas velocity dispersion, and $\kappa$ is the
epicyclic frequency, which is related to the rotation curve. The critical
density depends on rotation profile and velocity dispersion. Density of gas
in spirals matches reasonably well the profile of critical density (Figs 10
\& 11). In this model, the threshold surface density is a function of radius
(Fig 14), and varies from galaxy to galaxy. Suggests threshold as f(r) then
Schmidt law with $N\sim 1.3$ - with several possible systematic
uncertainties (extinction, other factors which vary with r). However,
behavior near the threshold is significantly steeper - and most regions of
most galaxies appear to be near the threshhold.

there are other possible explanations as well, e.g. ability to develop
molecular clouds at low densities..

see Leroy et al 2008 for nice discussion of lots of different
parameterizations
\begin{itemize}
    \item Focus on star formation efficiency (SFE), i.e. SFR / gas mass
    \item Find roughly constant SFE in molecular dominated regions, with
        declining SFE in outer regions (Figure 1). Note that this suggests
        there is not a simple threshold.
    \item Simple SF ``laws'' do not appear to well match observations (e.g.
        Figure 5 and Figure 7)
    \item Perhaps the issue has to do with conditions for molecular cloud
        formation
    \item Note issue of observing surface mass density may fall short if
        spatial resolution is insufficient to resolve real structure! (see,
        e.g. Leroy et al 2013a)
    \item Leroy et al 2013b suggest that there are measureable deviations
        of SFE in molecular dominated regions, that may be related to
        variations of CO/H
\end{itemize}
Simulations often adopt some prescription related to one of these
mechanisms

global star formation law. Star forming galaxies show range of current star
formation rates from $\sim$ 1 Msun/yr to hundreds of Msun/yr; what sets the
SFR? Results from a sample of normal galaxies (Kennicutt Fig 2), starburst
galaxies(Kennicutt Fig 5), and all galaxies combined (Fig 6). Remarkably
tight relation is seen over wide range of densities and star formation
rates.

\subsubsection{Supernovae and mass loss from stars}
supernovae clear out ``superbubbles'' in the ISM; it is possible that these
superbubbles even propagate to the vertical edges of the disks and allow
for mass outflow from the disk into the halo. If this is the case, one
might expect this gas to eventually fall back down onto the disk, leading
the the ideas of galactic fountains and chimneys.
\begin{itemize}
    \item Note active study/debate: extraplanar gas arising from disk
        (outflow/inflow) or from IGM (inflow), e.g. Milky Way high velocity
        clouds (HVCs), lagging gas in galaxies
    \item Note implication for chemical evolution
\end{itemize}
Significant galactic winds (outflows) are observed in starburst galaxies,
e.g. M82. These may be more common at higher redshift.

also possible that the shocks from supernovae act to compress gas clouds,
possibly trigering further star formation in nearby regions.

These processes are important for understanding:
\begin{itemize}
    \item feedback mechanisms by which star formation may be self-regulating
    \item chemical evolution of galactic disks
    \item the existence of gas halos around galaxies
    \item the escape of ionizing radiation from galactic disks.
    \item overall issue of ``gas cycle'' in galaxies, e.g. Fig 2 from Lilly
        et al 2013
\end{itemize}

\newpage
\section{Dust}
\subsection{General importance of dust}
\begin{itemize}
    \item Can obscure light from stars/gas: affects inferences on
        populations, abundances, etc.
    \item Emits radiation which can be observed (mostly IR) to probue
        obscured emission.
    \item Important for ehemistry of ISM, e.g.\ created of H$_{2}$.
    \item Important for heating and cooling of ISM
    \item Probably \emph{not} important as a significant mass component;
        likely only a fraction of a percent of ISM by mass
\end{itemize}

\subsection{Extinction}
Foreground and internal. Extinction laws and their variation.

Extinction of light is characterized by an \textbf{extinction curve}, which gives
the relative amount of light lost as a function of wavelength. Note that
extinction curves are often in magnitudes (logarithmic). Generally,
shorter wavelengths are attenuated more, leading to the phenomenon of
reddening. This implies that the dust grain sizes are comparable to the
wavelength of the light. \textcolor{bred}{Existence of the 2200\AA{} bump.}
The total amount of extinction is proportional to the amount of reddening,
e.g.\ $R_{V} \equiv A_{V}/E(B-V)$.
More generally, $R_{\lambda} \equiv A_{\lambda}/E(B-V)$.

Variations in the extinction curve exist, both within the Milky Way
and in nearby galaxies\footnote{Gordon et al. 2003}. This variation is
more significant in the UV. Cardelli, Clayton, and Mathis suggest that
extinction curves can be parameterized by a single parameter $R_{V}$.
Probably related to dust properties, e.g.\ grain size.

Extinction has both scattering and absorption components. Scattering
leads to, e.g., reflection nebula and diffuse light. Measuring scattering
suggests optical albedo $\sim$ 0.5, so roughly equal amounts of scattering
and absorption. Absorption leads to heating and re-emission.

Discussions of extinction curves refer to the extinction of background light by
foreground dust (i.e. a foreground screen). Integrated galaxy light likely does
not fall into this scenario (apart from foreground Milky Way extinction).
Remember Burstein \& Heiles and Sclegel, Fink beiner \& Davis for MW maps.
Scattering is significant and leads to ``saturation''\footnote{Witt \&
Gordon, 2003}.

\subsubsection{Amount of gas, dust-to-gas ratio, and transparency of galaxies}
The amount of extinction seems to be well correlated with the amount of
neutral H
\[
    N(H) = 5.8\times10^{21}E(B-V)\quad
    \left[\mathrm{cm}^{-2}\;\mathrm{mag}^{-1}\right]
\]\[
    A_{V} = 5.3\times10^{-22}N_{H}\quad
    \left[\mathrm{mag}\;\mathrm{cm}^{2}\right]
\]
This suggests that dense (inner) regions of ISM are likely to be optically
thick, but not low density ones. Confirmed by observations of ``overlapping''
galaxies.

\subsection{Emission from dust}
\begin{itemize}
    \item Predominant heating is from starlight (some collisional
        heating in dense clouds)
    \item Amount of energy absorbed depends on flux, size, and albedo.
        Amount of energy radiated depends on temperature, size, and emissivity.
    \item Dust particles are not ideal blackbodies. In addition, they may not
        be in temperature equilibrium, especially small particles.
    \item MW dust
    \item Dust emits in far IR, e.g. MW emission and models. Note sharper
        emission features from PAHs around 10 microns.
    \item Dust emission traces star formation; luminous IR galaxies
        (LIRGs, ULIRGs, 10 to 100x IR emission than typical galaxy).
    \item Implications for higher redshift: sub-mm observations and
        the negative K-correction
\end{itemize}

\subsection{Composition}
\begin{itemize}
    \item Solids have broad or non-existent absorption features, so abundances
        from absorption are difficult or impossible.
    \item Some inferences from interstellar gas abundances:
        if stellar abundance ratios are assumed, than observed gas abundances
        show depletion (from Draine), e.g. of C, Mg, Si, Fe at a level
        greater than 50\%.
    \item Still can't identify uniquely what molecules the dust is in.
        Possibilities:
        \begin{itemize}
            \item silicates
            \item oxides
            \item carbon solids
            \item hydrocarbons, e.g. PAHs (polycyclic aromatic hydrocarbons)
            \item carbides
            \item metallic Fe
        \end{itemize}
    \item Some absorption features are observed:
        \begin{itemize}
            \item 2200\AA{} feature: originally thought to be graphite,
                now perhaps PAHs
            \item 9.7 and 18 microns: silicates? (from Draine)
            \item diffuese interstellar bands (DIBs) (from Draine)
        \end{itemize}
    \item As stated above, emission features are observed in IR from PAHs
\end{itemize}
Overall, plausible to expect that the \textbf{dust-to-gas ratio might be
a function of metallicity}.

\subsection{Creation/destruction}
Dust is apparently difficult to create in the ISM. It is thought to be
created in envelopes of cool stars $\rightarrow$ \textbf{the amount of dust
may also be a function of star formation.}

\newpage
\section{Central black holes}
(some references: Peterson, ``An Introduction to Active Galactic Nuclei'',
Ferrarese \& Ford (2005)).

Galaxies harbor central supermassive black holes: how do we know?
\subsection{AGN as indicators of central black holes}
Some fraction nearby galaxies who ``active'' nuclei:
\begin{itemize}
    \item Optical emission lines
    \item Power-law continua spanning most of the EM spectrum.
    \item Radio sources, often with jets
    \item AGN review\footnote{see book by Peterson, 1997}\footnote{see
        Whittle (UVa) lecture notes}
        \begin{itemize}
            \item Seyfert galaxies: Optical emission lines. Types 1
                (broad lines) and 2 (narrow lines). Distinguishable from
                HII regions, etc. from \textbf{emission line ratios}.
            \item LINERS: weaker lines, lower ionization
            \item BL Lac objects
            \item Quasars and QSOs
            \item Radio galaxies, e.g.\ Cen A, M87. Emission is from
                synchrotron emission. Energy source may be the same,
                but don't generally directly see nuclear activity.
                Fanaroff-Riley (FR) types 1 (low radio luminosity,
                edge-darkened) and 2 (high radio luminosity, edge-brightened).
            \item See tables in section 4b of Whittle notes with some spectra.
        \end{itemize}
    \item Frequency of Seyfert phenomenon 1-2\% but LINERS much more common;
        perhaps tens of percents of local galaxies show activity at some
        level.
\end{itemize}
Generally considered to be powered by accretion onto a nuclear black hole.
Arguments for the existence of black holes:
\begin{itemize}
    \item Luminosity: can be as high as $\sim10^{45-58}$ erg s$^{-1}$
        for some AGN. \textbf{Eddington Luminosity:}
        \[
            L_{e} = \frac{4{\pi}GMm_{p}c}{\sigma_{T}} =
            1.3\times10^{46}\frac{M}{10^{8}M_{\odot}}
            \left[\mathrm{erg\;s}^{-1}\right]
        \]
    \item Non-stellar continua
    \item Timing arguments. AGN show rapid variability with timescales of
        days (optical/radio) to hours (X-ray): implies small objects
        (small fraction of a pc).
\end{itemize}
\textbf{Overall physical picture:}
different active nucleii phenomena all arise from
a common physical scenario. Some \emph{intrinsic} differences give rise
to different phenomena, whereas some differences arise from viewing angle.
\begin{itemize}
    \item Unified model: several main regions:
        \begin{itemize}
            \item Accretion disk
            \item Broad line region (BLR)
            \item Narrow line regions (NLR)
            \item Dust torus
        \end{itemize}
    \item Alternative model\footnote{Elvis (2000)}
    \item Viewing angle differences: Unified model (and another)
    \item Some possible intrinsic causes of variation between types
        \begin{itemize}
            \item Accretion rates and accretion rate differences
            \item Differences from variations in gas suppy (quantity and/or
                efficiency of infall)
            \item Differences from different black hole masses?
        \end{itemize}
\end{itemize}
\subsection{Black holes in non-active galaxies}
It appears that most galaxies, even inactive ones, harbor central
supermassive black holes. This is expected at some level from number density
of quasars at higher redshift.
Detection of SMBHs\footnote{Ferrarese \& Ford}\footnote{Whittle (UVa) lecture}:
\begin{itemize}
    \item \textbf{Sphere of influence} (region where velocity generated by
        central mass is comparable to or larger than velocity determined
        from overall galaxy potential) is very small:
        \[
            r_{h} \sim \frac{GM_{BH}}{\sigma^{2}} \sim
            11.2\frac{M_{BH}}{10^{8}M_{\odot}}\left(
            \frac{\sigma}{200\mathrm{km\;s}^{-1}}\right)^{-2}\quad
            \left[\mathrm{pc}\right]
            \]
    \item MW galactic center (e.g. UCLA group video)
    \item Other galaxies: BHs inferred but strictly speaking, not absolutley
        required\footnote{Ferrarese \& Ford 2005}.
\end{itemize}
If they have BHs, why aren't they active? Gas supply? Advection dominated
accretion flow (ADAF)?

\subsection{Measuring black hole masses}
\begin{itemize}
    \item Proper motions
    \item Stellar and gas kinematics
    \item Gas kinematics
    \item Broad Fe K$\alpha$ (6.4 KeV X-ray line)
    \item Reverberation mapping
\end{itemize}
BH masses have now been made for several dozen galaxies. Initially, correlation
seen between BH mass and galaxy luminosity. Relation between BH mass and
central velocity dispersion significantly tighter. Origin not well understood.
M$_{BH}-\sigma$ relation (or M$_{spheroid}$).

\subsection{Importance of central black holes}

\newpage
\section{Galaxy spectral energy distributions}

\newpage
\section{Dark matter and galaxy masses}
Dark matter dominates mass in the universe. How much is there? How is
it distributed, both among and within galaxies? To what extent does it
regulate galaxy properties?

Note from cosmology: we expect
\begin{itemize}
    \item $\Omega_{m} \sim 0.27$
    \item $\Omega_{b} \sim 0.047$
\end{itemize}

\subsection{Observations of dark matter in galaxies}
The amount of dark matter in a galaxy is often characterized by the
mass-to-light ratio (M/L). \textbf{Note that there is a distinction
between \emph{stellar} M/L and \emph{total} M/L.}

\subsubsection{Galaxy rotation curves}
Rotation curves (RCs) probe dark matter content and profile. The general
observation of flat profiles (sometimes reffered to as ``isothermal'' profiles)
implies that $M \propto r$ and $\rho \propto r^{-2}$.

Recall that the shape of the rotation curve is moderately well-correlated
with the luminosity, in the sense that one gets steeper RCs for lower L
galaxies\footnote{Persic and Salucci, figure 4}. However there is significant
variation at each luminosity. Note that not all RCs are flat; there is a
variety of shapes.

Comparison of the luminosity profile with the rotation profile shows that
the \emph{luminous} mass alone is unable to produce the rotation profile,
therefore dark matter is required. The fraction of ``dark mass'' increases
with radius, and also increases with fixed radius with decreasing luminosity.

\paragraph{Dark matter in disk galaxies}
Given the luminosity profile, the shape of the RC shows that DM has to be
present, because there's no way to get high enough stellar mass-to-light ratios
in the outer region.  However, determining the amount of dark matter in disk
galaxies requires knowledge of the M/L of the stellar population, which is
unknown (recall that M/L depends on the SFH, especially the IMF).  The stellar
M/L can't be derived from the RC because the density distribution of the DM is
unknown.  It's possible that the presence of dust makes this more challenging
to infer.

A common assumption is the \textbf{``maximum-disk hypothesis''}, in which
the innermost parts of the RC are assumed to be driven by the luminous mass
alone. These regions then give a M/L, which is assumed to be constant as
a function of radius. Combined with the outer roation curve, mass profiles
for the luminous and dark components are derived. However, data is often
degenerate in different ratios of disk to DM\footnote{see van der Kruit
example, from van der Kruit Galaxy Masses 2009 talk}.
The applicability of this hypothesis is debateable. For example,
for an exponential disk:
\[
    V_{max} \propto \sqrt{\frac{M}{r_{d}}}
    \]
so we might expect a trend at fixed L (if we extrapolate from M to L)
with scale length: not observed in Tully-Fisher relation.

The \emph{total} M/L in disk galaxies
has been found to be $\sim$ 10-50 in regions probed by RCs.

\paragraph{Dark matter in LSB galaxies}
The RCs for LSB galaxies confirm a high M/L. Compare two galaxies at
identical positions in TF relation and see that the LSB galaxy is much
more DM dominated\footnote{from de Blok \& McGaugh}. This suggests that
RCs are correlated not only with luminosity, but also with surface
brightness.

LSB RCs are of particular interest because they may provide the most
direct probes of DM distribution. The shapes of their RCs are currently under
active debate as to what they imply for DM distribution.
Numerical DM simulations predict a ``cuspy'' distribution of DM,
with $\rho_{DM} \propto r^{-1}$ in the inner regions, e.g.\ the
Navarro-Frenk-White (NFW) profile:
\[
    \rho = \frac{\rho_{0}}{\frac{r}{r_{s}}\left(1+\frac{r}{r_{s}}\right)^{2}}
    \]
family of profiles characterized by the concentration $r_{200}/r_{s}$.

However, there may also be an effect of baryons on DM (adiabatic compression,
or expansion?), since in inner regions, gravity from baryons may not be
negligible. Some observations\footnote{Swaters et al 2003} suggest more of
a core. The interpretation of relatively low velocities in LSB centers can
be complicated by a variety of effects, such as projection, beam smearing,
non-circular velocities, etc. The ``core vs.\ cusp'' debate is still active.

DM halos are expected to extend far beyond what we can see from physical
tracers (even HI gas) within the galaxy embedded in their centers.
Typical virial radii (from simulations) are expected to be several hundred
kpc. So total masses are not well probed by rotation curves.

\paragraph{Dark matter in ellipticals}
Theoretically, the presnece of DM in ellipticals is exptected. However, it
is harder to observe than in spirals because of the lack of an ``easy''
dynamical tracer, like a rotating component. THey also have higher SB,
so there is a larger contribution by baryones to kinematics\footnote{see
review by Gerhard 2006}.

For non-rotating components, the velocity dispersion profiel is insufficient
to measure mases without knowledge of \textit{velosity ellipsoid},
parameterized by \textit{velocity anisotropy}:
\[
    \beta \equiv 1 - \frac{<v_{\theta}^{2}>}{<v_{r}^{2}>}
    \]
Or more generally, distribution of orbits\footnote{e.g.\ velocity dispersion
profiles, from van der Marel (1994)}. Inclination/projection also plays a
role\ldots

Potential tracers:
\begin{itemize}
    \item Velocity distribution of stars (integrated light). Need to use
        line profile shapes, 2D distributions (integral field)
    \item Planetary nebulae
    \item Globular clusters
    \item Gas disks (rare, but do exist)
\end{itemize}
Note that even for non-rotating systems, distribution of mass often described
by the \textit{circular velocity}:
\[
    v_{c} \equiv \sqrt{\frac{GM(r)}{r}}
    \]
From stellar velocity distributions, most ellipticals show evidence of
flat circular velocity curves, indicating the presence of DM in the inner
regions, but not always a lot.

Modelling suggests that a simple estimate\footnote{Cappellari et al. (2006)}
\[
    M/L \sim 5\frac{\sigma_{e}^{2}r_{e}}{GL}
    \]
gives a reasonable rough estimate; c.f virial relation \mynotes{(???)}
and relation for
central velocity dispersion, e.g.\ for isothermal sphere.
There has been some debate about PN observations at larger radii, with some
observations gsuggesting declining circular velocities. However, understanding
of velocity anisotropy is important to the conclusion.

Indpendent mass estimate proveded by X-ray halos under the sasumption of
hydrostatic equilibrium (requires measuring densities and temperatures),
which, when they exist, generally indicate the presence of dark matter
halos\footnote{from Sato et al. 2000}
\[
    M(r) = -\frac{k_{B}T(r)r}{{\mu}m_{p}G}
    \left[\frac{\mathrm{d}\ln{\rho}}{\mathrm{d}\ln{r}} +
    \frac{\mathrm{d}\ln{T}}{\mathrm{d}\ln{r}}\right]
    \]
Bottom line: ellipcals probably have dark matter.

\subsubsection{Gravitational lensing}
Another way of probing mass. There is a distinction between \textit{strong}
and \textit{weak} lensing\footnote{both from Treu talk at Galaxy Masses 2009}.
Weak lensing is measured statistically.

\subsection{Total masses of galaxies}
\subsection{Halo abundance matching}

\end{document}
